\documentclass[12pt,a4paper]{article}
%-------------------------------------------
%---Packages--------------------------------
%-------------------------------------------
\usepackage[utf8]{inputenc}
%\usepackage[T1]{fontenc}
%\usepackage{txfonts}
\usepackage{amsmath}
\usepackage{amsthm}
\usepackage{amsfonts}
\usepackage{array}
\usepackage{amssymb}
\usepackage{blindtext}
\usepackage{caption}
\usepackage{color}
\usepackage{csquotes}	    %
\usepackage{enumitem}	    %pour mieux bosser avec les listes. ajoute option label
\usepackage[yyyymmdd]{datetime}        %pour définir date custom
\usepackage{etaremune}    
\usepackage{environ}
\usepackage{fancybox}
\usepackage{fancyhdr} 	    % Custom headers and footers
\usepackage{fancyref}
%\usepackage{float}
\usepackage{floatrow}       %float and floatrow can't be together...
\usepackage{gensymb}
\usepackage{graphicx}
\usepackage[colorlinks=true, linkcolor=purple, citecolor=cyan]{hyperref}
\usepackage{footnotebackref}
\usepackage{lipsum}
\usepackage{mathtools}
\usepackage{multicol}	    %gérer plusieurs colonnes
\usepackage{setspace}
\usepackage{subcaption}
\usepackage{todonotes}	    %Bonne gestion des TODOs
%TODO commenté pour tester l'utilité... à voir% \usepackage[tc]{titlepic}      %Permet de mettre une image en page de garde
\usepackage{tikz}	    % Pour outil de dessin puissant 
\usepackage{ulem}	    %underline sur plusieurs lignes (avec \uline{})
\usepackage{vmargin} 	    %gestion des marges, avec dans l'ordre : gauche, haut, droit, bas, en-tête, entre en-tête et texte, bas de page, hauteur entre bas de page et texte 
\usepackage{wrapfig}
\usepackage{xcolor}
\usepackage{xparse}                    %Pour utiliser NewDocumentCommand et des arguments 'mmooo'
%\usepackage{fullpage} 	    %supprime toutes les marges allouées aux notes, aussi en haut et en bas

%\ExplSyntaxOn
\pagestyle{fancyplain}	    %Makes all pages in the document conform to the custom headers and footers

%-------------------------------------------
%---Document Commands-----------------------
%---------------------------{----------------
\NewDocumentCommand{\framecolorbox}{oommm}
 {% #1 = width (optional)
  % #2 = inner alignment (optional)
  % #3 = frame color
  % #4 = background color
  % #5 = text
  \IfValueTF{#1}%
   {\IfValueTF{#2}%
    {\fcolorbox{#3}{#4}{\makebox[#1][#2]{#5}}}%
    {\fcolorbox{#3}{#4}{\makebox[#1]{#5}}}%
   }%
   {\fcolorbox{#3}{#4}{#5}}%
 }%
%------------------------------------------------
%------------------ENGLISH----------------------
%----------------------------------------------

\NewDocumentCommand{\epflTitle}{mO{Olivier Cloux}O{\today}O{Notes de Cours en}D<>{../../Common}}%Arguments : Matière, Auteur, Date, Titre du doc
{
\begin{titlepage}
    \vspace*{\fill}
    \begin{center}
        \normalfont \normalsize 
        \textsc{Ecole Polytechnique Fédérale de Lausanne} \\ [25pt] % Your university, school and/or department name(s)
        \textsc{#4} %Titre du doc
        \\ [0.4 pt]
        \horrule{0.5pt} \\[0.4cm] % Thin top horizontal rule
        \huge #1 \\ % Matière
        \horrule{2pt} \\[0.5cm] % Thick bottom horizontal rule
        \includegraphics[width=8cm]{#5/EPFL_logo}
        ~\\[0.5 cm]
        \small\textsc{#2}\\[0.4cm]
        \small\textsc{#3}\\
        ~\\
        ~\\
        \includegraphics[scale=0.5]{#5/creativeCommons}
    \end{center}
    \vspace*{\fill}
\end{titlepage}
}


%-------------------------------------------
%-------------MATH NEW COMMANDS-------------
%-------------------------------------------
\newcommand{\somme}[2]{\ensuremath{\sum\limits_{#2}^{#1}}}
\newcommand{\produit}[2]{\ensuremath{\prod\limits_{#2}^{#1}}}
\newcommand{\limite}{\lim\limits_}
\newcommand{\llimite}[3]{\limite{\substack{#1 \\ #2}}\left(#3\right)}	%limites à deux condiitons
\newcommand{\et}{\mbox{ et }}
\newcommand{\deriv}[1]{\ensuremath{\, \mathrm d #1}}	%sigle dx, dt,dy... des dérivées/intégrales
%\newcommand{\fx}{\ensuremath{f'(\textbf{x}_0 + h}}
\newcommand{\ninf}{\ensuremath{n \to \infty}}	       %pour les limites : n tend vers l'infini
\newcommand{\xinf}{\ensuremath{x \to \infty}}	       %pour les limites : x tend vers l'infini
\newcommand{\infint}{\ensuremath{\int_{-\infty}^{\infty}}}
\newcommand{\xo}{\ensuremath{x \to 0}}									%x to 0
\newcommand{\no}{\ensuremath{n \to 0}}									%n zéro
\newcommand{\xx}{\ensuremath{x \to x}}									%x to x
\newcommand{\Xo}{\ensuremath{x_0}}										%x zéro
\newcommand{\X}{\ensuremath{\mathbf{X}} }
\newcommand{\A}{\ensuremath{\mathbf{A}} }
\newcommand{\R}{\ensuremath{\mathbb{R}} }								%ensemble de R
\newcommand{\rn}{\ensuremath{\mathbb{R}^n} } 							%ensemble de R de taille n
\newcommand{\Rm}{\ensuremath{\mathbb{R}^m} }  							%ensemble de R de taille m
\newcommand{\C}{\ensuremath{\mathbb{C}} } 		
\newcommand{\N}{\ensuremath{\mathbb{N}} }
\newcommand{\Z}{\ensuremath{\mathbb{Z}} }
\newcommand{\Q}{\ensuremath{\mathbb{Q}} }
\newcommand{\rtor}{\ensuremath{\R \to \R} }
\newcommand{\pour}{\mbox{ pour }}
\newcommand{\coss}[1]{\ensuremath{\cos\(#1\)}}						%cosinus avec des parenthèses de bonne taille (genre frac)
\newcommand{\sinn}[1]{\ensuremath{\sin\(#1\)}}					%sinus avec des parentèses de bonne taille (genre frac)
\newcommand{\txtfrac}[2]{\ensuremath{\frac{\text{#1}}{\text{#2}}}}		%Fractions composées de texte
\newcommand{\evalfrac}[3]{\ensuremath{\left.\frac{#1}{#2}\right|_{#3}}}
\renewcommand{\(}{\left(}												%Parenthèse gauche de taille adaptive
\renewcommand{\)}{\right)}
\newcommand{\longeq}{=\joinrel=}												%Parenthèse droite de taille adaptive


%-------------------------------------------------------
%------------------MISC NEW COMMANDS--------------------
%-------------------------------------------------------
\newcommand{\degre}{\ensuremath{^\circ}}
%\newdateformat{\eudate}{\THEYEAR-\twodigit{\THEMONTH}-\twodigit{\THEDAY}}



%-------------------------------------------------------
%------------------TEXT NEW COMMANDS--------------------
%-------------------------------------------------------
\newcommand{\ts}{\textsuperscript}
\newcommand{\evid}[1]{\textbf{\uline{#1}}}        %mise en évidence (gras + souligné)



%\newcommand{\Exemple}{\underline{Exemple}}
\newcommand{\Theoreme}{\underline{Théorème}}
\newcommand{\Remarque}{\underline{Remarque}}
\newcommand{\Definition}{\underline{Définition} }
\newcommand{\skinf}{\sum^{\infty}_{k=0}}
\newcommand{\combi}[2]{\ensuremath{\begin{pmatrix} #1 \\ #2 \end{pmatrix}}}	%combinaison parmi 1 de 2
\newcommand{\intx}[3]{\ensuremath{\int_{#1}^{#2} #3 \deriv{x}}}				%intégrale dx
\newcommand{\intt}[3]{\ensuremath{\int_{#1}^{#2} #3 \deriv{t}}}				%intégrale dy
\newcommand{\misenforme}{\begin{center} Mis en forme jusqu'ici\\ \line(1,0){400}\\ normalement juste, mais à améliorer depuis ici\end{center}}	%raccourci pour mise en forme
\newcommand*\circled[1]{\tikz[baseline=(char.base)]{
            \node[shape=circle,draw,inner sep=1pt] (char) {#1};}}			%pour entourer un chiffre
\newcommand{\horrule}[1]{\rule{\linewidth}{#1}} 				% Create horizontal rule command with 1 argument of height

\theoremstyle{definition}
\newtheorem{exemp}{Exemple}
\newtheorem{examp}{Example}


%-------------------------------------------
%---Environments----------------------------
%-------------------------------------------
\NewEnviron{boite}[1][0.9]{%
	\begin{center}
		\framecolorbox{red}{white}{%
			\begin{minipage}{#1\textwidth}
 	 			\BODY
			\end{minipage}
		}
	\end{center}
}
\NewEnviron{blackbox}[1][0.9]{%
	\begin{center}
		\framecolorbox{black}{white}{%
			\begin{minipage}{#1\textwidth}
 	 			\BODY
			\end{minipage}
		}
	\end{center}
}
\NewEnviron{exemple}[1][0.8]{%
    \begin{center}
        \framecolorbox{white}{gray!20}{%
            \begin{minipage}{#1\textwidth}
                \begin{exemp}
                    \BODY
                \end{exemp}
            \end{minipage}
        }
    \end{center}
}
\NewEnviron{suiteExemple}[1][0.8]{%
    \begin{center}
        \framecolorbox{white}{gray!20}{%
            \begin{minipage}{#1\textwidth}
                \BODY
            \end{minipage}
        }
    \end{center}
}
\NewEnviron{colExemple}[1][0.8]{%
    \begin{center}
        \framecolorbox{white}{gray!20}{%
            \begin{minipage}{#1\columnwidth}
                \begin{exemp}
                    \BODY
                \end{exemp}
            \end{minipage}
        }
    \end{center}
}
\NewEnviron{example}[1][0.8]{%
    \begin{center}
        \framecolorbox{white}{gray!20}{%
            \begin{minipage}{#1\textwidth}
                \begin{examp}
                    \BODY
                \end{examp}
            \end{minipage}
	}
    \end{center}
}
\NewEnviron{systeq}[1][l]{
			\begin{center}
				$\left\{\begin{array}{#1}
					\BODY
				\end{array}\right.$
			\end{center}
 }





%-------------------------------------------
%---General settings-----------------------
%-------------------------------------------
\renewcommand{\headrulewidth}{1pt}										%ligne au haut de chaque page
\renewcommand{\footrulewidth}{1pt}										%ligne au pied de chaque page
\setstretch{1.6}
\author{Olivier Cloux}
\usepackage[tc]{titlepic}
\usepackage{graphicx}

\newcommand{\zn}{\ensuremath{(z_n)} }
\newcommand{\nz}{\overline{z}}
\renewcommand{\contentsname}{Table des Matières}
%%%%%%%%%%%%%%%%%%%%%%%%%%%%%%%%%%%%%%%%%%
%TODO : Supprimer quand plus de todo %%%%%
\marginparwidth = 75pt
\textwidth = 400pt
%%%%%%%%%%%%%%%%%%%%%%%%%%%%%%%%%%%%%%%%%%%%
\begin{document}
\setstretch{1}
\epflTitle{Algèbre}[Olivier Cloux][Automne 2016][Résumé Personnel en]
\newpage
\tableofcontents
\setstretch{1.2}
\section{Rappels}
L'ensemble \N contient les nombres \uline{strictement positifs} (1,2,3,...). Les autres ensembles ne diffèrent pas de la définition habituelle. Un nombre est \textbf{premier} s'il a exactement 2 diviseurs, 1 et lui-même. 1 n'est pas premier mais n'a qu'un seul diviseur (lui-même). Si $n>1$ est non-premier, alors il a au moins 3 diviseurs distincts. 

\section{Anneaux et Corps}
\subsection{Notions de base}
$R[X]$ est l'ensemble des polynômes à coefficients dans \R
\begin{boite}
    \evid{Définition :} Anneau\\
     Un ensemble $R$ muni de deux opérations internes $+$ et $\cdot$ et de deux éléments particuliers distincts 0 et 1 est un \textbf{anneau} s'il satisfait les propriétés suivantes ($\forall a,b,c \in R$):
    \begin{enumerate}[label=(\roman{*})]
        \item Associativité de l'addition : $(a+b)+c = a+(b+c)$
        \item Commutativité de l'addition : $a+b = b+a$
        \item 0 est l'élément neutre de l'addition : $a+0 = a$
        \item Il existe l'opposé de $a$ (noté $-a$ tel que $a+(-a) = 0$
        \item Associativité de la multiplication : $(a\cdot b) \cdot c = a \cdot (b \cdot c)$
        \item 1 est l'élément neutre de la multiplication : $a\cdot 1 = a$
        \item Distributivité de la multiplication par rapport à l'addition : $a\cdot (b+c) = (a\cdot b) + (a\cdot c)$
    \end{enumerate}
    Rigoureusement noté $(R, +, \cdot, 0, 1)$ mais noté $R$ si aucune ambiguïté.
\end{boite}
\begin{boite}
    \evid{Définition :} Sous anneau\\
    Pour $R$ un anneau et $S \subset R$ un sous-ensemble de $R$, on dit que $S$ est un sous-anneau de $R$ si 
    
    \begin{multicols}{2}
        \begin{enumerate}[label=(\roman{*})]
            \item $1_R \in S$
            \item $a \in S \to -a \in S$
            \item $a,b \in S \to a+b \in S$
            \item $a,b \in S \to a\cdot b \in S$
        \end{enumerate}
    \end{multicols}
    $0_R$ doit aussi être dans $S$, mais il découle de 2 et 4.
\end{boite}
Un anneau $R$ dans lequel $a\cdot b = b \cdot a$ est dit \textbf{commutatif}. Un anneau $R$ est \textbf{intègre} si $a\cdot b = 0$ implique que $a=0$ ou $b=0$ ; autrement dit, il est dit intègre ssi le produit de deux éléments non-nuls est non-nuls. S'il existe deux tels nombres ($a \neq 0 \neq 0$ et $ab = 0$), alors ces nombres sont appelés des \textbf{diviseurs de zéro}. \begin{boite}
    Un ensemble est \textbf{intègre} ssi il ne possède aucun diviseur de zéro, et donc que le produit de deux nombres non-nuls est forcément non-nul.
\end{boite}
\begin{exemple}
    $\Z$ et \R sont intègres, mais $\Z/12\Z$ ne l'est pas \big($[3]_{12}\cdot [4]_{12} = [12]_{12} = [0]_{12}$\big)
\end{exemple}
\begin{center}
    \evid{Dès maintenant, $R$ désigne un anneau commutatif}
\end{center}
S'il pour certains $a$ il existe des $b$ tels que $ab = ba = 1$, alors ces $b$ sont appelés les \textbf{inverses} de $a$. Un élément $a \in R$ est une \textbf{unité} s'il existe $b \in R$ tel que $ab = ba = 1$. Si un tel $b$ existe, il est unique. L'ensemble des unités de $R$ est noté $R^*$. Pour les matrices, il faut noter que 
\[(M_{nn}(R))^* = \{A \in M_{nn}(R) | det(A) \in R^*\}\]

\dots

Un anneau \textit{commutatif} dont tous les éléments non-nuls sont des \textit{unités} est appelé \textbf{un corps}
\begin{exemple}
    \Z est un anneau, mais \Q et \R sont des corps
\end{exemple}
Si $R$ est intègre, alors $ab = ac \to b = c$. Faux si $R$ n'est pas intègre !

\evid{Éléments associés}
\begin{boite}
    \evid{Définition} Soit $R$ un anneau commutatif. Pour $a,b \in R$, ils sont \textbf{associés} (et on note $a \sim b$) s'il existe $u \in R^*$ tel que $a = bu$ (et logiquement que $b = au^{-1}$)
\end{boite}
\begin{exemple}
    Pour $f(x) \in \R[X]$, les éléments associés sont de la forme $f \sim \lambda g,\ \lambda \in \R$, comme $x^2 + 1 \sim 2x^2 + 2$
\end{exemple}

\subsection{Divisibilité}
Pour $a,b \in R$, $a$ divise $b$ s'il existe $q \in R$ avec $aq = b$. On note $a | b$. Évidemment, si $a|b$ et $u \in R^*$, alors $au|b$

\begin{boite}
    \evid{Définition} Soit $p \in R,\ p \notin R^*$ ; $p$ est dit \textbf{irréductible} si $p = ab \to a$ ou $b$ est une unité 
\end{boite}

\subsection{Anneau euclidien}
\begin{boite}
    Soit $R$ un anneau commutatif et intègre. Il est \textbf{euclidien} s'il existe un \textit{stathme euclidien\footnote{une fonction $\nu : R\setminus {0} \to \N,\ b \to \nu(b)$}} t.q. $\forall a,b \in R,\ a \neq 0$, il existe $q,r \in R$ t.q. $b = aq + r$ avec $\nu(r) < \nu(a)$ ou $r = 0$
\end{boite}
\begin{exemple}
    \Z est euclidien : le stathme est $\Z \to \N,\ n \to |n|$. Regardons $18 \div 5 = 5\cdot 3 + 3$ et $|3| < |5|$\\
    $\R[X] \to \N,\ f \to \nu(x) = \deg(f)$
\end{exemple}
\begin{boite}
    \evid{Définition} Soit $a,b \in R$, alors $d \in R$ est un pgcd de $a$ et $b$ si     
    \begin{multicols}{2}
        \begin{enumerate}
            \item $d|a$ et $d|b$
            \item si $d'|a$ et $d'|b$ alors $d'|d$
        \end{enumerate}
    \end{multicols}
\end{boite}
\uline{Remarque :} $d_1$ et $d_2$ sont des pgcd de $a$ et $b$ $\iff\ d_1 \sim d_2$
\subsubsection{Algorithme d'Euclide}


\subsection{Homomorphisme d'anneaux}
Soient $R,S$ deux anneaux commutatifs, $f : R \to S$ un homomorphisme d'anneaux. Il faut que 
\begin{enumerate}
    \item $f(a+b) = f(a) + f(b)$
    \item $f(ab) = f(a)\cdot f(b)$
    \item $f(1_R) = f(1_S)$
\end{enumerate}

























\end{document}