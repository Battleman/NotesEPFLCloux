\documentclass[11pt,a4paper]{article}
%-------------------------------------------
%---Packages--------------------------------
%-------------------------------------------
\usepackage[utf8]{inputenc}
%\usepackage[T1]{fontenc}
%\usepackage{txfonts}
\usepackage{amsmath}
\usepackage{amsthm}
\usepackage{amsfonts}
\usepackage{array}
\usepackage{amssymb}
\usepackage{blindtext}
\usepackage{caption}
\usepackage{color}
\usepackage{csquotes}	    %
\usepackage{enumitem}	    %pour mieux bosser avec les listes. ajoute option label
\usepackage[yyyymmdd]{datetime}        %pour définir date custom
\usepackage{etaremune}    
\usepackage{environ}
\usepackage{fancybox}
\usepackage{fancyhdr} 	    % Custom headers and footers
\usepackage{fancyref}
%\usepackage{float}
\usepackage{floatrow}       %float and floatrow can't be together...
\usepackage{gensymb}
\usepackage{graphicx}
\usepackage[colorlinks=true, linkcolor=purple, citecolor=cyan]{hyperref}
\usepackage{footnotebackref}
\usepackage{lipsum}
\usepackage{mathtools}
\usepackage{multicol}	    %gérer plusieurs colonnes
\usepackage{setspace}
\usepackage{subcaption}
\usepackage{todonotes}	    %Bonne gestion des TODOs
%TODO commenté pour tester l'utilité... à voir% \usepackage[tc]{titlepic}      %Permet de mettre une image en page de garde
\usepackage{tikz}	    % Pour outil de dessin puissant 
\usepackage{ulem}	    %underline sur plusieurs lignes (avec \uline{})
\usepackage{vmargin} 	    %gestion des marges, avec dans l'ordre : gauche, haut, droit, bas, en-tête, entre en-tête et texte, bas de page, hauteur entre bas de page et texte 
\usepackage{wrapfig}
\usepackage{xcolor}
\usepackage{xparse}                    %Pour utiliser NewDocumentCommand et des arguments 'mmooo'
%\usepackage{fullpage} 	    %supprime toutes les marges allouées aux notes, aussi en haut et en bas

%\ExplSyntaxOn
\pagestyle{fancyplain}	    %Makes all pages in the document conform to the custom headers and footers

%-------------------------------------------
%---Document Commands-----------------------
%---------------------------{----------------
\NewDocumentCommand{\framecolorbox}{oommm}
 {% #1 = width (optional)
  % #2 = inner alignment (optional)
  % #3 = frame color
  % #4 = background color
  % #5 = text
  \IfValueTF{#1}%
   {\IfValueTF{#2}%
    {\fcolorbox{#3}{#4}{\makebox[#1][#2]{#5}}}%
    {\fcolorbox{#3}{#4}{\makebox[#1]{#5}}}%
   }%
   {\fcolorbox{#3}{#4}{#5}}%
 }%
%------------------------------------------------
%------------------ENGLISH----------------------
%----------------------------------------------

\NewDocumentCommand{\epflTitle}{mO{Olivier Cloux}O{\today}O{Notes de Cours en}D<>{../../Common}}%Arguments : Matière, Auteur, Date, Titre du doc
{
\begin{titlepage}
    \vspace*{\fill}
    \begin{center}
        \normalfont \normalsize 
        \textsc{Ecole Polytechnique Fédérale de Lausanne} \\ [25pt] % Your university, school and/or department name(s)
        \textsc{#4} %Titre du doc
        \\ [0.4 pt]
        \horrule{0.5pt} \\[0.4cm] % Thin top horizontal rule
        \huge #1 \\ % Matière
        \horrule{2pt} \\[0.5cm] % Thick bottom horizontal rule
        \includegraphics[width=8cm]{#5/EPFL_logo}
        ~\\[0.5 cm]
        \small\textsc{#2}\\[0.4cm]
        \small\textsc{#3}\\
        ~\\
        ~\\
        \includegraphics[scale=0.5]{#5/creativeCommons}
    \end{center}
    \vspace*{\fill}
\end{titlepage}
}


%-------------------------------------------
%-------------MATH NEW COMMANDS-------------
%-------------------------------------------
\newcommand{\somme}[2]{\ensuremath{\sum\limits_{#2}^{#1}}}
\newcommand{\produit}[2]{\ensuremath{\prod\limits_{#2}^{#1}}}
\newcommand{\limite}{\lim\limits_}
\newcommand{\llimite}[3]{\limite{\substack{#1 \\ #2}}\left(#3\right)}	%limites à deux condiitons
\newcommand{\et}{\mbox{ et }}
\newcommand{\deriv}[1]{\ensuremath{\, \mathrm d #1}}	%sigle dx, dt,dy... des dérivées/intégrales
%\newcommand{\fx}{\ensuremath{f'(\textbf{x}_0 + h}}
\newcommand{\ninf}{\ensuremath{n \to \infty}}	       %pour les limites : n tend vers l'infini
\newcommand{\xinf}{\ensuremath{x \to \infty}}	       %pour les limites : x tend vers l'infini
\newcommand{\infint}{\ensuremath{\int_{-\infty}^{\infty}}}
\newcommand{\xo}{\ensuremath{x \to 0}}									%x to 0
\newcommand{\no}{\ensuremath{n \to 0}}									%n zéro
\newcommand{\xx}{\ensuremath{x \to x}}									%x to x
\newcommand{\Xo}{\ensuremath{x_0}}										%x zéro
\newcommand{\X}{\ensuremath{\mathbf{X}} }
\newcommand{\A}{\ensuremath{\mathbf{A}} }
\newcommand{\R}{\ensuremath{\mathbb{R}} }								%ensemble de R
\newcommand{\rn}{\ensuremath{\mathbb{R}^n} } 							%ensemble de R de taille n
\newcommand{\Rm}{\ensuremath{\mathbb{R}^m} }  							%ensemble de R de taille m
\newcommand{\C}{\ensuremath{\mathbb{C}} } 		
\newcommand{\N}{\ensuremath{\mathbb{N}} }
\newcommand{\Z}{\ensuremath{\mathbb{Z}} }
\newcommand{\Q}{\ensuremath{\mathbb{Q}} }
\newcommand{\rtor}{\ensuremath{\R \to \R} }
\newcommand{\pour}{\mbox{ pour }}
\newcommand{\coss}[1]{\ensuremath{\cos\(#1\)}}						%cosinus avec des parenthèses de bonne taille (genre frac)
\newcommand{\sinn}[1]{\ensuremath{\sin\(#1\)}}					%sinus avec des parentèses de bonne taille (genre frac)
\newcommand{\txtfrac}[2]{\ensuremath{\frac{\text{#1}}{\text{#2}}}}		%Fractions composées de texte
\newcommand{\evalfrac}[3]{\ensuremath{\left.\frac{#1}{#2}\right|_{#3}}}
\renewcommand{\(}{\left(}												%Parenthèse gauche de taille adaptive
\renewcommand{\)}{\right)}
\newcommand{\longeq}{=\joinrel=}												%Parenthèse droite de taille adaptive


%-------------------------------------------------------
%------------------MISC NEW COMMANDS--------------------
%-------------------------------------------------------
\newcommand{\degre}{\ensuremath{^\circ}}
%\newdateformat{\eudate}{\THEYEAR-\twodigit{\THEMONTH}-\twodigit{\THEDAY}}



%-------------------------------------------------------
%------------------TEXT NEW COMMANDS--------------------
%-------------------------------------------------------
\newcommand{\ts}{\textsuperscript}
\newcommand{\evid}[1]{\textbf{\uline{#1}}}        %mise en évidence (gras + souligné)



%\newcommand{\Exemple}{\underline{Exemple}}
\newcommand{\Theoreme}{\underline{Théorème}}
\newcommand{\Remarque}{\underline{Remarque}}
\newcommand{\Definition}{\underline{Définition} }
\newcommand{\skinf}{\sum^{\infty}_{k=0}}
\newcommand{\combi}[2]{\ensuremath{\begin{pmatrix} #1 \\ #2 \end{pmatrix}}}	%combinaison parmi 1 de 2
\newcommand{\intx}[3]{\ensuremath{\int_{#1}^{#2} #3 \deriv{x}}}				%intégrale dx
\newcommand{\intt}[3]{\ensuremath{\int_{#1}^{#2} #3 \deriv{t}}}				%intégrale dy
\newcommand{\misenforme}{\begin{center} Mis en forme jusqu'ici\\ \line(1,0){400}\\ normalement juste, mais à améliorer depuis ici\end{center}}	%raccourci pour mise en forme
\newcommand*\circled[1]{\tikz[baseline=(char.base)]{
            \node[shape=circle,draw,inner sep=1pt] (char) {#1};}}			%pour entourer un chiffre
\newcommand{\horrule}[1]{\rule{\linewidth}{#1}} 				% Create horizontal rule command with 1 argument of height

\theoremstyle{definition}
\newtheorem{exemp}{Exemple}
\newtheorem{examp}{Example}


%-------------------------------------------
%---Environments----------------------------
%-------------------------------------------
\NewEnviron{boite}[1][0.9]{%
	\begin{center}
		\framecolorbox{red}{white}{%
			\begin{minipage}{#1\textwidth}
 	 			\BODY
			\end{minipage}
		}
	\end{center}
}
\NewEnviron{blackbox}[1][0.9]{%
	\begin{center}
		\framecolorbox{black}{white}{%
			\begin{minipage}{#1\textwidth}
 	 			\BODY
			\end{minipage}
		}
	\end{center}
}
\NewEnviron{exemple}[1][0.8]{%
    \begin{center}
        \framecolorbox{white}{gray!20}{%
            \begin{minipage}{#1\textwidth}
                \begin{exemp}
                    \BODY
                \end{exemp}
            \end{minipage}
        }
    \end{center}
}
\NewEnviron{suiteExemple}[1][0.8]{%
    \begin{center}
        \framecolorbox{white}{gray!20}{%
            \begin{minipage}{#1\textwidth}
                \BODY
            \end{minipage}
        }
    \end{center}
}
\NewEnviron{colExemple}[1][0.8]{%
    \begin{center}
        \framecolorbox{white}{gray!20}{%
            \begin{minipage}{#1\columnwidth}
                \begin{exemp}
                    \BODY
                \end{exemp}
            \end{minipage}
        }
    \end{center}
}
\NewEnviron{example}[1][0.8]{%
    \begin{center}
        \framecolorbox{white}{gray!20}{%
            \begin{minipage}{#1\textwidth}
                \begin{examp}
                    \BODY
                \end{examp}
            \end{minipage}
	}
    \end{center}
}
\NewEnviron{systeq}[1][l]{
			\begin{center}
				$\left\{\begin{array}{#1}
					\BODY
				\end{array}\right.$
			\end{center}
 }





%-------------------------------------------
%---General settings-----------------------
%-------------------------------------------
\renewcommand{\headrulewidth}{1pt}										%ligne au haut de chaque page
\renewcommand{\footrulewidth}{1pt}										%ligne au pied de chaque page
\setstretch{1.6}
\author{Olivier Cloux}
%%%%%%%%%%%%%%%%%%%%%%%%%%%%%%%%%%%%%%%%%%%%
%%TODO : Supprimer quand plus de todo %%%%%%
\marginparwidth = 75pt
\textwidth = 400pt
%%%%%%%%%%%%%%%%%%%%%%%%%%%%%%%%%%%%%%%%%%%%
\usepackage{titlesec}
%\numberwithin{equation}{section}
\newtheorem{defin}{Definition}
\NewEnviron{definition}[1][0.9]{%
    \begin{center}
        \framecolorbox{black}{white}{%
            \begin{minipage}{#1\textwidth}
                \begin{defin}
                    \BODY
                \end{defin}
            \end{minipage}
	}
    \end{center}
}
%\titleformat{\section}{\Large\bfseries}{}{0pt}{Lecture \thesection:\ }
\begin{document}
\epflTitle{Principle of Digital Communications}[Olivier Cloux][Spring 2017][Lecture notes in]
\tableofcontents

\section{Introduction}
Communications are basically going through a medium with some random noise, that distort the signal. From an emitted signal $x(t)$ and noise $N(t)$, we obtain the result $r(t)$ by some composition of both. To avoid ending up with a fucked up signal, we create ``boxes'' : before sending an encoder and an waveform (bits -> waves) ; after sending, on the other hand, we design a decoder.

The encoder should take $k$ bits, and output $n\geq k$ bits to add some redundancy. Hence, even if some bits are corrupted, we can reconstruct the signal and the decoder can find the original signal perfectly. The waveform box will simple take these $n$ bits to output 2D signal.
\subsection{Probability review}
Let $\omega \in \Omega$ be the outcome of a ``random experiment''.
\begin{example}
    Let $\Omega = [0,1]$, and let $\omega$ be uniformly distributed in $\Omega$. Then for 
    \[X(\omega) = \left\{\begin{array}{ll}
        0 & 0\leq \omega < \frac{1}{2}\\
        1 & \frac{1}{2} < \omega \leq 1
\end{array}\right.\]
Then $P(X=0) = P(X=1) = \frac{1}{2}$
\end{example}
For the same support, we define $Y(\omega)$ as 0 when $\omega$ is in $\{0, \frac{1}{4}\} \cup \{\frac{1}{2},\frac{3}{4}\}$, 1 elsewhere. And $Z(\omega)$ as 1 between $\frac{1}{4}$ and $\frac{3}{4}$, 0 elsewhere. The of course, \[P(Y=1) = P(Y=0) = P(X=Y) = \frac{1}{2}\] while for every $x,y,z \in \{0,1\}$
\[P((X,Z) = (x,z)) = P((X,Y) = (x,y)) = P((Y,Z) = (y,z)) = \frac{1}{4}\]
We recall the definition of \textit{independence} : 
\begin{definition}
    $X,Y$ are \textbf{independent} if 
    \[P((X,Y) = (x,y)) = P(X=x) \cdot P(Y=y)\]
\end{definition}
And for 3 variables : 
\begin{definition}
    $X,Y,Z$ are independent if
    \[P((X,Y,Z) = (x,y,z)) = P(X=x)P(Y=y)P(Z=z)\]
\end{definition}
Let's review also the definition of \textit{conditional probabilities}. The probability for a variable, having knowledge of the other. It is written as
\[P_{Y|X}(y|x) \equiv P(Y=y | X=x)\]
From that, we know have more definitions for independence : 
\begin{definition}
    $X,Y$ are independent if and only if
    \[P(Y=y | X=x) = P(Y=y)\]
    If they are independent, having knowledge on one has no effect on the second.
\end{definition}
The following rule (Bayes')is also very useful : 
\begin{definition}
    Bayes' rule :
    \[P(A|B) = \frac{P(A,B)}{P(B)}\]
\end{definition}
Following, the Condition Independence. $X,Y$ are independent, condition on $Z$ (written $X-Z-Y$) means, for all $x,y,z$ :
\[\begin{array}{ll}
    P(X=x,Y=y|Z=z) &= P(X=x|Z=z)P(Y=y|Z=z)\\& = P(X=x,Y=y)P(Z=z)\\& = P(X=x,Z=z)P(Y=y,Z=z)
\end{array}\]
\textbf{Expected value} : 
\[E[f(X)] = \somme{}{x} f(x)P(X=x)\]
If $(X,Y)$ are independent, then $E[XY] = E[X]E[Y]$

The Law of large numbers : Suppose $X_1,X_2,X_3,...$ are independent, $\R$-valued random variables such as $P(X_n = x) = p(x)$. Then $A_N = \frac{1}{N}\somme{N}{i=1} \underset{N\to\infty}{\longrightarrow} E[X]$

\section{Lecture 2}
A sender has a message in mind, and wants to send it. The message ($i$) is not completely arbitrary, and belongs in a set $\mathcal{H}$ (\textit{the set of possible messages}). We pass the message through a \textit{transmitter} that changes the message to a signal $c_i$. This $c_i$ belongs in some vector space ; for now, we say this vector space is $\rn$. This vector is then sent through a channel to obtain some $y \in \rn$. But the channel has such properties that $y$ can't be predicted perfectly. We need then to analyse the probability that, given the original signal $c$ as input to the channel, we obtain this particular $y$. Then, $y$ is going through a \textit{receiver} that will give an estimate of $i$.

\subsection{Hypothesis testing}
Let's start with a simple set $\mathcal{H} = \{0,1\}$. We know that $P(H=0) = 0.7$, while $P(H=1) = 0.3$. Our guess for $H$ is written $\hat{H}$. 

Then, our goal is to maximize $P(\hat{H} = H)$. The most logical solution, \uline{if there is no observation at all}, is to guess $\hat{H} = \underset{i \in \mathcal{H}}{\text{argmax }}P(H=i)$. But now suppose we have 
\[P_{Y|H}(y|i)\]
That is the probability that the observation is $y$ when the hypothesis is $i$. Once gone through the channel, we have $y \in \{a,b\}$, with repartition ($P(Y=a | H=0) = 0.9$)
\[\begin{bmatrix}
    0.9 & 0.1\\
    0.6 & 0.5
\end{bmatrix}\] Suppose we observe $y=a$. We look for
\begin{align*}
    P(H=0 | Y=a) = \frac{P(H=0,Y=a)}{P(Y=a)} = \frac{P(H=0)P(Y=a|H=0)}{P(Y=a)} = \frac{0.7 \cdot 0.9}{P(Y=a)}\\
    P(H=1|Y=a) = \frac{P(H=1)P(Y=a|H=1)}{P(Y=a)} = \frac{0.3\cdot0.6}{P(Y=a)}
\end{align*}
Then logically, $\hat{H} = 0$ as the first probability is much bigger.

Now suppose we observe that $Y=b$ ; after applying the same computations as above, we find $P(X=0 | Y=b) = \frac{0.7\cdot0.1}{...}$ and $P(H=1 | Y=b) = \frac{0.3\cdot0.4}{...}$, and thus $\hat{b}=1 = 1$.

\begin{definition}
    This approach is known as the \textbf{MAP rule} (max aposteriori proh)
\end{definition}
\section{Lecture 3}
Reminder about the binary hypothesis testing. We have an hypothesis $H \in \{0,1\}$, that influences an observation $y$. Based on this observation, we make a decision $\hat{H} \in \{0,1\}$. We can't decide ``not to decide'', we must chose something and thus will be either wrong or right. Thus, the system $(H - Y - \hat{H})$ is Markov, and is 
\[P(H=h, \hat{H} = \hat{h} | Y=y) = P(H=h|Y=y) P(\hat{H} = \hat{h} | Y=y)\]
As the decision is completely decided on $y$, we define 
\[P(\hat{H} = 1 | Y=y) = \phi(y)\]
Note that sometimes, if we are undecided, we might ``toss a coin'' to decide. This model allows randomized decision strategies. 

Likelihood ratio test :  those are rules of the type that for some ``threshold'' $t$, $\phi(y)$ is of the form 
\[\phi(y) =  \left\{\begin{array}{lll}
    1 & \Lambda(y) = \frac{P(Y=y|H=1)}{P(Y=y|H=o)} & > t\\
    0 & & < t
\end{array}\right.\]

Given a rule, we can compute $P(\hat{H} \neq H)$, that is the probability of error. Sometimes, it is more interesting to compute a second number $P(\hat{H} = 1 | H=0)$ (and the other way around $P(\hat{H} = 0 | H=1)$) ; those are the two error we can make. They are defined 
\begin{align}
    P(\hat{H} = 1 | H=0) = \sum_y P(Y=y | H=0) \phi(y)\\
    P(\hat{H} = 0 | H=1) = \sum_y P(Y=y | H=1) (1-\phi(y))\\
\end{align}
Consider $tP(\hat{H} = 1 | H=0) + P(\hat{H} = 0 | H=1) = \sum_y \phi(y)\underbrace{tP(Y0y|H=0)} + (1-\phi(y))\underbrace{P(Y0y|H=1)}$ And this is $\geq \sum_y \min\{t\Pr(Y=y|H=0, P(Y=y|H=1)\}$

\begin{boite}
    \evid{Lemma :} The likelihood ratio test with threshold $t$ minimizes (among all possible tests $\phi$) $tP(\hat{H} = 1 | H=0) + P(\hat{H}=0|H=1)$
\end{boite}
For the likelihood ratio test, we have 
\[\phi(y) = \left\{\begin{array}{ll}
    1 & tP(Y=y| H=0)  < P(Y=y|H=1)\\
    0  & tP(Y=y| H=0)  > P(Y=y|H=1)
\end{array}\right.\]

\begin{boite}
    \evid{Lemma} (Neymann-Pierson) : For any $0 \leq \alpha \leq 1$, there is a likelihood ratio test $\phi$ with $\alpha = P_\phi(\hat{H} = 1 | H=0)$. 
    
    For a second claim : For any other test $\psi$, either $P_\psi(\hat{H}=1|H=0) \geq P\phi(\hat{H}=1|H=0)$ or$P_\psi(\hat{H}=0|H=1) \geq P_\phi(\hat{H}=0|H=1)$ 
\end{boite}


\end{document}