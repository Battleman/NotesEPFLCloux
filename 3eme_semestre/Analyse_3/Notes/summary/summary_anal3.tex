\documentclass[12pt,a4paper]{article}
%-------------------------------------------
%---Packages--------------------------------
%-------------------------------------------
\usepackage[utf8]{inputenc}
%\usepackage[T1]{fontenc}
%\usepackage{txfonts}
\usepackage{amsmath}
\usepackage{amsthm}
\usepackage{amsfonts}
\usepackage{array}
\usepackage{amssymb}
\usepackage{blindtext}
\usepackage{caption}
\usepackage{color}
\usepackage{csquotes}	    %
\usepackage{enumitem}	    %pour mieux bosser avec les listes. ajoute option label
\usepackage[yyyymmdd]{datetime}        %pour définir date custom
\usepackage{etaremune}    
\usepackage{environ}
\usepackage{fancybox}
\usepackage{fancyhdr} 	    % Custom headers and footers
\usepackage{fancyref}
%\usepackage{float}
\usepackage{floatrow}       %float and floatrow can't be together...
\usepackage{gensymb}
\usepackage{graphicx}
\usepackage[colorlinks=true, linkcolor=purple, citecolor=cyan]{hyperref}
\usepackage{footnotebackref}
\usepackage{lipsum}
\usepackage{mathtools}
\usepackage{multicol}	    %gérer plusieurs colonnes
\usepackage{setspace}
\usepackage{subcaption}
\usepackage{todonotes}	    %Bonne gestion des TODOs
%TODO commenté pour tester l'utilité... à voir% \usepackage[tc]{titlepic}      %Permet de mettre une image en page de garde
\usepackage{tikz}	    % Pour outil de dessin puissant 
\usepackage{ulem}	    %underline sur plusieurs lignes (avec \uline{})
\usepackage{vmargin} 	    %gestion des marges, avec dans l'ordre : gauche, haut, droit, bas, en-tête, entre en-tête et texte, bas de page, hauteur entre bas de page et texte 
\usepackage{wrapfig}
\usepackage{xcolor}
\usepackage{xparse}                    %Pour utiliser NewDocumentCommand et des arguments 'mmooo'
%\usepackage{fullpage} 	    %supprime toutes les marges allouées aux notes, aussi en haut et en bas

%\ExplSyntaxOn
\pagestyle{fancyplain}	    %Makes all pages in the document conform to the custom headers and footers

%-------------------------------------------
%---Document Commands-----------------------
%---------------------------{----------------
\NewDocumentCommand{\framecolorbox}{oommm}
 {% #1 = width (optional)
  % #2 = inner alignment (optional)
  % #3 = frame color
  % #4 = background color
  % #5 = text
  \IfValueTF{#1}%
   {\IfValueTF{#2}%
    {\fcolorbox{#3}{#4}{\makebox[#1][#2]{#5}}}%
    {\fcolorbox{#3}{#4}{\makebox[#1]{#5}}}%
   }%
   {\fcolorbox{#3}{#4}{#5}}%
 }%
%------------------------------------------------
%------------------ENGLISH----------------------
%----------------------------------------------

\NewDocumentCommand{\epflTitle}{mO{Olivier Cloux}O{\today}O{Notes de Cours en}D<>{../../Common}}%Arguments : Matière, Auteur, Date, Titre du doc
{
\begin{titlepage}
    \vspace*{\fill}
    \begin{center}
        \normalfont \normalsize 
        \textsc{Ecole Polytechnique Fédérale de Lausanne} \\ [25pt] % Your university, school and/or department name(s)
        \textsc{#4} %Titre du doc
        \\ [0.4 pt]
        \horrule{0.5pt} \\[0.4cm] % Thin top horizontal rule
        \huge #1 \\ % Matière
        \horrule{2pt} \\[0.5cm] % Thick bottom horizontal rule
        \includegraphics[width=8cm]{#5/EPFL_logo}
        ~\\[0.5 cm]
        \small\textsc{#2}\\[0.4cm]
        \small\textsc{#3}\\
        ~\\
        ~\\
        \includegraphics[scale=0.5]{#5/creativeCommons}
    \end{center}
    \vspace*{\fill}
\end{titlepage}
}


%-------------------------------------------
%-------------MATH NEW COMMANDS-------------
%-------------------------------------------
\newcommand{\somme}[2]{\ensuremath{\sum\limits_{#2}^{#1}}}
\newcommand{\produit}[2]{\ensuremath{\prod\limits_{#2}^{#1}}}
\newcommand{\limite}{\lim\limits_}
\newcommand{\llimite}[3]{\limite{\substack{#1 \\ #2}}\left(#3\right)}	%limites à deux condiitons
\newcommand{\et}{\mbox{ et }}
\newcommand{\deriv}[1]{\ensuremath{\, \mathrm d #1}}	%sigle dx, dt,dy... des dérivées/intégrales
%\newcommand{\fx}{\ensuremath{f'(\textbf{x}_0 + h}}
\newcommand{\ninf}{\ensuremath{n \to \infty}}	       %pour les limites : n tend vers l'infini
\newcommand{\xinf}{\ensuremath{x \to \infty}}	       %pour les limites : x tend vers l'infini
\newcommand{\infint}{\ensuremath{\int_{-\infty}^{\infty}}}
\newcommand{\xo}{\ensuremath{x \to 0}}									%x to 0
\newcommand{\no}{\ensuremath{n \to 0}}									%n zéro
\newcommand{\xx}{\ensuremath{x \to x}}									%x to x
\newcommand{\Xo}{\ensuremath{x_0}}										%x zéro
\newcommand{\X}{\ensuremath{\mathbf{X}} }
\newcommand{\A}{\ensuremath{\mathbf{A}} }
\newcommand{\R}{\ensuremath{\mathbb{R}} }								%ensemble de R
\newcommand{\rn}{\ensuremath{\mathbb{R}^n} } 							%ensemble de R de taille n
\newcommand{\Rm}{\ensuremath{\mathbb{R}^m} }  							%ensemble de R de taille m
\newcommand{\C}{\ensuremath{\mathbb{C}} } 		
\newcommand{\N}{\ensuremath{\mathbb{N}} }
\newcommand{\Z}{\ensuremath{\mathbb{Z}} }
\newcommand{\Q}{\ensuremath{\mathbb{Q}} }
\newcommand{\rtor}{\ensuremath{\R \to \R} }
\newcommand{\pour}{\mbox{ pour }}
\newcommand{\coss}[1]{\ensuremath{\cos\(#1\)}}						%cosinus avec des parenthèses de bonne taille (genre frac)
\newcommand{\sinn}[1]{\ensuremath{\sin\(#1\)}}					%sinus avec des parentèses de bonne taille (genre frac)
\newcommand{\txtfrac}[2]{\ensuremath{\frac{\text{#1}}{\text{#2}}}}		%Fractions composées de texte
\newcommand{\evalfrac}[3]{\ensuremath{\left.\frac{#1}{#2}\right|_{#3}}}
\renewcommand{\(}{\left(}												%Parenthèse gauche de taille adaptive
\renewcommand{\)}{\right)}
\newcommand{\longeq}{=\joinrel=}												%Parenthèse droite de taille adaptive


%-------------------------------------------------------
%------------------MISC NEW COMMANDS--------------------
%-------------------------------------------------------
\newcommand{\degre}{\ensuremath{^\circ}}
%\newdateformat{\eudate}{\THEYEAR-\twodigit{\THEMONTH}-\twodigit{\THEDAY}}



%-------------------------------------------------------
%------------------TEXT NEW COMMANDS--------------------
%-------------------------------------------------------
\newcommand{\ts}{\textsuperscript}
\newcommand{\evid}[1]{\textbf{\uline{#1}}}        %mise en évidence (gras + souligné)



%\newcommand{\Exemple}{\underline{Exemple}}
\newcommand{\Theoreme}{\underline{Théorème}}
\newcommand{\Remarque}{\underline{Remarque}}
\newcommand{\Definition}{\underline{Définition} }
\newcommand{\skinf}{\sum^{\infty}_{k=0}}
\newcommand{\combi}[2]{\ensuremath{\begin{pmatrix} #1 \\ #2 \end{pmatrix}}}	%combinaison parmi 1 de 2
\newcommand{\intx}[3]{\ensuremath{\int_{#1}^{#2} #3 \deriv{x}}}				%intégrale dx
\newcommand{\intt}[3]{\ensuremath{\int_{#1}^{#2} #3 \deriv{t}}}				%intégrale dy
\newcommand{\misenforme}{\begin{center} Mis en forme jusqu'ici\\ \line(1,0){400}\\ normalement juste, mais à améliorer depuis ici\end{center}}	%raccourci pour mise en forme
\newcommand*\circled[1]{\tikz[baseline=(char.base)]{
            \node[shape=circle,draw,inner sep=1pt] (char) {#1};}}			%pour entourer un chiffre
\newcommand{\horrule}[1]{\rule{\linewidth}{#1}} 				% Create horizontal rule command with 1 argument of height

\theoremstyle{definition}
\newtheorem{exemp}{Exemple}
\newtheorem{examp}{Example}


%-------------------------------------------
%---Environments----------------------------
%-------------------------------------------
\NewEnviron{boite}[1][0.9]{%
	\begin{center}
		\framecolorbox{red}{white}{%
			\begin{minipage}{#1\textwidth}
 	 			\BODY
			\end{minipage}
		}
	\end{center}
}
\NewEnviron{blackbox}[1][0.9]{%
	\begin{center}
		\framecolorbox{black}{white}{%
			\begin{minipage}{#1\textwidth}
 	 			\BODY
			\end{minipage}
		}
	\end{center}
}
\NewEnviron{exemple}[1][0.8]{%
    \begin{center}
        \framecolorbox{white}{gray!20}{%
            \begin{minipage}{#1\textwidth}
                \begin{exemp}
                    \BODY
                \end{exemp}
            \end{minipage}
        }
    \end{center}
}
\NewEnviron{suiteExemple}[1][0.8]{%
    \begin{center}
        \framecolorbox{white}{gray!20}{%
            \begin{minipage}{#1\textwidth}
                \BODY
            \end{minipage}
        }
    \end{center}
}
\NewEnviron{colExemple}[1][0.8]{%
    \begin{center}
        \framecolorbox{white}{gray!20}{%
            \begin{minipage}{#1\columnwidth}
                \begin{exemp}
                    \BODY
                \end{exemp}
            \end{minipage}
        }
    \end{center}
}
\NewEnviron{example}[1][0.8]{%
    \begin{center}
        \framecolorbox{white}{gray!20}{%
            \begin{minipage}{#1\textwidth}
                \begin{examp}
                    \BODY
                \end{examp}
            \end{minipage}
	}
    \end{center}
}
\NewEnviron{systeq}[1][l]{
			\begin{center}
				$\left\{\begin{array}{#1}
					\BODY
				\end{array}\right.$
			\end{center}
 }





%-------------------------------------------
%---General settings-----------------------
%-------------------------------------------
\renewcommand{\headrulewidth}{1pt}										%ligne au haut de chaque page
\renewcommand{\footrulewidth}{1pt}										%ligne au pied de chaque page
\setstretch{1.6}
\author{Olivier Cloux}
\usepackage[tc]{titlepic}
\usepackage{graphicx}
\usepackage{blindtext}
%\usepackage[a4paper]{geometry}
\setcounter{section}{-1}
\date{Automne 2015}
\title{	
\normalfont \normalsize 
\textsc{Ecole Polytechnique Fédérale de Lausanne} \\ [25pt] % Your university, school and/or department name(s)
\textsc{Résumé Personnel en }\\ [0pt] %Name of the course
\horrule{0.5pt} \\[0.4cm] % Thin top horizontal rule
\huge Analyse III\\ % The assignment title<
\horrule{2pt} \\[0.5cm] % Thick bottom horizontal rule
%\includegraphics[width=8cm]{images/EPFL_logo}
}
\renewcommand{\contentsname}{Table des Matières}
\begin{document}
\setstretch{1}
\maketitle
\newpage
\tableofcontents
\setstretch{1.6}
\newcommand{\coss}[1]{\ensuremath{\cos\left(#1\))}}
\newcommand{\sinn}[1]{\ensuremath{\sin\left((#1\))}}
\renewcommand{\(}{\left(}
\renewcommand{\)}{\right)}
\newcommand{\dive}{\text{ div }}
\newcommand{\rot}{\text{ rot }}
\newcommand{\grad}{\text{ grad }}

\section{Rappels :}
\subsection{Multiplication vectorielle}
\begin{itemize}
	\item Produit \evid{vectoriel} : Soit $\vec{A} = \begin{pmatrix}a_1 \\ a_2 \\ a_3 \end{pmatrix}$ et $\vec{B} = \begin{pmatrix}b_1 \\ b_2 \\ b_3 \end{pmatrix}$ similaire. Alors : \[\vec{A}\times \vec{B} = \begin{pmatrix} a_2b_3 - a_3b_2 \\ a_3b_1 - a_1b_3 \\ a_1b_2 - a_2b_1 \end{pmatrix}\]
	\item Produit \evid{scalaire} : Soit les même $\vec{A}, \vec{B}$ qu'au dessus. Alors le produit scalaire est défini par : \[\vec{A}.\vec{B} = a_1b_1 + a_2b_2 + a_3b_3\]
\end{itemize}
\subsection{Dérivées partielles :}
Soit une $f : \rn \to \R$ et $v \in \rn$. Alors nous pouvons poser que \[\frac{\delta f}{\delta x_i},\ D_x f(x) : \limite{t\to 0} \frac{f(x+t_0) - f(x)}{t} \et \frac{\delta f}{\delta x_i} = D_{e_i} f\]
\begin{boite}
Nous définissons \evid{le gradient} de $f$ comme :
\[\nabla f = \(\frac{\delta f}{\delta x_1},...,\frac{\delta f}{\delta x_n}\) \in \rn\]
\end{boite}
Quelques propriétés utiles :\begin{align*}
	\frac{\delta}{\delta x}\Big(f(x) + g(x)\Big) = \frac{\delta }{\delta x} f(x) + \frac{\delta }{\delta x} g(x)\\
	\frac{\delta }{\delta x}\Big(f(x)\cdot g(x)\Big) = \(\frac{\delta }{\delta x}f(x)\) * g(x) + f(x) \cdot \(\frac{\delta }{\delta x} g(x)\)\\
	\left.\begin{array}{r}
	\Big(f\big(g(x)\big)\Big)' = f'\big(g(x)\big) \cdot g'(x)\\
	 \nabla (f \circ \phi)(x) = \nabla f(\phi(x)) \cdot \nabla \phi(x),\ f : \Rm \to \R^k,\ \phi : \rn \to \Rm
	\end{array}\right\} \text{égaux}\\
	\text{si } 	 f \in C^2 \text{ alors } \frac{\delta^2 f}{\delta x_i \delta x_j} = \frac{\delta^2 f}{\delta x_j \delta x_i}
\end{align*}
\subsection{Intégrales} 
\evid{Fubini : }
\begin{align*}
	 \int_{a}^{b}\int_c^d f(x,y) \deriv{x}\deriv{y} = \int_a^b\(\int_c^d f(x,y) \deriv{y}\) \deriv{x} = \int_c^d \(\int_a^b f(x,y) \deriv{x}\)\deriv{y}
\end{align*}
\subsection{Changement de Variables}
\[\int f(x) \deriv{x} = \int f(\phi(y)) \Big|\det\big(\nabla \phi(y)\big)\Big| \deriv{y}\]

\section[Opérateurs diff. de la physique]{Chapitre 1 Opérateurs différentiels de la physique}
\begin{boite}
	\evid{Définition : Le Laplacien} de $f : \rn \to \R$
	\begin{equation*}
		\Delta f := \frac{\delta^2 f}{\delta x_1^2} + \ldots + \frac{\delta^2 f}{\delta x_n^2} = \somme{n}{i=1} \frac{\delta^2 f}{\delta x_i^2}
	\end{equation*}
\end{boite}
\uline{Exemple :} 
\begin{align*}
	f(x) = x_1^2 + 3x_1x_2 + x_2, \quad f : \R^2 \to \R\\
	\Delta f = \frac{\delta^2 f}{\delta x_1^2} + \frac{\delta^2 f}{\delta x_2^2} = \frac{\delta}{\delta x_1}(2x_1 + 3x_2) +  \frac{\delta }{\delta x_2}(3x_1 + 1)\\
	=2
\end{align*}

\begin{boite}
	\evid{Définition divergence :} Soit une fonction $F(x) = \big(F_1(x),...,F_n(x)\big)$, telle que $F : \rn \to \rn,\quad F_i : \rn \to \R$. Alors nous définissons la divergence :
	\begin{equation*}
		\text{\dive } F = \frac{\delta F_1}{\delta x_1} +  \ldots + \frac{\delta F_n}{\delta x_n} = \somme{n}{i=1} \frac{\delta F_i}{\delta x_i}
	\end{equation*}
	avec "div" la divergence, et "F" un champs de vecteurs
\end{boite}
\uline{Exemple :} Soit la fonction F définie par $F(x_1,x_2) = (x_1^2 + x_1\ , \ x_1x_2)$
\begin{equation*} 	
	\dive F = \frac{\delta F_1}{\delta x_1} + \frac{\delta F_2}{\delta x_2} = \frac{\delta}{\delta x_1}(x_1^2 + x_2) + \frac{\delta }{\delta x_2}(x_1x_2)= 2x_1 + x_1 = \uline{3x_1}
\end{equation*}

\begin{boite}
	\evid{Définition : La rotation}\\
	Soit $\uline{n = 2},\ F:\R^2 \to \R^2,\ F=(F_1,F_2)$. Alors :
	\begin{equation*}
		\nabla  \times F = \rot\ F = \frac{\delta F_2}{\delta x_1}-\frac{\delta F_1}{\delta x_2}
	\end{equation*}
	Qui est en fait le déterminant de la matrice : $\begin{pmatrix}
		\frac{\delta }{\delta x_1} & \frac{\delta }{\delta x_2}\\
		F_1 & F_2
		\end{pmatrix}$
	
	Et pour $\uline{n=3},\ F : \R^3 \to \R^3,\ F = (F_1,F_2,F_3)$ :
	\begin{equation*}
		\nabla \times F = \rot\ F =\(\frac{\delta F_3}{\delta x_2} - \frac{\delta F_2}{\delta x_3}\ ,\ \frac{\delta F_1}{\delta x_3} - \frac{\delta F_3}{\delta x_1}\ ,\ \frac{\delta F_2}{\delta x_1} - \frac{\delta F_1}{\delta x_2}\)
	\end{equation*}		 
	Qui, à nouveau, est le "déterminant" de la matrice $\begin{pmatrix}
	e_1 & e_2 & e_3\\
	\frac{\delta}{\delta x_1} & \frac{\delta}{\delta x_2} & \frac{\delta}{\delta x_3}\\
	F_1 & F_2 & F_3
	\end{pmatrix}$
\end{boite}
\uline{Exemple :} $F : \R^3 \to \R^3, \ F(x_1,x_2,x_3) = (x_1^2 + x_2,\ x_3,\ x_1)$
\begin{align*}
\rot F = \(\frac{\delta}{\delta x_2}x_1 - \frac{\delta}{\delta x_3}x_3\ ,\ \frac{\delta}{\delta x_3}(x_1^2 + x_2) - \frac{\delta }{\delta x_1}x_1\ ,\ \frac{\delta }{\delta x_1}x_3 - \frac{\delta }{\delta x_2}(x_1^2 + x_2)\)\\
=(0-1,0-1,0-1) =\uline{ (-1,-1,-1)}
\end{align*}
\section[Intégrales curvilignes]{Chapitre 2 : Intégrales curvilignes}
\begin{boite}
	\evid{Définition (courbe simple)} : $\Gamma \in \R^d$ est une courbe \uline{simple} si pour un $I \in \R$ (intervalle), il existe $\gamma : I \to \R^d$ continue telle que 
	\begin{itemize}
		\item 	$\gamma(I) = \Gamma$ (courbe)
		\item 	$\gamma(t_1)\neq \gamma(t_2) \quad \forall t_1,t_2 \in$ Int I 
	\end{itemize}
	$\gamma$ est appelé la \textit{paramétrisation}
\end{boite}
\evid{Exemple ok:} 
\[\Gamma = \{x\in \R^2 : |x| = 1\} \qquad \gamma : [0,2\pi] \to \R^2 \quad \gamma(\theta) = (\cos\theta,\sin\theta)\]
\evid{Exemple faux} (dessin courbe n'importe nawak.
\begin{boite}
	\evid{Définition (courbe fermée) :} $\Gamma$ : une courbe simple est dite \uline{fermée} si 
	\[\exists (a,b) \text{ t.q. } \gamma(a) =\gamma(b),\quad \big(I = [a,b]\big)\]
	Géométriquement, cela se caractérise par les deux extrêmes de I qui se rejoignent graphiquement.
\end{boite}
\evid{Exemple ok} le cercle est une courbe est fermée, car il existe un couple, $(0,2\pi)$, qui répond au critère : 
\[\gamma(0) = \gamma(2\pi) = (1,0)\]
Notons aussi que l'intuition géométrique est correcte : la forme est un cercle, ce qui est une courbe fermée.\\
\evid{Exemples faux} Une courbe de type (hélix) (dessin chute avion)
\begin{boite}
	\evid{Définition régulière} Une courbe $\Gamma$ est dire régulière s'il existe un intervalle $[a,b]$ et une paramétrisation $\gamma : [a,b] \to \R^d$ tels que 
	\begin{align*}
		\gamma \in C^1 \([a,b] \to \R^d\)\\
		\big(\gamma_1'(t),...,\gamma_d'(t)\big) \neq (0,...,0)\quad \forall t \in [a,b]
	\end{align*}
\end{boite}

\evid{Exemple ok} Le cercle vu précédemment est aussi une courbe régulière car : $\gamma'(t) = (-\sin t, \cos t) \neq (0,0)$
 
\evid{Exemple faux} La courbe définie par la paramétrisation \[\gamma : \R \to \R^2,\quad \gamma(t) = (t^3,t^2)\] 
n'est pas régulière. En effet nous voyons que $\gamma'(t) = (3t^2, 2t)$ et que $\gamma'(0) = (0,0)$, donc $\gamma$ n'est pas régulière
(dessin rebond)
\begin{boite}
	\evid{Définition} La longueur de  de $\Gamma$ sur l'intervalle [a,b] est définie par: 
	\[\int_\Gamma 1 = \int_a^b ||\gamma'(t)||\deriv{t} \et \int_\Gamma f = \int_a^b f(\gamma(t)) ||y'(t)|| \deriv{t}\]
\end{boite}
\evid{Exemple :} \[f(x,y) = \sqrt{x^2 + 4y^2},\quad \Gamma = \{(x,y) \in \R^2 : 2y = x^2, x \in [0,1]\},\quad \gamma : [0,1] \to R^2, \gamma(t) = \(t, \frac{t^2}{2}\)\]
Ainsi : \[\int_\Gamma f = \int_0^1 \Big(f\big(\gamma(t)\big)\cdot \big|\big|\gamma'(t)\big|\big| \deriv{t}\Big) = \int_0^1 \sqrt{t^2 + 4\frac{t^4}{4}}\sqrt{1+t^2} \deriv{t} = \int_0^1 t(1+t^2) = \left[\frac{t^2}{2} + \frac{t^4}{4}\right]_0^1 = \frac{3}{4}\]
\begin{boite}
	\evid{Définition} L'intégrale curviligne (pour un champs vectoriel). Un champs de vecteur est une fonction de $\R^d \to \R^d$. Soit $F : \Gamma \to \R^d$ avec $\Gamma$ une courbe et $\gamma$ sa paramétrisation sur $[a,b]$ Alors 
	\[\int_\Gamma F. \vec{\deriv{l}} := \int_a^b F(\gamma(t)).\gamma'(t) \deriv{t}\]
	avec le point (".") le produit scalaire.
\end{boite}

\evid{Exemple} Soit $\Gamma = \{(x,y) \in \R^2 : y = \cosh(x), x\in [0,1]\}$, donc $\gamma(t) = (t,\cosh(t))$, et soit $F(x,y) = (x^2,0)$. Alors nous savons que $\int_\Gamma F. \vec{\deriv{t}} =  \int_0^1F(\gamma(t)).\gamma'(t)\deriv{t} = \int_0^1 (t^2, 0).(1,\sinh(t)) \deriv{t} = \int_0^1 t^2 \deriv{t} = \frac{1}{3}$
\begin{boite}
	\evid{Définition} courbe régulière \uline{par morceaux}. $\exists \gamma : [a,b] \to \R^d$ avec $t_0=a,...,t_k = b$ t.q. 
	\begin{itemize}
		\item	$\gamma \in C^1 ([t_i,t_i+1])$
		\item 	$\gamma'(t) \neq 0 \forall t\neq t_i$
	\end{itemize}~\\
	\evid{Définition} Soit $\Gamma$ une courbe régulière par morceaux. Alors \[\int_\Gamma f := \somme{k-1}{i=0} \int_{t_i}^{t_{i+1}} f(\gamma(t)) ||\gamma'(t)|| \deriv{t}\]
	et soit $F : \Gamma \to \R^d$ un champs vectoriel. Alors 
	\[\int_\Gamma F.\vec{\deriv{t}} = \somme{k-1}{i=0} \int_{t_i}^{t_{i+1}} F(\gamma(t)).||\gamma'(t)|| \deriv{t}\]
\end{boite}
\evid{Exemple} Soit un carré posé en 0,0 avec chaque côté nommé $\Gamma_1,...$ par l'axe x puis anti-horaire. Alors 
\begin{align*}
\Gamma = \Gamma_1 \cup \Gamma_2 \cup \Gamma_3 \cup \Gamma_4\\
\Gamma_1 = \{\gamma_1(t) = (t,0) :  t\in [0,1]\}\\
\Gamma_2 = \{\gamma_2(t) = (1,t-1) :  t\in [1,2]\}\\
\Gamma_3 = \{\gamma_3(t) = (3-t,1) :  t\in [2,3]\}\\
\Gamma_4 = \{\gamma_4(t) = (0,4-t) :  t\in [3,4]\}\\
\gamma : [0,4] \to \R^2 : \gamma(t) = \gamma_1(t)\quad [0,1], \gamma_2(t) \quad [1,2], ...
\end{align*}
Alors nous voyons que pour $f=1$, nous avons 
\begin{align*}
|\Gamma| = \int_\Gamma f = \int_0^1 ||\gamma'(t)|| \deriv{t} + \int_1^2 ||\gamma'(t)|| \deriv{t} + \int_2^3||\gamma'(t)|| \deriv{t} + \int_3^4 ||\gamma'(t)|| \deriv{t}\\
\mbox{Mais comme } ||\gamma'(t)|| = 1 \mbox{ nous avons}\\
= \int_0^1 1 + \int_1^2 1 + \int_2^3 1 + \int_3^4 1 = 4
\end{align*}
\evid{Exemple} Soit 
\[
\int_\Gamma F. \vec{\deriv{l}} \text{ avec :}
\left\{\begin{array}{l}
	F(x,y) = (xy, y^2-x)\\
	\Gamma = \{(t,t) : t\in [0,1]\}\\
	\gamma : [0,1] \to \R^2, \gamma(t) = (t,t), \gamma'(t) = (1,1)
\end{array}\right.\]
Alors, nous pouvons calculer \[\int_\Gamma F.\vec{\deriv{l}} = \int_0^1 (t^2, t^2-t).(1,1) \deriv{t} = \int_0^1 (2t^2-t)\deriv{t} = \frac{2}{3}-\frac{1}{2} = \frac{1}{6}\]\ \^{$\Gamma$} $= \{(t,e^t) : t \in [0,1]\},$ \^{$\gamma$} $: [0,1] \to \R^2,$ \^{$\gamma$}$(t) = (t,e^t),$ \^{$\gamma'$}$(t) = (1,e^t)$



\subsection{Propriétés}
\begin{boite}
\begin{enumerate}
	\item	$\Delta f = \dive (\grad f) = \nabla. (\nabla f)$
			
		\item 	$n= 3 : \rot(\nabla f) = 0 \in \R^3$\\
				$n = 2 : \rot(\nabla f) = 0 \in \R$
				\begin{align*}
					n= 3 : \dive\big(\rot(F)\big) = 0 
					\left\{
						\begin{array}{l}
							f : \rn \to \R\\
							F : \R^3 \to \R^3
						\end{array}
					\right.
				\end{align*}
		\item	$\dive(f\nabla g) = f\Delta g) + \nabla f\nabla g,\ f,g : \rn \to \R$
		\item	$\nabla (fg) = g\nabla f + f\nabla g$
		\item 	$\dive(fF) = f\ \dive F + \nabla f \cdot F$
		\item 	$\rot(\rot F) = -\Delta F + \grad \dive f$
		\item	$\uline{n=3} : \rot(fF) = \nabla f \times F + r\ \rot F$
	\end{enumerate}
\end{boite}





















\end{document}  