\documentclass[12pt,a4paper]{book}
%-------------------------------------------
%---Packages--------------------------------
%-------------------------------------------
\usepackage[utf8]{inputenc}
%\usepackage[T1]{fontenc}
%\usepackage{txfonts}
\usepackage{amsmath}
\usepackage{amsthm}
\usepackage{amsfonts}
\usepackage{array}
\usepackage{amssymb}
\usepackage{blindtext}
\usepackage{caption}
\usepackage{color}
\usepackage{csquotes}	    %
\usepackage{enumitem}	    %pour mieux bosser avec les listes. ajoute option label
\usepackage[yyyymmdd]{datetime}        %pour définir date custom
\usepackage{etaremune}    
\usepackage{environ}
\usepackage{fancybox}
\usepackage{fancyhdr} 	    % Custom headers and footers
\usepackage{fancyref}
%\usepackage{float}
\usepackage{floatrow}       %float and floatrow can't be together...
\usepackage{gensymb}
\usepackage{graphicx}
\usepackage[colorlinks=true, linkcolor=purple, citecolor=cyan]{hyperref}
\usepackage{footnotebackref}
\usepackage{lipsum}
\usepackage{mathtools}
\usepackage{multicol}	    %gérer plusieurs colonnes
\usepackage{setspace}
\usepackage{subcaption}
\usepackage{todonotes}	    %Bonne gestion des TODOs
%TODO commenté pour tester l'utilité... à voir% \usepackage[tc]{titlepic}      %Permet de mettre une image en page de garde
\usepackage{tikz}	    % Pour outil de dessin puissant 
\usepackage{ulem}	    %underline sur plusieurs lignes (avec \uline{})
\usepackage{vmargin} 	    %gestion des marges, avec dans l'ordre : gauche, haut, droit, bas, en-tête, entre en-tête et texte, bas de page, hauteur entre bas de page et texte 
\usepackage{wrapfig}
\usepackage{xcolor}
\usepackage{xparse}                    %Pour utiliser NewDocumentCommand et des arguments 'mmooo'
%\usepackage{fullpage} 	    %supprime toutes les marges allouées aux notes, aussi en haut et en bas

%\ExplSyntaxOn
\pagestyle{fancyplain}	    %Makes all pages in the document conform to the custom headers and footers

%-------------------------------------------
%---Document Commands-----------------------
%---------------------------{----------------
\NewDocumentCommand{\framecolorbox}{oommm}
 {% #1 = width (optional)
  % #2 = inner alignment (optional)
  % #3 = frame color
  % #4 = background color
  % #5 = text
  \IfValueTF{#1}%
   {\IfValueTF{#2}%
    {\fcolorbox{#3}{#4}{\makebox[#1][#2]{#5}}}%
    {\fcolorbox{#3}{#4}{\makebox[#1]{#5}}}%
   }%
   {\fcolorbox{#3}{#4}{#5}}%
 }%
%------------------------------------------------
%------------------ENGLISH----------------------
%----------------------------------------------

\NewDocumentCommand{\epflTitle}{mO{Olivier Cloux}O{\today}O{Notes de Cours en}D<>{../../Common}}%Arguments : Matière, Auteur, Date, Titre du doc
{
\begin{titlepage}
    \vspace*{\fill}
    \begin{center}
        \normalfont \normalsize 
        \textsc{Ecole Polytechnique Fédérale de Lausanne} \\ [25pt] % Your university, school and/or department name(s)
        \textsc{#4} %Titre du doc
        \\ [0.4 pt]
        \horrule{0.5pt} \\[0.4cm] % Thin top horizontal rule
        \huge #1 \\ % Matière
        \horrule{2pt} \\[0.5cm] % Thick bottom horizontal rule
        \includegraphics[width=8cm]{#5/EPFL_logo}
        ~\\[0.5 cm]
        \small\textsc{#2}\\[0.4cm]
        \small\textsc{#3}\\
        ~\\
        ~\\
        \includegraphics[scale=0.5]{#5/creativeCommons}
    \end{center}
    \vspace*{\fill}
\end{titlepage}
}


%-------------------------------------------
%-------------MATH NEW COMMANDS-------------
%-------------------------------------------
\newcommand{\somme}[2]{\ensuremath{\sum\limits_{#2}^{#1}}}
\newcommand{\produit}[2]{\ensuremath{\prod\limits_{#2}^{#1}}}
\newcommand{\limite}{\lim\limits_}
\newcommand{\llimite}[3]{\limite{\substack{#1 \\ #2}}\left(#3\right)}	%limites à deux condiitons
\newcommand{\et}{\mbox{ et }}
\newcommand{\deriv}[1]{\ensuremath{\, \mathrm d #1}}	%sigle dx, dt,dy... des dérivées/intégrales
%\newcommand{\fx}{\ensuremath{f'(\textbf{x}_0 + h}}
\newcommand{\ninf}{\ensuremath{n \to \infty}}	       %pour les limites : n tend vers l'infini
\newcommand{\xinf}{\ensuremath{x \to \infty}}	       %pour les limites : x tend vers l'infini
\newcommand{\infint}{\ensuremath{\int_{-\infty}^{\infty}}}
\newcommand{\xo}{\ensuremath{x \to 0}}									%x to 0
\newcommand{\no}{\ensuremath{n \to 0}}									%n zéro
\newcommand{\xx}{\ensuremath{x \to x}}									%x to x
\newcommand{\Xo}{\ensuremath{x_0}}										%x zéro
\newcommand{\X}{\ensuremath{\mathbf{X}} }
\newcommand{\A}{\ensuremath{\mathbf{A}} }
\newcommand{\R}{\ensuremath{\mathbb{R}} }								%ensemble de R
\newcommand{\rn}{\ensuremath{\mathbb{R}^n} } 							%ensemble de R de taille n
\newcommand{\Rm}{\ensuremath{\mathbb{R}^m} }  							%ensemble de R de taille m
\newcommand{\C}{\ensuremath{\mathbb{C}} } 		
\newcommand{\N}{\ensuremath{\mathbb{N}} }
\newcommand{\Z}{\ensuremath{\mathbb{Z}} }
\newcommand{\Q}{\ensuremath{\mathbb{Q}} }
\newcommand{\rtor}{\ensuremath{\R \to \R} }
\newcommand{\pour}{\mbox{ pour }}
\newcommand{\coss}[1]{\ensuremath{\cos\(#1\)}}						%cosinus avec des parenthèses de bonne taille (genre frac)
\newcommand{\sinn}[1]{\ensuremath{\sin\(#1\)}}					%sinus avec des parentèses de bonne taille (genre frac)
\newcommand{\txtfrac}[2]{\ensuremath{\frac{\text{#1}}{\text{#2}}}}		%Fractions composées de texte
\newcommand{\evalfrac}[3]{\ensuremath{\left.\frac{#1}{#2}\right|_{#3}}}
\renewcommand{\(}{\left(}												%Parenthèse gauche de taille adaptive
\renewcommand{\)}{\right)}
\newcommand{\longeq}{=\joinrel=}												%Parenthèse droite de taille adaptive


%-------------------------------------------------------
%------------------MISC NEW COMMANDS--------------------
%-------------------------------------------------------
\newcommand{\degre}{\ensuremath{^\circ}}
%\newdateformat{\eudate}{\THEYEAR-\twodigit{\THEMONTH}-\twodigit{\THEDAY}}



%-------------------------------------------------------
%------------------TEXT NEW COMMANDS--------------------
%-------------------------------------------------------
\newcommand{\ts}{\textsuperscript}
\newcommand{\evid}[1]{\textbf{\uline{#1}}}        %mise en évidence (gras + souligné)



%\newcommand{\Exemple}{\underline{Exemple}}
\newcommand{\Theoreme}{\underline{Théorème}}
\newcommand{\Remarque}{\underline{Remarque}}
\newcommand{\Definition}{\underline{Définition} }
\newcommand{\skinf}{\sum^{\infty}_{k=0}}
\newcommand{\combi}[2]{\ensuremath{\begin{pmatrix} #1 \\ #2 \end{pmatrix}}}	%combinaison parmi 1 de 2
\newcommand{\intx}[3]{\ensuremath{\int_{#1}^{#2} #3 \deriv{x}}}				%intégrale dx
\newcommand{\intt}[3]{\ensuremath{\int_{#1}^{#2} #3 \deriv{t}}}				%intégrale dy
\newcommand{\misenforme}{\begin{center} Mis en forme jusqu'ici\\ \line(1,0){400}\\ normalement juste, mais à améliorer depuis ici\end{center}}	%raccourci pour mise en forme
\newcommand*\circled[1]{\tikz[baseline=(char.base)]{
            \node[shape=circle,draw,inner sep=1pt] (char) {#1};}}			%pour entourer un chiffre
\newcommand{\horrule}[1]{\rule{\linewidth}{#1}} 				% Create horizontal rule command with 1 argument of height

\theoremstyle{definition}
\newtheorem{exemp}{Exemple}
\newtheorem{examp}{Example}


%-------------------------------------------
%---Environments----------------------------
%-------------------------------------------
\NewEnviron{boite}[1][0.9]{%
	\begin{center}
		\framecolorbox{red}{white}{%
			\begin{minipage}{#1\textwidth}
 	 			\BODY
			\end{minipage}
		}
	\end{center}
}
\NewEnviron{blackbox}[1][0.9]{%
	\begin{center}
		\framecolorbox{black}{white}{%
			\begin{minipage}{#1\textwidth}
 	 			\BODY
			\end{minipage}
		}
	\end{center}
}
\NewEnviron{exemple}[1][0.8]{%
    \begin{center}
        \framecolorbox{white}{gray!20}{%
            \begin{minipage}{#1\textwidth}
                \begin{exemp}
                    \BODY
                \end{exemp}
            \end{minipage}
        }
    \end{center}
}
\NewEnviron{suiteExemple}[1][0.8]{%
    \begin{center}
        \framecolorbox{white}{gray!20}{%
            \begin{minipage}{#1\textwidth}
                \BODY
            \end{minipage}
        }
    \end{center}
}
\NewEnviron{colExemple}[1][0.8]{%
    \begin{center}
        \framecolorbox{white}{gray!20}{%
            \begin{minipage}{#1\columnwidth}
                \begin{exemp}
                    \BODY
                \end{exemp}
            \end{minipage}
        }
    \end{center}
}
\NewEnviron{example}[1][0.8]{%
    \begin{center}
        \framecolorbox{white}{gray!20}{%
            \begin{minipage}{#1\textwidth}
                \begin{examp}
                    \BODY
                \end{examp}
            \end{minipage}
	}
    \end{center}
}
\NewEnviron{systeq}[1][l]{
			\begin{center}
				$\left\{\begin{array}{#1}
					\BODY
				\end{array}\right.$
			\end{center}
 }





%-------------------------------------------
%---General settings-----------------------
%-------------------------------------------
\renewcommand{\headrulewidth}{1pt}										%ligne au haut de chaque page
\renewcommand{\footrulewidth}{1pt}										%ligne au pied de chaque page
\setstretch{1.6}
\author{Olivier Cloux}
\renewcommand{\contentsname}{Table des matières}
\begin{document}
\setstretch{1}
\epflTitle{SHS : Science et Littérature B}[Olivier Cloux][Printemps 2016]
\tableofcontents
\setstretch{1.2}

\chapter{Utopie}
Avant de définir la dystopie, nous devons premièrement définir son opposé, à savoir \textit{l'utopie}. Une utopie est un idéal parfait, impossible à atteindre. Il s'agit avant tout d'un modèle. Le mot utopie est à la fois apparu soudainement dans la littérature (titre d'un récit) mais fut rapidement dépassé et adopté comme concept. Le texte d'origine est un texte de 1516 par Thomas More. \textit{L'utopie ou la meilleure forme de gouvernement possible}. C'est un récit court qui vient dialoguer avec un autre texte écrit par un de ses amis (Erasme), \textit{Éloge de la folie}.Ce dernier texte associe déjà l'individu avec le système politique dans lequel ce personnage évolue.

Le mot utopie est décomposé en \textit{U-topie}, signifiant littéralement \textit{non-lieu}. Le mot dystopie n'est pas l'inverse, car \textit{dys} n'est pas l'inverse de \textit{u}. Depuis l'anglais, on peut aussi jouer avec le titre : \textit{eu-topie}, à savoir \textit{meilleur-lieu}.  Ce jeu de mot induit déjà un certain type de lecture. Peut-on prendre ce lieu au sérieux ? Évidemment non, car nous mettons en place une lecture ironique\footnote{Qui semble dire quelque chose mais qui dit quelque chose d'autre.}. On semble indiquer un lieu parfait mais finissons par indiquer quelque chose d'autre. Le texte de More est lui-même truffé d'ironie, par exemple le fleuve qui y coule s'appelle \textit{l'anhydre}, donc \textit{sans-eau}.

Le texte est séparé en deux livres ; le premier met en place deux hommes, dont un d'eaux a beaucoup voyagé, qui discutent des systèmes politiques. à travers le monde. Vite est pointé le problème de la criminalité en Angleterre, et accuse le concept de propriété privée, car les pauvres risques de vouloir ce qu'ils n'ont pas mais les riches oui. Au début du second livre, il dit ensuite avoir rencontré un lieu où les hommes vivent en paix et harmonie, appelé \textit{Utopie}. Notons le premier livre qui évoque le réel alors que le second évoque l'imaginaire. La solution au problème actuel semble alors être la \textit{communauté des biens}, à savoir une répartition équitable des ressources et du travail entre les Utopiens.

Cette idée de communauté des biens est présente sur plusieurs niveaux. En effet, ces gens s'habillent tous pareillement et vivent tous dans les mêmes maisons, afin que l'un ne puisse plus envier l'autre. Ce système vient d'abord d'une première idée (communauté des biens) répartie sur toutes les couches de la société. Le texte se finit par la phrase énigmatique ``[\ldots] Ce pays je le rêve plutôt que je ne l'espère''. Nous avons alors plus face à une image qu'à un modèle. 

\section{Les technologies}
Dans les textes utopiques/dystopiques récents (début du 20ème siècle), les technologies sont souvent présentes. Notons que l'avancée technologique ne signifie pas forcément un meilleur dans les conditions de vie. Mais surtout, la technologie annonce un monde qui va vers le mieux, mais uniquement lorsque la société va déjà vers le mieux. En revanche, dans les sociétés qui doivent résister au pire, tout changement induit du mauvais, qui crée une force conservatrice.

C'est pour cette raison que l'on trouve dans le texte de More des allusions à la technologie, mais à titre "protectrice". Mais alors pourquoi ``rêver ce système mais pas le désirer''. Simplement, le modèle de la dystopie est le même que celui de l'utopie. 

\evid{Exemple :} dans une société, on traverse au vert et évite le rouge. Cela est une construction de société, alors on pourrait imaginer une société inverse. Cela rentre dans le cadre des \textit{signes sémiotique}, tous les signes qui sont définis par opposition à un autre. Un mot (ou signe) est fait de deux parties : le \uline{signifiant} (le rouge du feu, ce qu'on sait) et le signifié. À un même signifiant peuvent être associés plusieurs signifiés. 

Voyons le rapport avec l'utopie. Si nous posons un postulat, et définissons le reste du monde (de l'utopie) depuis ce postulat, alors nous retrouvons ce concept de signes sémiotiques. Dans le cas du texte de More, le signifiant est la communauté des biens. Toute différence dans les signifiés casserait l'utopie. Le texte de More est alors un système sémiotique cohérent, soudé et gigantesque. 

Avec un système sémiotique, il est aussi possible de créer un retournement. Le monde alors crée ne sera pas totalement opposé, mais différent. C'est ainsi que nous définissons une dystopie : le retournement de système de signes d'une utopie. En prenant la première page de 1984 de Georges Orwell et en prenant l'antonyme de chacun des adjectifs, le monde passe de terne à magique. 

\section{Le point de vue}
Une des raisons pour laquelle l'utopie semble si magique est parce que l'utopie est racontée de l'extérieur, montrant comme ces gens semblent heureux. La dystopie est également un monde parfait\footnote{Reste à définir le concept de perfection} mais montré de \textit{l'intérieur}. Dans tous les textes que nous analyserons, nous devrons tâcher de trouver quelle est l'utopie montrée.

[...]

Dès le moment où nous générons des mondes dystopiques, ce n'est pas pour avertir de ce qui pourrait arriver mais de ce qui est déjà là. Principe du petit chaperon rouge : si on ne connaît pas les loups, le texte ne sert à rien, alors que le connaître signifie que l'on a déjà été eu, donc le texte ne sert plus. 

Avec Divergente, on parle de choisir non pas des métiers mais des castes, à savoir parmi 4. Dans 1984, tout le monde est tout le temps triste, et la télévision nous regarde tout le temps. Il s'agit de l'accentuation de l'effet de lecture ironique par \uline{l'exagération}. Avec les textes que nous allons analyser nous allons chercher quel est le message ironique. 1984 dénonce (en 1948) un système fasciste, communiste, bref totalitaire. Mais il avertit du système actuel, et non pas d'un système futur ni du passé (Allemagne nazie). 

\section*{Présentation finale}
\begin{enumerate}
    \item     Commencer par un petit résumé de l'intrigue. 
    \item     Poser une question au texte. Assez générale pour pouvoir débattre dessus en plus que 5 minutes. Doit être en lien avec les enjeux de la dystopie. Fournir une réponse argumentée, en utilisant des exemples (citations,...) et l'interprétation de ces exemples. ``On va y répondre en passant par plusieurs étapes''
    \item     Conclusion (autocritique + réflexion sur ``l'utopie'' cachée derrière la dystopie)
    \item     
\end{enumerate}

\section{Analyse du texte ``Matin Brun''}
Questions possibles à poser au texte : qui sont ces Bruns (quel est le passage entre un simple adjectif et la nomination floue Bruns) ? A quel moment le personnage comprend-il l'aliénation ? 

\section{Analyse de Nouvelle Vie}
On voit un abrutissement des masses par les médias, qui mène à l'aliénation. Lien probable à \textit{Fahrenheit 451} et \textit{V pour Vendetta} (BD). Parler de ``L'antique Web'' met une distance avec nous. On trouve une référentialité mais distancée. Transformer un mot en acronyme insiste sur le côté tabou. Par exemple WC, pour ne pas parler des toilettes, ou IVV dans le texte. Appelé \textbf{brachylogie}.

Dans les dystopies, on utilise aussi souvent des mots-valise (\uline{AngSoc} dans \textit{1984})

\section{Semaine 5}
Les utopies/dystopies sont en fait des \uline{conjectures} : que se serait-il passé si... ? Dans ``nouvelle vie'', la conjecture est ``que ce serait-il passé si on avait pu (pouvait) privatiser le génome humain ?'' 

\end{document}