\documentclass[10p,a4paper]{scrartcl}
%-------------------------------------------
%---Packages--------------------------------
%-------------------------------------------
\usepackage[utf8]{inputenc}
%\usepackage[T1]{fontenc}
%\usepackage{txfonts}
\usepackage{amsmath}
\usepackage{amsthm}
\usepackage{amsfonts}
\usepackage{array}
\usepackage{amssymb}
\usepackage{blindtext}
\usepackage{caption}
\usepackage{color}
\usepackage{csquotes}	    %
\usepackage{enumitem}	    %pour mieux bosser avec les listes. ajoute option label
\usepackage[yyyymmdd]{datetime}        %pour définir date custom
\usepackage{etaremune}    
\usepackage{environ}
\usepackage{fancybox}
\usepackage{fancyhdr} 	    % Custom headers and footers
\usepackage{fancyref}
%\usepackage{float}
\usepackage{floatrow}       %float and floatrow can't be together...
\usepackage{gensymb}
\usepackage{graphicx}
\usepackage[colorlinks=true, linkcolor=purple, citecolor=cyan]{hyperref}
\usepackage{footnotebackref}
\usepackage{lipsum}
\usepackage{mathtools}
\usepackage{multicol}	    %gérer plusieurs colonnes
\usepackage{setspace}
\usepackage{subcaption}
\usepackage{todonotes}	    %Bonne gestion des TODOs
%TODO commenté pour tester l'utilité... à voir% \usepackage[tc]{titlepic}      %Permet de mettre une image en page de garde
\usepackage{tikz}	    % Pour outil de dessin puissant 
\usepackage{ulem}	    %underline sur plusieurs lignes (avec \uline{})
\usepackage{vmargin} 	    %gestion des marges, avec dans l'ordre : gauche, haut, droit, bas, en-tête, entre en-tête et texte, bas de page, hauteur entre bas de page et texte 
\usepackage{wrapfig}
\usepackage{xcolor}
\usepackage{xparse}                    %Pour utiliser NewDocumentCommand et des arguments 'mmooo'
%\usepackage{fullpage} 	    %supprime toutes les marges allouées aux notes, aussi en haut et en bas

%\ExplSyntaxOn
\pagestyle{fancyplain}	    %Makes all pages in the document conform to the custom headers and footers

%-------------------------------------------
%---Document Commands-----------------------
%---------------------------{----------------
\NewDocumentCommand{\framecolorbox}{oommm}
 {% #1 = width (optional)
  % #2 = inner alignment (optional)
  % #3 = frame color
  % #4 = background color
  % #5 = text
  \IfValueTF{#1}%
   {\IfValueTF{#2}%
    {\fcolorbox{#3}{#4}{\makebox[#1][#2]{#5}}}%
    {\fcolorbox{#3}{#4}{\makebox[#1]{#5}}}%
   }%
   {\fcolorbox{#3}{#4}{#5}}%
 }%
%------------------------------------------------
%------------------ENGLISH----------------------
%----------------------------------------------

\NewDocumentCommand{\epflTitle}{mO{Olivier Cloux}O{\today}O{Notes de Cours en}D<>{../../Common}}%Arguments : Matière, Auteur, Date, Titre du doc
{
\begin{titlepage}
    \vspace*{\fill}
    \begin{center}
        \normalfont \normalsize 
        \textsc{Ecole Polytechnique Fédérale de Lausanne} \\ [25pt] % Your university, school and/or department name(s)
        \textsc{#4} %Titre du doc
        \\ [0.4 pt]
        \horrule{0.5pt} \\[0.4cm] % Thin top horizontal rule
        \huge #1 \\ % Matière
        \horrule{2pt} \\[0.5cm] % Thick bottom horizontal rule
        \includegraphics[width=8cm]{#5/EPFL_logo}
        ~\\[0.5 cm]
        \small\textsc{#2}\\[0.4cm]
        \small\textsc{#3}\\
        ~\\
        ~\\
        \includegraphics[scale=0.5]{#5/creativeCommons}
    \end{center}
    \vspace*{\fill}
\end{titlepage}
}


%-------------------------------------------
%-------------MATH NEW COMMANDS-------------
%-------------------------------------------
\newcommand{\somme}[2]{\ensuremath{\sum\limits_{#2}^{#1}}}
\newcommand{\produit}[2]{\ensuremath{\prod\limits_{#2}^{#1}}}
\newcommand{\limite}{\lim\limits_}
\newcommand{\llimite}[3]{\limite{\substack{#1 \\ #2}}\left(#3\right)}	%limites à deux condiitons
\newcommand{\et}{\mbox{ et }}
\newcommand{\deriv}[1]{\ensuremath{\, \mathrm d #1}}	%sigle dx, dt,dy... des dérivées/intégrales
%\newcommand{\fx}{\ensuremath{f'(\textbf{x}_0 + h}}
\newcommand{\ninf}{\ensuremath{n \to \infty}}	       %pour les limites : n tend vers l'infini
\newcommand{\xinf}{\ensuremath{x \to \infty}}	       %pour les limites : x tend vers l'infini
\newcommand{\infint}{\ensuremath{\int_{-\infty}^{\infty}}}
\newcommand{\xo}{\ensuremath{x \to 0}}									%x to 0
\newcommand{\no}{\ensuremath{n \to 0}}									%n zéro
\newcommand{\xx}{\ensuremath{x \to x}}									%x to x
\newcommand{\Xo}{\ensuremath{x_0}}										%x zéro
\newcommand{\X}{\ensuremath{\mathbf{X}} }
\newcommand{\A}{\ensuremath{\mathbf{A}} }
\newcommand{\R}{\ensuremath{\mathbb{R}} }								%ensemble de R
\newcommand{\rn}{\ensuremath{\mathbb{R}^n} } 							%ensemble de R de taille n
\newcommand{\Rm}{\ensuremath{\mathbb{R}^m} }  							%ensemble de R de taille m
\newcommand{\C}{\ensuremath{\mathbb{C}} } 		
\newcommand{\N}{\ensuremath{\mathbb{N}} }
\newcommand{\Z}{\ensuremath{\mathbb{Z}} }
\newcommand{\Q}{\ensuremath{\mathbb{Q}} }
\newcommand{\rtor}{\ensuremath{\R \to \R} }
\newcommand{\pour}{\mbox{ pour }}
\newcommand{\coss}[1]{\ensuremath{\cos\(#1\)}}						%cosinus avec des parenthèses de bonne taille (genre frac)
\newcommand{\sinn}[1]{\ensuremath{\sin\(#1\)}}					%sinus avec des parentèses de bonne taille (genre frac)
\newcommand{\txtfrac}[2]{\ensuremath{\frac{\text{#1}}{\text{#2}}}}		%Fractions composées de texte
\newcommand{\evalfrac}[3]{\ensuremath{\left.\frac{#1}{#2}\right|_{#3}}}
\renewcommand{\(}{\left(}												%Parenthèse gauche de taille adaptive
\renewcommand{\)}{\right)}
\newcommand{\longeq}{=\joinrel=}												%Parenthèse droite de taille adaptive


%-------------------------------------------------------
%------------------MISC NEW COMMANDS--------------------
%-------------------------------------------------------
\newcommand{\degre}{\ensuremath{^\circ}}
%\newdateformat{\eudate}{\THEYEAR-\twodigit{\THEMONTH}-\twodigit{\THEDAY}}



%-------------------------------------------------------
%------------------TEXT NEW COMMANDS--------------------
%-------------------------------------------------------
\newcommand{\ts}{\textsuperscript}
\newcommand{\evid}[1]{\textbf{\uline{#1}}}        %mise en évidence (gras + souligné)



%\newcommand{\Exemple}{\underline{Exemple}}
\newcommand{\Theoreme}{\underline{Théorème}}
\newcommand{\Remarque}{\underline{Remarque}}
\newcommand{\Definition}{\underline{Définition} }
\newcommand{\skinf}{\sum^{\infty}_{k=0}}
\newcommand{\combi}[2]{\ensuremath{\begin{pmatrix} #1 \\ #2 \end{pmatrix}}}	%combinaison parmi 1 de 2
\newcommand{\intx}[3]{\ensuremath{\int_{#1}^{#2} #3 \deriv{x}}}				%intégrale dx
\newcommand{\intt}[3]{\ensuremath{\int_{#1}^{#2} #3 \deriv{t}}}				%intégrale dy
\newcommand{\misenforme}{\begin{center} Mis en forme jusqu'ici\\ \line(1,0){400}\\ normalement juste, mais à améliorer depuis ici\end{center}}	%raccourci pour mise en forme
\newcommand*\circled[1]{\tikz[baseline=(char.base)]{
            \node[shape=circle,draw,inner sep=1pt] (char) {#1};}}			%pour entourer un chiffre
\newcommand{\horrule}[1]{\rule{\linewidth}{#1}} 				% Create horizontal rule command with 1 argument of height

\theoremstyle{definition}
\newtheorem{exemp}{Exemple}
\newtheorem{examp}{Example}


%-------------------------------------------
%---Environments----------------------------
%-------------------------------------------
\NewEnviron{boite}[1][0.9]{%
	\begin{center}
		\framecolorbox{red}{white}{%
			\begin{minipage}{#1\textwidth}
 	 			\BODY
			\end{minipage}
		}
	\end{center}
}
\NewEnviron{blackbox}[1][0.9]{%
	\begin{center}
		\framecolorbox{black}{white}{%
			\begin{minipage}{#1\textwidth}
 	 			\BODY
			\end{minipage}
		}
	\end{center}
}
\NewEnviron{exemple}[1][0.8]{%
    \begin{center}
        \framecolorbox{white}{gray!20}{%
            \begin{minipage}{#1\textwidth}
                \begin{exemp}
                    \BODY
                \end{exemp}
            \end{minipage}
        }
    \end{center}
}
\NewEnviron{suiteExemple}[1][0.8]{%
    \begin{center}
        \framecolorbox{white}{gray!20}{%
            \begin{minipage}{#1\textwidth}
                \BODY
            \end{minipage}
        }
    \end{center}
}
\NewEnviron{colExemple}[1][0.8]{%
    \begin{center}
        \framecolorbox{white}{gray!20}{%
            \begin{minipage}{#1\columnwidth}
                \begin{exemp}
                    \BODY
                \end{exemp}
            \end{minipage}
        }
    \end{center}
}
\NewEnviron{example}[1][0.8]{%
    \begin{center}
        \framecolorbox{white}{gray!20}{%
            \begin{minipage}{#1\textwidth}
                \begin{examp}
                    \BODY
                \end{examp}
            \end{minipage}
	}
    \end{center}
}
\NewEnviron{systeq}[1][l]{
			\begin{center}
				$\left\{\begin{array}{#1}
					\BODY
				\end{array}\right.$
			\end{center}
 }





%-------------------------------------------
%---General settings-----------------------
%-------------------------------------------
\renewcommand{\headrulewidth}{1pt}										%ligne au haut de chaque page
\renewcommand{\footrulewidth}{1pt}										%ligne au pied de chaque page
\setstretch{1.6}
\author{Olivier Cloux}

\setstretch{1}
\fancyhead[L]{Sciences de l'information } % formatage en-tête - matière
\fancyhead[C]{Série 7} 	% formatage en-tête - Série
\fancyhead[R]{Olivier Cloux}	% formatage en-tête - Nom
\setmarginsrb{20mm}{20mm}{20mm}{15mm}{-5pt}{15mm}{0pt}{8mm}
\title{	
\normalfont \normalsize 
\textsc{Ecole Polytechnique Fédérale de Lausanne} \\ [25pt] % Your university, school and/or department name(s)
\textsc{Sciences de l'information}\\ [0pt] %Name of the course
\horrule{0.5pt} \\[0.4cm] % Thin top horizontal rule
\huge Série 7 \\ % The assignment title
\horrule{2pt} \\[0.5cm] % Thick bottom horizontal rule
}
\date{}
\begin{document}

\maketitle
\setcounter{section}{7}
\subsection{}
\begin{enumerate}
	\item 	Rappelons nous qu'une classe de congruence $[a]_m$ est inversible si et seulement s'il existe une classe de congruence $[b]_m$ telle que $[a]_m \cdot [b]_m = [1]_m$ ; le théorème 8.5 nous dit aussi que Soient $a$ et $m$ deux entiers avec $m > 2.\ [a]_m$ est inversible si et seulement si $a$ et $m$ sont premiers entre eux.
			\begin{enumerate}[label=(\alph*)]
				\item 	3 et 7 sont premiers (donc premiers entre eux), donc cette classe est inversible ; $([3]_7)^{-1} =$ \fbox{$[5]_7$} car $[3]_7 \cdot [5]_7 = [3\cdot 5]_7 = [15]_7 = [1]_7$
				\item 	3 et 9 ne sont pas premiers entre eux ($pgcd(3,9) = 3$), donc cette classe de congruence \uline{n'est pas inversible}.
				\item 	4 et 8 ne sont pas premiers entre eux ($pgcd(4,8) = 4$), donc cette classe de congruence \uline{n'est pas inversible}.
				\item 	Les \textit{i} qui ne sont pas inversibles sont 2, 4, 5, 6, 8 (car ils ne sont pas premiers avec 10 ; les pairs ont 2 comme multiple commun avec 10, et 5 a 5). Pour les autres \textit{i} :
						\begin{itemize}
							\item[i=1]	Son inverse est lui-même : $[1]_{10} \cdot [1]_{10} = [1]_{10}$
							\item[i=3] 	Son inverse est 7 : $[7]_{10} \cdot [3]_{10} = [21]_{10} = [1]_{10}$
							\item[i=7]	Son inverse est 3 : $[7]_{10} \cdot [3]_{10} = [21]_{10} = [1]_{10}$
							\item[i=9] 	Son inverse est lui-même : $[9]_{10} \cdot [9]_{10} = [81]_{10} = [1]_{10}$
						\end{itemize}
			\end{enumerate}
	\item 	Nous résolvons avec les opérations standard, et le concept d'inverse :\\
			$\left.\begin{array}{rl}
						& [3]_9x + [4]_9 = [1]_9\\
				\iff	& [3]_9x = [1]_9 - [4]_9\\
				\iff 	& [3]_9 x = [-3]_9 = [6]_9\\
				\iff	& [1]_3 x = x = [2]_3
			\end{array}\right\}$
			Donc $x = 3k + 2\ \forall k \in \Z$\\
			Mais nous le voulons dans la classe 9. Il faut donc regarder nos éléments avec le modulo 9 $((2\mod 3) \mod 9)$. Les éléments de $2\mod 3$ sont 2,5,8,11,14,17,20,...  En prenant le modulo 9 de ces chiffres, nous obtenons 2,5,8,2,5,8,... Nous voyons une boucle. Ainsi, \fbox{$x = \{2,5,8\}$}. Nous pouvons vérifier en les entrant dans notre équation de base :
			\begin{align*}
				[3]_9\cdot [2]_9 + [4]_9 = [10]_9 = [1]_9 \surd\\
				[3]_9\cdot [5]_9 + [4]_9 = [19]_9 = [1]_9 \surd\\
				[3]_9\cdot [8]_9 + [4]_9 = [28]_9 = [1]_9 \surd
			\end{align*}
			Nous avions donc raison quant à notre $x$ !
\end{enumerate}
\subsection{}
Soient nos deux équations de base :
\begin{align}
	4[x]_{17} = [1]_{17}\label{1}\\
	3[x]_5 = [2]_5 \label{2}
\end{align}
Calculer \textit{x} dans \eqref{1} revient à calculer l'inverse de $[4]_{17}$ (car cette équation correspond à la définition de l'inverse). Nous trouvons donc que $x = [13]_{17}$ (car $[4]_{17} \cdot [13]_{17} = [68]_{17} = [1]_{17}$), donc $x = 17k + 13\ \forall k \in \Z$

De même, pour trouver $x$ dans \eqref{2}, il faut d'abord trouver l'inverse de $[3]_5$, qui est $[2]_5$ (car... vous savez pourquoi, j'en suis sur). Donc\\
$\begin{array}{rl}
	3[x]_5 = [2]_5 	\iff & [2]_5[3]_5[x]_5 = [2]_5[2]_5\\
	\iff			& [x]_5 = [2\cdot2]_5\\
	\iff			& x = [4]_5
\end{array}$\\
Ainsi, x doit satisfaire deux équation : (pour tout $k \in \Z$)
\begin{align}
		[x]_{17} = [13]_{17} \iff x= 17k + 13 \label{x_classe_17}\\
		[x]_5 = [4]_5 \iff  x = 5k + 4\label{x_classe_5}
\end{align}

Afin de pouvoir travailler sur nos équations, nous les multiplions pour avoir la même classe. Nous multiplions chacune afin d'arriver à leur $ppmc$ (comme 5 et 17 sont premiers entre eux, $ppmc(5,17) = 5\cdot 17 = 85$). Ainsi, nous allons chercher un $x$ dans une classe de congruence 85 ($[x]_{85}$). En modifiant (1) et (2) (en les multipliant par respectivement 5 et 17), nous obtenons :

\begin{align}
	20[x]_{85} = [5]_{85} \iff 20x = 5 \mod 85\label{1_85}\\
	51[x]_{85} = [34]_{85} \iff 51x = 34 \mod 85\label{2_85}
\end{align}

Pour ensuite trouver le reste, nous cherchons un chiffre entre 0 et 85, qui soit en commun dans les équations \eqref{x_classe_17} et \eqref{x_classe_5}. Après une rapide investigation, nous trouvons l'unique chiffre 64 ($= 17\cdot3 + 13 = 5\cdot 12 + 4$). Les augures semblent indiquer donc que $x = [64]_{85}$.

Pour le vérifier, nous n'avons qu'à injecter notre $x$ dans \eqref{1_85} et \eqref{2_85}. Cela nous donne :
\begin{align*}
	20*64 \equiv 1280 \equiv 5 \mod 85\\
	51*64 \equiv 3264 \equiv 64 \mod 85
\end{align*}
ce qui correspond bien aux résultats attendus. Cela confirme donc les augures ; ainsi, \fbox{$x = [64]_{85}$}
\subsection{}
\begin{enumerate}
	\item	Si un nombre est divisible par 2 et par 7, cela signifie qu'il a, parmi sa liste de facteurs premiers, $2^{\alpha_1}\cdot7^{\alpha_2}$ (au moins), avec $\alpha_{1,2} \geq 1$. De ce fait, nous pouvons \enquote{mettre en évidence} un 14, ce qui fait que ce chiffre est divisible par 14.
	
			De même, si un nombre est divisible par 14, cela signifie qu'il a, dans ses facteurs premiers, $2^{\alpha_1}\cdot7^{\alpha_2}$ (au moins), avec $\alpha_{1,2} \geq 1$. De ce fait, nous pouvons sans problème diviser par 2 et par 7.
			
			Les deux énoncés sont donc vrais. Cela s'explique par le fait que $14 = 2\cdot7$. Diviser par 14 ou par 2 et 7 est totalement égal, sans aucuns différence.
			
	\item 	Pour qu'on nombre soit divisible par 6 ($2\cdot3$) et par 8 ($2^3$), il doit donc être divisible par $ppmc(6,8) = 24 = 2^3\cdot 3 = (3\cdot 2^2)\cdot 2 = (3\cdot 4)\cdot 2 = 12 \cdot2 $. Donc le nombre est forcément divisible par 12.
	
			En revanche, si notre nombre est divisible par 12, cela signifie qu'il est divisible par $3\cdot 2^2$. Pour être divisible par 6 et 8, il doit être divisible, nous l'avons vu avant, par $24 = 2^3\cdot3$. Nous voyons bien que c'est impossible dans bien des cas. Pour s'en convaincre, nous pouvons prendre un exemple : 12. Ce nombre est bien divisible par 12 et est divisible par 6 mais il n'est pas divisible par 8.
			
			Ainsi, notre premier énoncé est correcte, mais sa réciproque est fausse.
	
	\item 	La somme des chiffres de ce 561 est un multiple de 3, donc il est divisible par 3. De plus, la somme du premier et dernier chiffre donne le chiffre du milieu, il est donc divisible par 11. En divisant par 33 nous voyons que le résultat est 17. Ce chiffre n'est donc pas premier car il vaut $3\cdot 11 \cdot 17$ (un produit de nombres premiers différents de lui-même et 1).
	
	\item 	Pour commencer, notons que $a$ et $p$ sont premiers entre eux (selon le théorème 7.6.2). Comme $0 < a < p$ et que $p$ est premier, nous pouvons assurer que $a \mod p = a$ (grâce au théorème 8.6). Maintenant, prouvons par l'absurde que $a,2a,3a,\ldots,(p-1)a$ sont distincts (mod $p$) :\\
			Posons deux entiers \uline{distincts} $k,k' \in ]0,(p-1)[\subset \Z$.  Imaginons, tels des fous, qu'il existe une telle paire d'entiers, tels que le modulo de leurs multiples de a  ne sont pas distincts, autrement dit que 
			\begin{align*}
				ka \mod p = k'a \mod p\\
				\iff (ka-k'a) \mod p = 0 \mod p\\
				\iff a(k-k') \mod p = 0 \mod p
			\end{align*}
			Cette dernière égalité ne peut être vraie que dans 3 cas :
			\begin{enumerate}
				\item 	Si $a(k-k') = p$
				\item 	Si $a= 0$
				\item 	Si $k-k'$ = 0
			\end{enumerate}
			Le premier cas est impossible, car $p$ est premier, donc ne peut pas être le produit d'autres entiers (différents de 1 et lui-même). Le second cas est également impossible par donnée (on veut que $0 < a < p$). Seul le 3\ts{ème} cas est envisageable. Mais cela implique que $k = k'$, ce que nous avions posé comme faux au début. Nous voulons donc deux entiers distincts égaux. Comme cela est impossible, notre proposition (il existe deux entiers distincts tels que[...]) est fausse. \uline{Cela implique que $a,2a,3a,...,(p-1)a$ sont tous distincts (mod $p$)}
			
			Maintenant, attaquons nous à la preuve : nous savons que, l'inverse de $a \mod p$ est entre 0 et $p$ (non inclus), soit dans une plage de $p-1$ éléments. Nous savons également que l'inverse d'un nombre est unique. Nous savons aussi que le produit de $a,2a,3a,...,(p-a)a = a^{p-1}(p-1)!$. Par le pigeonhole principle, nous pouvons déduire que $a^{-1} \neq (2a)^{-1} \neq (3a)^{-1} \neq ... \neq ((p-1)a)^{-1}$ et que $(a\cdot 2a\cdot 3a\cdot\ldots\cdot(p-1)a) \mod p = (p-1)! \mod p$ (car nous avons le produit de $p-1$ éléments distincts, sur une plage de $p-1$ éléments ; quel que soit l'ordre nous auront le produit de ces éléments).\\
			De plus, nous pouvons trouver que les inverses de $a,2a,3a,...$ sont tous distincts sur $p-1$. Cela implique que l'inverse de $(p-1)!$ est lui-même ! En mettant cela en équations, nous avons
			\begin{align*}
				a\cdot 2a \cdot 3a \cdot\ldots\cdot (p-1)a \mod p = (p-1)! \mod p\\
				\iff a^{p-1}(p-1)! \mod p = (p-1)! \mod p\\
				\iff a^{p-1}\underbrace{(p-1)!(p-1)!}_{=1} \mod p = \underbrace{(p-1)!(p-1)!}_{=1} \mod p\\
				\iff \fbox{$a^{p-1} \equiv 1 \mod p$}
			\end{align*}
\end{enumerate}
\end{document}