\documentclass[10p,a4paper]{scrartcl}
%-------------------------------------------
%---Packages--------------------------------
%-------------------------------------------
\usepackage[utf8]{inputenc}
%\usepackage[T1]{fontenc}
%\usepackage{txfonts}
\usepackage{amsmath}
\usepackage{amsthm}
\usepackage{amsfonts}
\usepackage{array}
\usepackage{amssymb}
\usepackage{blindtext}
\usepackage{caption}
\usepackage{color}
\usepackage{csquotes}	    %
\usepackage{enumitem}	    %pour mieux bosser avec les listes. ajoute option label
\usepackage[yyyymmdd]{datetime}        %pour définir date custom
\usepackage{etaremune}    
\usepackage{environ}
\usepackage{fancybox}
\usepackage{fancyhdr} 	    % Custom headers and footers
\usepackage{fancyref}
%\usepackage{float}
\usepackage{floatrow}       %float and floatrow can't be together...
\usepackage{gensymb}
\usepackage{graphicx}
\usepackage[colorlinks=true, linkcolor=purple, citecolor=cyan]{hyperref}
\usepackage{footnotebackref}
\usepackage{lipsum}
\usepackage{mathtools}
\usepackage{multicol}	    %gérer plusieurs colonnes
\usepackage{setspace}
\usepackage{subcaption}
\usepackage{todonotes}	    %Bonne gestion des TODOs
%TODO commenté pour tester l'utilité... à voir% \usepackage[tc]{titlepic}      %Permet de mettre une image en page de garde
\usepackage{tikz}	    % Pour outil de dessin puissant 
\usepackage{ulem}	    %underline sur plusieurs lignes (avec \uline{})
\usepackage{vmargin} 	    %gestion des marges, avec dans l'ordre : gauche, haut, droit, bas, en-tête, entre en-tête et texte, bas de page, hauteur entre bas de page et texte 
\usepackage{wrapfig}
\usepackage{xcolor}
\usepackage{xparse}                    %Pour utiliser NewDocumentCommand et des arguments 'mmooo'
%\usepackage{fullpage} 	    %supprime toutes les marges allouées aux notes, aussi en haut et en bas

%\ExplSyntaxOn
\pagestyle{fancyplain}	    %Makes all pages in the document conform to the custom headers and footers

%-------------------------------------------
%---Document Commands-----------------------
%---------------------------{----------------
\NewDocumentCommand{\framecolorbox}{oommm}
 {% #1 = width (optional)
  % #2 = inner alignment (optional)
  % #3 = frame color
  % #4 = background color
  % #5 = text
  \IfValueTF{#1}%
   {\IfValueTF{#2}%
    {\fcolorbox{#3}{#4}{\makebox[#1][#2]{#5}}}%
    {\fcolorbox{#3}{#4}{\makebox[#1]{#5}}}%
   }%
   {\fcolorbox{#3}{#4}{#5}}%
 }%
%------------------------------------------------
%------------------ENGLISH----------------------
%----------------------------------------------

\NewDocumentCommand{\epflTitle}{mO{Olivier Cloux}O{\today}O{Notes de Cours en}D<>{../../Common}}%Arguments : Matière, Auteur, Date, Titre du doc
{
\begin{titlepage}
    \vspace*{\fill}
    \begin{center}
        \normalfont \normalsize 
        \textsc{Ecole Polytechnique Fédérale de Lausanne} \\ [25pt] % Your university, school and/or department name(s)
        \textsc{#4} %Titre du doc
        \\ [0.4 pt]
        \horrule{0.5pt} \\[0.4cm] % Thin top horizontal rule
        \huge #1 \\ % Matière
        \horrule{2pt} \\[0.5cm] % Thick bottom horizontal rule
        \includegraphics[width=8cm]{#5/EPFL_logo}
        ~\\[0.5 cm]
        \small\textsc{#2}\\[0.4cm]
        \small\textsc{#3}\\
        ~\\
        ~\\
        \includegraphics[scale=0.5]{#5/creativeCommons}
    \end{center}
    \vspace*{\fill}
\end{titlepage}
}


%-------------------------------------------
%-------------MATH NEW COMMANDS-------------
%-------------------------------------------
\newcommand{\somme}[2]{\ensuremath{\sum\limits_{#2}^{#1}}}
\newcommand{\produit}[2]{\ensuremath{\prod\limits_{#2}^{#1}}}
\newcommand{\limite}{\lim\limits_}
\newcommand{\llimite}[3]{\limite{\substack{#1 \\ #2}}\left(#3\right)}	%limites à deux condiitons
\newcommand{\et}{\mbox{ et }}
\newcommand{\deriv}[1]{\ensuremath{\, \mathrm d #1}}	%sigle dx, dt,dy... des dérivées/intégrales
%\newcommand{\fx}{\ensuremath{f'(\textbf{x}_0 + h}}
\newcommand{\ninf}{\ensuremath{n \to \infty}}	       %pour les limites : n tend vers l'infini
\newcommand{\xinf}{\ensuremath{x \to \infty}}	       %pour les limites : x tend vers l'infini
\newcommand{\infint}{\ensuremath{\int_{-\infty}^{\infty}}}
\newcommand{\xo}{\ensuremath{x \to 0}}									%x to 0
\newcommand{\no}{\ensuremath{n \to 0}}									%n zéro
\newcommand{\xx}{\ensuremath{x \to x}}									%x to x
\newcommand{\Xo}{\ensuremath{x_0}}										%x zéro
\newcommand{\X}{\ensuremath{\mathbf{X}} }
\newcommand{\A}{\ensuremath{\mathbf{A}} }
\newcommand{\R}{\ensuremath{\mathbb{R}} }								%ensemble de R
\newcommand{\rn}{\ensuremath{\mathbb{R}^n} } 							%ensemble de R de taille n
\newcommand{\Rm}{\ensuremath{\mathbb{R}^m} }  							%ensemble de R de taille m
\newcommand{\C}{\ensuremath{\mathbb{C}} } 		
\newcommand{\N}{\ensuremath{\mathbb{N}} }
\newcommand{\Z}{\ensuremath{\mathbb{Z}} }
\newcommand{\Q}{\ensuremath{\mathbb{Q}} }
\newcommand{\rtor}{\ensuremath{\R \to \R} }
\newcommand{\pour}{\mbox{ pour }}
\newcommand{\coss}[1]{\ensuremath{\cos\(#1\)}}						%cosinus avec des parenthèses de bonne taille (genre frac)
\newcommand{\sinn}[1]{\ensuremath{\sin\(#1\)}}					%sinus avec des parentèses de bonne taille (genre frac)
\newcommand{\txtfrac}[2]{\ensuremath{\frac{\text{#1}}{\text{#2}}}}		%Fractions composées de texte
\newcommand{\evalfrac}[3]{\ensuremath{\left.\frac{#1}{#2}\right|_{#3}}}
\renewcommand{\(}{\left(}												%Parenthèse gauche de taille adaptive
\renewcommand{\)}{\right)}
\newcommand{\longeq}{=\joinrel=}												%Parenthèse droite de taille adaptive


%-------------------------------------------------------
%------------------MISC NEW COMMANDS--------------------
%-------------------------------------------------------
\newcommand{\degre}{\ensuremath{^\circ}}
%\newdateformat{\eudate}{\THEYEAR-\twodigit{\THEMONTH}-\twodigit{\THEDAY}}



%-------------------------------------------------------
%------------------TEXT NEW COMMANDS--------------------
%-------------------------------------------------------
\newcommand{\ts}{\textsuperscript}
\newcommand{\evid}[1]{\textbf{\uline{#1}}}        %mise en évidence (gras + souligné)



%\newcommand{\Exemple}{\underline{Exemple}}
\newcommand{\Theoreme}{\underline{Théorème}}
\newcommand{\Remarque}{\underline{Remarque}}
\newcommand{\Definition}{\underline{Définition} }
\newcommand{\skinf}{\sum^{\infty}_{k=0}}
\newcommand{\combi}[2]{\ensuremath{\begin{pmatrix} #1 \\ #2 \end{pmatrix}}}	%combinaison parmi 1 de 2
\newcommand{\intx}[3]{\ensuremath{\int_{#1}^{#2} #3 \deriv{x}}}				%intégrale dx
\newcommand{\intt}[3]{\ensuremath{\int_{#1}^{#2} #3 \deriv{t}}}				%intégrale dy
\newcommand{\misenforme}{\begin{center} Mis en forme jusqu'ici\\ \line(1,0){400}\\ normalement juste, mais à améliorer depuis ici\end{center}}	%raccourci pour mise en forme
\newcommand*\circled[1]{\tikz[baseline=(char.base)]{
            \node[shape=circle,draw,inner sep=1pt] (char) {#1};}}			%pour entourer un chiffre
\newcommand{\horrule}[1]{\rule{\linewidth}{#1}} 				% Create horizontal rule command with 1 argument of height

\theoremstyle{definition}
\newtheorem{exemp}{Exemple}
\newtheorem{examp}{Example}


%-------------------------------------------
%---Environments----------------------------
%-------------------------------------------
\NewEnviron{boite}[1][0.9]{%
	\begin{center}
		\framecolorbox{red}{white}{%
			\begin{minipage}{#1\textwidth}
 	 			\BODY
			\end{minipage}
		}
	\end{center}
}
\NewEnviron{blackbox}[1][0.9]{%
	\begin{center}
		\framecolorbox{black}{white}{%
			\begin{minipage}{#1\textwidth}
 	 			\BODY
			\end{minipage}
		}
	\end{center}
}
\NewEnviron{exemple}[1][0.8]{%
    \begin{center}
        \framecolorbox{white}{gray!20}{%
            \begin{minipage}{#1\textwidth}
                \begin{exemp}
                    \BODY
                \end{exemp}
            \end{minipage}
        }
    \end{center}
}
\NewEnviron{suiteExemple}[1][0.8]{%
    \begin{center}
        \framecolorbox{white}{gray!20}{%
            \begin{minipage}{#1\textwidth}
                \BODY
            \end{minipage}
        }
    \end{center}
}
\NewEnviron{colExemple}[1][0.8]{%
    \begin{center}
        \framecolorbox{white}{gray!20}{%
            \begin{minipage}{#1\columnwidth}
                \begin{exemp}
                    \BODY
                \end{exemp}
            \end{minipage}
        }
    \end{center}
}
\NewEnviron{example}[1][0.8]{%
    \begin{center}
        \framecolorbox{white}{gray!20}{%
            \begin{minipage}{#1\textwidth}
                \begin{examp}
                    \BODY
                \end{examp}
            \end{minipage}
	}
    \end{center}
}
\NewEnviron{systeq}[1][l]{
			\begin{center}
				$\left\{\begin{array}{#1}
					\BODY
				\end{array}\right.$
			\end{center}
 }





%-------------------------------------------
%---General settings-----------------------
%-------------------------------------------
\renewcommand{\headrulewidth}{1pt}										%ligne au haut de chaque page
\renewcommand{\footrulewidth}{1pt}										%ligne au pied de chaque page
\setstretch{1.6}
\author{Olivier Cloux}

\setstretch{1}
\fancyhead[L]{Sciences de l'information } % formatage en-tête - matière
\fancyhead[C]{Série 9} 	% formatage en-tête - Série
\fancyhead[R]{Olivier Cloux}	% formatage en-tête - Nom
\setmarginsrb{20mm}{20mm}{20mm}{15mm}{-5pt}{15mm}{0pt}{8mm}
\title{	
\normalfont \normalsize 
\textsc{Ecole Polytechnique Fédérale de Lausanne} \\ [25pt] % Your university, school and/or department name(s)
\textsc{Sciences de l'information}\\ [0pt] %Name of the course
\horrule{0.5pt} \\[0.4cm] % Thin top horizontal rule
\huge Série 9 \\ % The assignment title
\horrule{2pt} \\[0.5cm] % Thick bottom horizontal rule
}
\date{}
\begin{document}

\maketitle
\setcounter{section}{9}
\subsection{}
\begin{enumerate}
	\item	Nous savons que $K = m = 451$. Comme le chiffre du milieu est la somme des deux autres, 11 est un diviseur$ \to 451 = 11\cdot41$, donc $p = 11,\ q = 41$. Depuis là, il est facile de déterminer $k = ppmc(10,40) = 40$. Comme nous savons que $e = 13$, calculer $f$ devient facile : 
			\begin{align*}
				[f]_k = ([e]_k)^{-1}\\
				[f]_{40} = ([13]_{40})^{-1}\\
				40 = 3\cdot 13 + 1\\
				\iff 1 = 40 - 3\cdot 13\\
				\iff [f]_{40} = [-3]_{40} = [37]_{40}
			\end{align*}
			Donc \fbox{$f = 37 \equiv -3$}
			
	\item	Comme $m = 451$, nous allons considérer $P$ et $C$ dans $\Z/451\Z$. Par définition, $C = ([P]_m)^e = ([109]_{451})^{13}$. Depuis là, nous pouvons calculer :
			\begin{align*}
				\big(109^{13}\big) \mod 451\\
				= \big((109^2 \mod 451)^6 \cdot 109 \big) \mod 451
			\end{align*}
			Or, $109^2 = 11881$ et $26\cdot 451 = 11726 \to 109^2 \mod 451 = 11881-11726 = 155$
			\begin{align*}
				= \big((155^2 \mod 451)^3 \cdot 109) \mod 451
			\end{align*}
			Or, $155^2 = 24025$ et $53\cdot 451 = 23903 \to 155^2 \mod 451 = 24025-23903=122$
			\begin{align*}
				= \big((122^3 \mod 451) \cdot 109\big) \mod 451
			\end{align*}
			Or, $122^3 = 1815848$ et $4026\cdot 451 = 1815726 \to 122^3 \mod 451 = 1815848-1815726 = 122$
			\begin{align*}
			 = \big(122 \cdot 109) \mod 451
			\end{align*}
			Or, $122\cdot109 = 13298$ et $29*451 = 13079 \to (122\cdot109) \mod 451 = 13296 - 13079 = 219$
			\begin{align*}
			 =(122\cdot 109) \mod 451 = 219
			\end{align*}
			Donc $[109^{13}]_{451} =$  \fbox{$219 = C$}\\
			L'idée ici était de décomposer en plus petits facteurs. Pour ensuite faire les modulos intermédiaire, je prenais le floor de la division par 451. Par exemple, $109^2 = 11881$ et $\lfloor \frac{11881}{451}\rfloor = \lfloor 26.3436...\rfloor = 26$. Depuis la, on prend $26 \cdot 451$, et on prend la différence pour avoir le reste de la division entière.
			
	\item	$P'$ s'obtient de la manière inverse : $P' = ([C]_{m})^{f} = ([43]_{451})^{37}$ Vous l'aurez compris avant, la démarche pour trouver le reste d'une telle division entière est long à faire (et à écrire). Je vais donc appliquer la même technique que précédemment, sans justifier les calculs intermédiaires. Ils ont cependant été fait avec la même technique.
			\begin{align*}
				43^{37} \mod 451\\
				=\Big(\big((43^4)^3\big)^3 \cdot 43 \Big) \mod 451\\
				= \big((221^3)^3 \cdot 43 \big) \mod 451\\
				= \big(78^3 \cdot 43 \big) \mod 451\\
				= (100 \cdot 43) \mod 451\\
				= 241
			\end{align*}
			A titre d'exemple : $43^4 = 3418801 \to \lfloor \frac{3418801}{451}\rfloor = 7580 \to 7580 \cdot 451 = 3418580 \to 43^4 - 7580\cdot 451 = 3418801 - 3418580 = 221 \to 43^4 \mod 451 = 221$. La suite des résultats (78,100,241) s'est obtenue avec la même méthode. Tout ça à la calculatrice :)
\end{enumerate}

\subsection{}
\begin{enumerate}
	\item	Il lui est facile de le vérifier : comme nous l'avons vu dans les propriétés de RSA et des classes de congruence, en élevant la signature (modulo K), à la puissance $e$ (publique, l'inverse de $f$), il obtiendra $P(u)$. Il n'a ensuite qu'à recalculer $P(u)$ selon la méthode vue (rien n'est caché à ce stade), pour vérifier que les deux correspondent. S'ils ne correspondent pas, la signature est fausse (faite par la mauvaise personne, ou le message a été modifié).
	\item	$(p,q) = (97,173) \to k = ppmc(96,172) = ppmc(2^5\cdot 3\ ,\ 2^2 \cdot 43) = 2^5 \cdot 3 \cdot 43 = 4128 = k$\\
			$[f]_k = ([e]_k)^{-1} \to [f]_{4128} = ([17]_{4128})^{-1}$. Pour cela, on utilise l'algorithme d'Euclide et Bézout : Soit $4128 = a$ et $17 = b$
			\begin{align*}
				4128 = 242\cdot 17 + 14\\
				a = 242b + 14 \to 14 = a-242b\\
				b = 14 + 3 \to 3 = b-14 = b-(a-242b) = -a + 243b\\
				14 = 4\cdot 3 + 2 \to 2 = 14 - 4\cdot 3 = (a-242b) -4\cdot(-a+243b) = 5a - 1214b\\
				3 = 2 + 1 \to 1 = 3-2 = (-a+243b) - (5a - 1214 b) = -6a + 1457 b
			\end{align*}
			Donc $[1]_{4128} = [-6\cdot 4128]_{4128} + [1457 \cdot 17]_{4128} \to$ \fbox{$f = 1457$}
			
	\item	$u = (C,I,A,O) \to c(u) = (12,18,10,24)\\
			\to P(u) = \somme{n}{i=1} c(u_i)\cdot 37^{i-1} = 12 \cdot 1 + 18\cdot 37 + 10\cdot 37^2 + 24 \cdot37^3 = 1'230'040$\\
			Comme $(p,q) = (97,173)$, alors $K = 97\cdot 173 = 16'781$\\			
			$\to [P(u)]_K = [1'230'040]_{16'781}$\\
			De plus, $1'230'040 = 73\cdot 16'781 + 5'027$, donc $[P(u)]_K = [5'027]_{16'781}$\\
			Finalement, $[\sigma(u)]_{16781} = ([5027]_{16781})^{f}$ avec $f = 1457$. Donc \fbox{$\sigma = 1759$}
			
	\item	Calculons la signature de $u = (P,O,I,N,C,A,R,E)$. $c(u) = (25, 24, 18, 23, 12, 10, 27, 14)$\\
	 		$\to P(u) = 25\cdot1 + 24\cdot37 + 18\cdot 37^2 + \ldots + 14 \cdot 37^7 = 	1'399'038'012'981$\\
	 		$\to [P(u)]_K = [1'399'038'012'981]_{16'781} = [1'821]_{16'781}$\\
	 		$\to [\sigma(u)]_K = ([P(u)]_K)^f = ([1'821]_{16'781})^{1'457} \to$ \fbox{$\sigma(POINCARE) =  5'313 \neq  14'812$} Donc la signature est fausse.
	 		
	\item	La signature est facile à trouver. En effet, notons $u_1$ = BART DOIT LISA CHF 100 et $u_2 =$ BART DOIT LISA CHF 10000000. Nous pouvons affirmer que $P(u_1) = P(u_2)$, car nous avons noté que c(0) = 0. Donc la fin du calcul de $P(u_2)$ sera $[...] + 0\cdot 37^{21} + 0 \cdot 37^{22} + \ldots$. Nous voyons qu'ajouter des 0 (sans rien mettre d'autre à la suite) ne change rien au calcul. Le message pourrait parler de CHF 1 ou de CHF 1000000000000, cela ne changera pas les $P(u_i)$. Comme les $P(u_i)$ sont identiques, la suite sera la même, donc les signatures seront identiques. Donc $\sigma(u_2) = \sigma(u_1) = S$. Notons cependant que si le message avait un caractère quelconque à la fin (différent de 0), cela changerait tout. Car ce \enquote{caractère de contrôle} comme on pourrait l'appeler, passerait de\footnote{chiffres indicatifs} $c(u_{21})\cdot37^{21}$ à $c(u_21) \cdot 37^{28}$. Le nombre de 0 changerait alors tout.
	
			Comme nous venons de le voir, la méthode dans son état actuel n'est pas efficace, justement à cause de ce problème ; si le message était intercepté et modifié (de la bonne manière), la signature ne serait pas modifiée, donc la modification serait invisible.
	\item	 Une solution, comme évoquée au dessus, consisterait en l'ajout d'un caractère de contrôle, avec $c(u_{controle}) \neq 0$ a ajouter a la fin de chaque message. De cette manière, le problème des 0 de fin disparaîtrait.

\end{enumerate}

\subsection{}
\begin{enumerate}
	\item 	$p = 83 = 2\cdot41 + 1,\ q =59 = 2\cdot 29 + 1 \to k = ppmc(82,58) = 41\cdot 2\cdot 29 = 2378$ Comme il n'y a pas de facteur 17 dans ce chiffre, $k$ et $e$ sont premiers entre eux. \uline{Donc ces paramètres sont valides.}\\
			\\
			Donc \fbox{$K = m = 83\cdot 59 = 4897$}\\
			\\
			$[f]_{m} =([e]_{m})^{-1} = ([17]_{4987})^{-1}$\\
			$4897 = 288\cdot 17 + 1 \iff 1 = 4897 -	288\cdot 17$\\
			$[-288]_{4897}\cdot[17]_{4897} = [1]_{4897}$\\
			$[f]_{4897} = [-288]_{4897} = [4609]_{4897} \iff$ \fbox{$f = 4609$}
			
	\item	$83 = 2 \cdot 41 + 1$, avec 41 premier, et $59 = 2\cdot 29 + 1$ avec 29 premier. Donc \uline{oui, ces chiffres sont sûrs}.
	
	\item	Comme nous savons que $K = 4897 = 59\cdot 83 = p\cdot q$, le théorème des restes chinois, nous permet de poser que $P^e = P \mod K 			\iff 
			\left\{
				\begin{array}{l}
					P^e = P \mod p\\
					P^e = P \mod q
				\end{array}
			\right.$. Nous sommes exactement dans le cas du théorème 10.4 (aussi car $e-1 = 17-1$ est une puissance de 2). Ce théorème dit qu'il y a exactement 9 solutions, qui sont les combinaisons de $[0]_p,[1]_p,[-1]_p$ avec $[0]_q,[1]_q,[-1]_q$. Si certaines combinaisons sont triviales ($[0]_p,[0]_q$ par exemple), d'autre le sont moins. Je montrerai ici la manière d'en trouver une, les autres se feront exactement de la même manière. La combinaisons que nous allons considérer est $[1]_{59},[82]_{83}$.
	
			Pour que notre résultat dans $\Z/pq\Z$ soit valable, nous devons poser 
			\begin{equation}
				x = 59 q + 1\text{ et }x = 83k + 82,\ \forall k,q \in \Z
				\label{deux_eq}
			\end{equation}, puis les égaler. Nous cherchons donc un $k$ et un $q$ tels que 
			\begin{equation}
				59q - 83k = 81
				\label{combine}
			\end{equation}
			Pour cela, nous appliquons l'égalité de Bézout sur 59 et 83. Il doit exister un couple $u,v$ tels que $59u + 83 v = pgcd(83,59) = 1$. Comme fait plus haut :
			\begin{align*}
				83 = 59 + 24 \to 24 = 83 - 59\\
				59 = 2\cdot 24 + 11 \to 11 = 59 - 2\cdot 24 = 59-2\cdot(83-59) = 3\cdot 59 + 5\cdot 83\\
				24 = 2\cdot 11 + 2 \to 2 = 24 - 2\cdot 11 = (83-59) - 2\cdot (3\cdot 59 -2\cdot 83) = 5\cdot 83 - 7\cdot 59\\
				11 = 5\cdot 2 + 1 \to 1 = 11 - 5\cdot 2 = (3\cdot 59-2\cdot 83) - 5\cdot (5\cdot 83-7\cdot 59) = 38\cdot 59 - 27\cdot 83
			\end{align*}
			Donc, nous savons que
			\begin{align*}
			1 = 38\cdot 59 - 27\cdot 83\\
			\to 81 = 3078 \cdot 59 - 2187 \cdot 83\\
			\to u = 3078,\ v = 2187
			\end{align*}
			Ces chiffres correspondent a notre équation. Cependant, afin de les simplifier, nous pouvons appliquer sur $u$, respectivement $v$, un modulo 83, respectivement 59. Cela nous donne les résultats $q = 7,\ k = 4$. Nous pouvons bien vérifier que ces chiffres sont valides en les remplaçant dans \eqref{combine}. Ensuite, nous pouvons remplacer nos valeurs dans \eqref{deux_eq}, afin de trouver $x = 414$. Une des 9 solutions est donc $[414]_{4897}$
			
			En appliquant cette méthode, nous trouvons toutes les autres solutions : 
			\begin{equation*}
			P = \{0,\ 1,\ 414,\ 2241,\ 2242,\ 2655,\ 2656,\ 4483,\ 4896\}
			\end{equation*}
			(bien entendu, dans $\Z/4897\Z$)
			
	
	\item	\begin{enumerate}
				\item 	$p = 17 = 2^{4} + 1 = 2\cdot 8 + 1$, mais 8 n'est pas premier. Donc 17 n'est pas sûr.	
				\item 	Cela ressemble très fortement au petit théorème de Fermat.... Celui-ci énonce que, si $p$ est un nombre premier, pour tout entier $a$ 
						\begin{equation*}
							([a]_p)^p = [a]_p \iff a^p = a\ (\text{mod }p)		
						\end{equation*}
						que nous ne prouverons pas, car il s'agit là de l'objet d'unes série précédente.\\
						Cela nous permet d'affirmer que $x^{17} = x\ (\text{mod }17) \ \forall x \in \Z/17\Z$. Les solutions sont donc \fbox{$x = \{0,1,2,3,4,5,6,7,8,9,10,11,12,13,14,15,16\}$}
				\item 	Selon le théorème des restes chinois, cela revient à chercher ne nombre de solutions dans 
				$\left\{
					\begin{array}{l}
					P^{17} = P \mod 17\\
					P^{17} = P \mod 59
					\end{array}
				\right.$. Comme 59 est sûr ($2 \cdot 29 +1$), la seconde équation admettra 3 solutions\footnote{comme prouvé dans le théorème 10.4, et utilisé dans l'exercice 9.3.3} : $[-1]_{59},[1]_{59},[0]_{59}$. L'équation en 17, nous l'avons vu auparavant, admet 17 solutions. Il y aura donc autant de solutions que de combinaisons, à savoir $3 \cdot 17 =$ \fbox{$51$ messages distincts}
			\end{enumerate}
\end{enumerate}

\end{document}