\documentclass[12pt,a4paper]{article}
\usepackage[utf8]{inputenc}
\usepackage{amsmath}
\usepackage{amsfonts}
\usepackage{amssymb}
\usepackage{graphicx}
\author{Olivier Cloux}
\title{Physique1}
\begin{document}
\section{Introduction}
\subsection{Structure atomique de la matière}
\begin{itemize}
\item Toute la matière est formée d'atomes, de types différents et assemblés de diverses manières.
\item Rayon d'un atome : 10$^{-10}$ mètres
\item L'atome a nu noyau très lourd (99.9\% de la masse)
\item le noyau est formé de protons et de neutrons.
\end{itemize}
\subsection{La physique}
Science dont le but est d'étudier et de comprendre les composants de la matière et leurs interactions mutuelles. En fonction de ces interactions, on tente d'expliquer les propriétés de la matière et les phénomènes naturels. Les explications que l'on donne sont données sous forme de lois aussi fondamentales que possibles. Elles s'expriment sous forme mathématique. C'est la langage de la physique.
\subsection{Exemple : la vitesse instantanée}
au temps $t_1$, position $x_1$ = x($t_1$)
\\au temps $t_2$, position $x_2$ = x($t_2$)
\\ vitesse moyenne entre $t_1$ et $t_2$ : $\frac{x_2-x_1}{t_2-t_1}$ = $\frac{\Delta x}{\Delta t}$
\\ Vitesse instantanée v =     $\lim\limits_{x \to 0}$ [TROU]

\subsection{Accélération moyenne}
\subsection{Addition des vitesses}
Les vitesses peuvent s'additionner, mai en général vectoriellement. Loi indépendante des vitesses en jeu, des objets en présence,... $\Rightarrow$ c'est une loi
\subsection{Les lois de la physique}
Les lois viennent d'observations. On utilise les maths, mais ce n'est pas construit comme les maths. Nous ne connaissons pas tout des lois physiques. Les lois viennent de l'expérimentation. La mécanique classique n'est plus valable aux grandes vitesses. La théorie de la relativité démontre que la théorie classique est fausse, mais la différence est négligeable aux petites vitesses.
\part{Première partie : Sensibilisation aux objectifs de la mécanique}
Projectile dans un champs de pesanteur (balistique) et Oscillateur harmonique. Le but est de se familiariser avec les équations différentielles et de comprendre les lois du mouvement (F=ma...).
\paragraph{Le point matériel : système assimilé à un point géométrique auquel on attribue toute ma masse d'un système physique. L'état est décrit par une position et une vitesse.}
\subsection{MRU}
ON définit un axe x et une origine O.
v(t) $\equiv$ $\frac{dx(t)}{dt}$ = $\dot x$ = $v_0$ = constante
\\Solution : x(t) = $v_0$t + $x_0$ où $x_0$ est une constante
\subsection{MRUA}
Point se déplaçant sur une ligne droite, avec une accélération constante.
\\ a(t) $\equiv \frac{d^2x(t)}{dt^2} \equiv \ddot x$ = $a_0$ = constante
\\ Dérivée seconde de la fonction inconnue x(t)
\\Solutions : x(t) = $\frac{1}{2} a_0t^2$ + $v_0$t + $x_0$ où $x_0$ est une constante.
\subsection{Chute des corps/balistique}
Principe d'inertie. Si on n'exerce aucune force sur u corps, il ne bougera pas., il restera au repos. Toute autre déviation ou accélération est due à une force. 
\subsubsection{Lois de Newton}
\begin{enumerate}
\item[Loi d'inertie :] Tant qu'on touche pas, ça bouge pas.
\item[Seconde loi :] F = ma
\item[Action-réaction :] À chaque action, une action égale et opposée.$\vec{F}_{1\rightarrow 2}$ = -$\vec{F}_{2\rightarrow 1}$
\end{enumerate}
\subsubsection{Balistique}
$\vec{x}_0 = 
\begin{pmatrix}
x_0\\
y_0\\
z_0
\end{pmatrix}$
\section{Oscillateur}
\subsection{Phénomène de résonance }
On trouve souvent de la résonance. Des fois elle veut être provoquée (tuyaux d'orgue, balançoire de jardins,..) et des fois évitée (lave-linge, pont,...)
\section{equations différentielles et chaos}
Parfois les systèmes sont chaotiques, bien que ce soit décrit par des équations déterministes. Le chaos déterministe apparaît aussi dans des systèmes comme l'atmosphère terrestre.
\part{Cinématique et dynamique du point matériel}
\section{Introduction à la cinématique}
\subsection{Repère}
Origine O, avec trois axes orthogonaux définis par des vecteurs de longueur unité (vecteurs unitaires).
\subsection{Produit scalaire}
$a\times b = a \cdot b \cdot \cos\theta$ (norme de la projection de b sur a, ou inversément).
le produit scalaire est commutatif, distributif et linéaire. 
\\vecteurs orthogonaux : $a \cdot b = 0$. 
\\norme$^2$ d'un vecteur : $a \cdot a = |a|^2 \geq 0$
\subsection{Produit vectoriel}
direction de $a \wedge b$ normal au plan. Le sens est conventionnel (règle de la main droite, et la norme est de $|a| \cdot |b| \cdot \sin \theta$. il est anti-comutatif, distributif et linéaire. Des vecteurs parallèles a et b: $a \wedge b = 0$
\subsection{Référentiel}
\section{cinématique du point matériel}

\end{document}