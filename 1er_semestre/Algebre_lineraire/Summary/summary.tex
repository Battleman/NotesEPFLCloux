\documentclass[12pt,a4paper]{article}
\usepackage[utf8]{inputenc}
\usepackage{amsmath}
\usepackage{amsthm}
\usepackage{amsfonts}
\usepackage{array}
\usepackage{amssymb}
\usepackage{color}
\usepackage{enumitem}
\usepackage{etaremune}
\usepackage{environ}
\usepackage{fancybox}
\usepackage{gensymb}
\usepackage{graphicx}
\usepackage[colorlinks,linkcolor=purple]{hyperref}
\usepackage{mathtools}
\usepackage{setspace}
\usepackage{wrapfig}
\usepackage{xcolor}
\usepackage{xparse}

\newcommand{\somme}[2]{\ensuremath{\sum\limits_{#2}^{#1}}}
\newcommand{\produit}[2]{\ensuremath{\prod\limits_{#2}^{#1}}}
\newcommand{\limite}{\lim\limits_}
\newcommand{\llimite}[3]{\limite{\substack{#1 \\ #2}}\left(#3\right)}
\newcommand{\evid}[1]{\textbf{\underline{#1}}}
\newcommand{\ninf}{\ensuremath{n \to \infty}}
\newcommand{\xinf}{\ensuremath{x \to \infty}}
\newcommand{\xo}{\ensuremath{x \to 0}}
\newcommand{\no}{\ensuremath{n \to 0}}
\newcommand{\xx}{\ensuremath{x \to x}}
\newcommand{\Xo}{\ensuremath{x_0}}
\newcommand{\X}{\ensuremath{\mathbf{X}}}
\newcommand{\A}{\ensuremath{\mathbf{A}}}
\newcommand{\R}{\ensuremath{\mathbb{R}} }
\newcommand{\N}{\ensuremath{\mathbb{N}} }
\newcommand{\Z}{\ensuremath{\mathbb{Z}} }
\newcommand{\Q}{\ensuremath{\mathbb{Q}} }
\newcommand{\rtor}{\ensuremath{\R \to \R} }
\newcommand{\pour}{\mbox{ pour }}
\newcommand{\et}{\mbox{ et }}
\newcommand{\Exemple}{\underline{Exemple} }
\newcommand{\Theoreme}{\underline{Théorème} }
\newcommand{\Remarque}{\underline{Remarque} }
\newcommand{\Definition}{\underline{Définition} }
\newcommand{\skinf}{\sum^{\infty}_{k=0}}

\NewEnviron{boiteV}[1]{
\begin{center}
\framecolorbox{red}{white}{\begin{minipage}{#1\textwidth}
\BODY
\end{minipage}}
\end{center}}

\NewEnviron{boite}[1][0.9]{
\begin{center}
	\framecolorbox{red}{white}{
		\begin{minipage}{#1\textwidth}
  			\BODY
	\end{minipage}
	}
\end{center}
}

\author{Olivier Cloux}
\title{Algèbre Linéraire : Résumé}
\begin{document}
\maketitle
\newpage
\tableofcontents
\newpage
\section{Définitions de base}
\begin{itemize}[label=$\sqrt{.}$]
	\item un système est compatible ou consistant d'il a au moins une solution.
	\item Une \evid{Position de pivot} l'emplacement d'un coefficient principal ($\to$ colonne pivot)
	\item[Dans le plan/volume/... ?] on pose $a_1 = 
	\begin{bmatrix}
	1\\
	-2\\
	3
	\end{bmatrix}, a_2 = \begin{bmatrix}
	5\\
	-13\\
	-3
	\end{bmatrix}$ et $b = \begin{bmatrix}
	-3\\
	8\\
	1
	\end{bmatrix}$ Pour savoir si b est dans l'espace généré par $a_1$ et $a_2$, il faut chercher les solutions du système $\begin{bmatrix}
	1 & 5 & 3\\
	-2 & -13 & 8\\
	3 & -3 & 1
	\end{bmatrix}$
	\item L'équation $Ax = b$ a une solution seulement si b est une combinaison linéaire de (est dans l'espace généré par) les colonnes de A.
	\item 4 propriétés équivalentes :
		\begin{itemize}
			\item Pour tout b appartenant a $\R^m$, l'équation$Ax= b$ admet au moins une solution
			\item Tout vecteur $\R^m$ est une combinaison linéaire des colonnes de A
			\item Les colonnes de A engendrent $\R^m$
			\item Chaque ligne de A possède un pivot.
		\end{itemize}
	\item plus de colonnes que de lignes $\to$ colonnes liées (impossible d'avoir un pivot par colonne)
	\item une famille qui contient le vecteur nul est nécessairement liée/dépendante
	\item Une application (ou transformation) est linéaire si
		\begin{itemize}
			\item T(u+v) = T(u) + T(v)
			\item T(cu) = cT(u)
		\end{itemize}
\end{itemize}
\section{Déterminant}
\begin{itemize}
\item 
\end{itemize}
\end{document}