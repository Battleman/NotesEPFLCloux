\documentclass[a4paper,10pt]{article}
\usepackage[utf8]{inputenc}
\usepackage{amssymb}

%opening
\title{ICC}
\author{Brunner Loïc}

\begin{document}

\maketitle


\subsection{}
http://moodle.epfl.ch/course/view.php?id=14006
\newline
http://lacal.epfl.ch/IC-ICC
\newline
akl@epfl.ch
\newline
attention, il y aura des tests en cours de semestre
\section{introduction}
this course contain all the computer science. it will show the adcances trigger changes in society:
\begin{itemize}
 \item transportation
 \item computation
 \item communication(telephone,arpanet,web)
\end{itemize}
the question is: what is the next trigger...
\newline
the conmputer has become inconceivable, we could predict it becouse we need:
\begin{enumerate}
 \item raw computing
 \item storage capacity
 \item reliability, low cost
\end{enumerate}
\subsection{calcul scientifique}
l'utilisation dans des systèmes complexes et si on exige une grande puissance de calcul. La nouvelle tendance est à la grappe d'ordianteurs qui commencent à remplacer les super-calculateurs.
\subsection{fondement du calcul}
Les fondements du calcul viennent de besoin particuliers:
\begin{itemize}
 \item calcul algorithme(recherche, tri, plus court)
 \item stratégie de calcul(itération, réccurtions,...)
 \item théorie du calcul
 \item représentation de l'information
\end{itemize}
\subsection{a savoir}
\begin{equation}
 10^3 = 2^{10}\blacktriangleright 10^{18} = 2^{60}
\end{equation}

\subsection{computer}
they are programmable automata
\begin{description}
 \item automation:\\{device that performs a fixed task}
 \item programmable:\\{if the nature of the task can be changed as well by providing a different way}
 \item stuff:\\{numbers, textual data}
 \item operated on:\\{added, multiplied, managed, edited}
 \item task:\\{sth that had to be done}
 \end{description}

 \subsection{algorithme}
 given finite amount of data
 \newline
 some task T related to D
 \newline
 an algorithme is a finite sequence of precise operations, well-defined by D, that results in completion of T
\paragraph{}
algorithme have to be:
 \begin{itemize}
  \item sequential
  \item parallel
  \item distributed
 \end{itemize}
algorithme not the same as programm. It a conceptual, human-readable solution. Programme works on computers. Programms use algorithmes.

\subsection{basic ingredient of algorithmes}

\begin{description}
 \item variables:\\{a piece of data}
 \item assignment:\\{variable is changed using an assignment}
 \item controle structure:\\{to allow instructions to be carried out in a data-dependent manner(conditional,count-controller loop,condition controller loop}
\end{description}

ideally, the algorithme terminate and performs the right task. Meticulous work with a serious reflexion.

\subsection{common algorithme and tools}
\begin{description}
 \item searching:\\{searching x for a list of e elements}
 \item big O:\\{used to compare the algorithmes}
 \item optimisation:\\{shortest, cheapest, fastest, most profitable}
\end{description}
the complexity of the algorithme depends of the operations that are use $O(2^n)$ is the worse. The complexity of sorting things is $O(n^2)$.

\subsection{the big O}
 f is the O(g)   f,g R$\longrightarrow R$
 \newline
 $\exists x_0,c\forall x>=x_0$  $|f(x)| \leq c|g(x)|$
 \newline
 $\forall k>0\forall \varepsilon>0$ $(log(n))^k$ is $O(n^2)$
 
 \section{common approache to solve problems}
 To solve problems, ofter we hasad to separate them into pieces. Every problem came of a subproblem which is mayby wrong...
 
 \paragraph{}
 sur les diaporamas il y a tout alors fait pas chier
 
 \paragraph{complexity}
 \begin{enumerate}
  \item O(i)
  \item =O(log(n))
  \item O(n)
  \item O(nlog(n))
  \item O(n$^2$)
  \item $O(n^3)$
 \end{enumerate}
 
 regarde les diapos boulet...
 \newline
 Fais gaffe quand même parce que les additions binaires sont importantes ;)
 \newline
 Bon reprenons sérieusement:
 \section{signals}
 \begin{description}
  \item signal:\\{just a function from $R^n\mapsto R^m$}
 \end{description}
Nous allons considérer a tord que tout signal est une somme de sinus
\begin{description}
\item fitre:\\{transform the function to reduce the "nose"}
\end{description}
Floyd d(i,j) distances between cities <i,j, $1\leq i,j\leq n$
\newline
first allow just city, as intermediate city. $\forall i,j d(i,j)=min(d(i,j),d(i,1)+d(1,j))$
\newline
now allow both cities 1 and 2 $\forall i,j:d(i,j)=min(d(i,j),d(i,2)+d(2,i))$
\newline
continue for all cities...
\newline
reconstruire un signal à partir de sa forme échantillonée(interpolation)polynome de Lagrange (oublie pas le colège, je me demande si tu en auras vraimetn besoin mais bon...;)
\newline
Mais on utilise d'autres techniques, cf le cours
\section{partie II du cours}
une sinusoide:X(t)=$asin(2f*pi+\delta)$
\begin{itemize}
 \item a=amplitude
 \item f=fréquence
 \item $\delta$= déphasage
\end{itemize}
\paragraph{signaux en général}
Nous supposons que tout signal est une somme de sinusoides
\begin{itemize}
 \item période T($\frac{1}{f}$)
\end{itemize}
\subsection{les filtres passe-bas}
suppression des hautes fréquences d'un signal. En gros tu supprime tous les sinus qui ont une fréquence plus gandes que la fréquence de coupure $f_c$, plus simple tu meurs
\subsection{filtre à moyenne mobile}
X'(t)=$\frac{1}{T_c}\int^t_{t-T_c}X(s)ds$
\paragraph
Avec un tel filtre nous voyons venir une borne supérieure:$\frac{1}{pi*f*T_c}$
\subsection{échantillonage}
Nous devons choisir la bonne période d'échantillonage sinon nous avons soit triop d'info à traiter, soit pas assez d'info pour recréer le signal
\paragraph
pour pouvoir reconstruire la sinisoide à partir de sa version échantillonée: $f_e>2f$
\subsubsection{reconstruction d'un signal}
Nous avons une  formule d'interpolation:$X_1(t)=\sum_{n\in Z}X(nT_e)sinc(\frac{1-nT_e}{T_e})$
\subsection{compression de données}
Nous voulons éliminer les redondances qui ne sont pas utiles pour faciliter le stockage et la transmission
\subsection{entropie}
est définie par:H(x)=$p_1log_2(\frac{1}{p_1})+...+p_nlog_2(\frac{1}{p_n})$
\paragraph
Pour deviner qu'une lettre apparaît avec une probabilité p, on a besoin de $log_2(\frac{1}{p})$ questions
\begin{description}
 \item entropie:\\{mesure la qté d'information contenue dans un signal}
\end{description}




\end{document}

 
