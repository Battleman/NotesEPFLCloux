\documentclass[a4paper,10pt]{article}
\usepackage[utf8]{inputenc}
%\usepackage{eurosym}
\usepackage{graphicx}
\usepackage{amssymb}
\usepackage{amsmath}


%opening
\title{discrete structure}
\author{}

\begin{document}

\maketitle

\part{introduction}
exercices: igp.epfl.ch

\section{probabilités et conditions}

introdution au discrétes structures. Comporte des probabilités. On parle également d'assutiations de conditions. Va nous servir au condition sen programmation.
on résoud aussi des trucs logiques: $m+n < 2k$ alors c'est vrai si $m<k$ et $n<k$. Présentation de certains problèmes de logique.
\subsection{exemple1}
nous voyons plein d'exemples de logique:
\newline
 il existe un x pour tout y /= pour tout y il existe un x
\subsection{commentaire}
exemple de logique classique
\newline
la logique sera approfondie dans le cours
\subsection{exemple2}
f(x) root, x element of R
\newline
f(x) is continuous
way to see why f(x) is contunous: if $|x-X| < \theta\\ than |f(x) -f(X)| < \theta$
it exists $\varepsilon >0 if if |x-X| < \varepsilon\\ than |f(x) -f(X)| < \varepsilon$
\subsection{commentaire}
we can look at the graph but... that's a other exemple of logic(acording to a graph/function)

\section{propositional logic}
\begin{description}
 \item proposition{statment that can be true or false}
 \item compotitions of proposition{proposition put together}
 
\end{description}
logical operands:
\begin{itemize}
 \item and: p $\wedge$ q: si les deux sont les memes: it's a conjonction
 \item or: p $\vee$ q: si un des deux est vrai:it's a disjonction
 \item xor(exclusiv or):  poq: si les deux sont différents
 \item not :if p not q: si le second est faux ou si les deux sont les mêmes
 \item conditional operand: if/only if p than q
\end{itemize}

\section{truth tables}
they are propostions that are true or false in function of the logical operands:
\newline 
with: p not q
\newline
if p is true, q is false and if p is false q is true
\paragraph
 to see how the operands works, look at logical operands' description. the xor operand is special: for p and q
\newline
if pt and qt:f
\newline
if pt and qf:t
\newline
if pf and qt:t
\newline
if pf and qf:f
\paragraph{}
we ca make assotiations of the operand!!!

\begin{itemize}
 \item totology $=>$ always true
 \item contraditon $=>$ always false
 \item contrigency $=>$ sometime true sometimes false
\end{itemize}

observation: these 2 comparands have the same truth table: p->q / $not\\p\vee q$
\paragraph{}
these two operation are voquivalent because they have the same truth table. We write:$p\equiv q$
\newline
if $ p->q\\ and\\ p\equiv q$ $p->q$ is a totology.

\begin{tabular}{|c|c|c|c|c|c|}
 p & q &not q& $not p \vee q $&$ p->q$ & $(not p \vee q) <-> (p->q)$\cr
 \hline
 T&T&F&T&T&T\cr
 \hline
 T&F&F&F&F&T\cr
 \hline
 F&T&T&T&T&T\cr
 \hline
 F&F&T&T&T&T\cr
\end{tabular}
We have here a good exemple of totology. The solution is always true. So $(p->q)\equiv (not\\ p \vee q)$
\paragraph{exemple:}
De Morgen's laws:
\newline
\[not\\ (p\vee q \equiv\\ not\\ p\vee not\\ q\]
\[not\\ (p\vee q \equiv\\ not\\ p\wedge not\\ q\]
\begin{tabular}{|c|c|c|c|c|c|c|}
p&q&not p&not q &$ p \vee q$& $not(p\vee q$)&$ not p \wedge not q$\cr
\hline
T&T&F&F&T&F&F\cr
\hline
T&F&F&T&T&F&F\cr
\hline
F&T&T&F&T&F&F\cr
\hline
F&F&T&T&F&T&T\cr
\end{tabular}

\subsection{table of basic equivalences}
domination laws:
\newline
$p\wedge q \equiv p$
\newline
$p \vee F \equiv p$
\newline
$p\wedge F \equiv F$
\newline
$p\vee T \equiv T$
\paragraph{}
idempatent laws:
\begin{description}
 \item $P \vee p \equiv p$
 \item $p \wedge p \equiv p $
\end{description}

Double negation:
\begin{description}
 \item $P not ( not p) \equiv p$
\end{description}

commutative laws:
\begin{description}
 \item $P \vee p \equiv q \vee p$
 \item $p \wedge p \equiv q \wedge p $
\end{description}

associative laws:
\begin{description}
 \item $ (p \vee p) \vee r \equiv p \vee (q \vee r)$
 \item $ (p \wedge p) \wedge r  \equiv p \wedge ( q\wedge r) $
\end{description}

distributive laws:
\begin{description}
 \item $P \vee (q \wedge r) \equiv (p \vee q) \wedge (p \vee r)$
 \item $P \wedge (q \vee r) \equiv (p \wedge q) \vee (p \wedge r)$
\end{description}
absorbtion laws:
\begin{description}
 \item $ p\vee (p\wedge q) \equiv p$
 \item $ p \wedge (p\vee q) \equiv p$
\end{description}
negation laws:
\begin{description}
 \item $ p\wedge not p\equiv T$
 \item $ p \vee not  p\equiv F$
 \end{description}
 
 implication law:
 \begin{description}
  \item $p\longrightarrow q\equiv notp\vee q$
 \end{description}

\[ not(p\vee (not\\ p \wedge)) \equiv\\ not\\ p\wedge \\not\\ q\]
\[ not(p\vee (not\\ p \wedge)) \equiv\\ not\\ p\wedge \\not\\ (not\\ p \wedge q)\]
\[ not(p\vee (not\\ p \wedge)) \equiv\\ not \wedge ( \\not (not\\ p)\vee \\not\\ q)\]
\[ not(p\vee (not\\ p \wedge)) \equiv\\ not\\ p \wedge(p \vee \\not\\ q)\]
\[ not(p\vee (not\\ p \wedge)) \equiv\\ (not\\ p \wedge q) \vee (not\\ p \wedge not\\ q)\]
\[ not(p\vee (not\\ p \wedge)) \equiv\\ F \vee (not(p \wedge \\not \]
\[ not(p\vee (not\\ p \wedge)) \equiv\\ (not\\ p \wedge \\not\\ q) \vee F\]
\[ not(p\vee (not\\ p \wedge)) \equiv\\ not\\ p \wedge \\not\\ q \]

\paragraph{note}
Disjonctive Nonnel form:
\[(p_1\wedge p_2\wedge p_3)\vee (notp_1 \wedge p_2 \wedge ...)\vee (...)\]

\section{predicats}
\begin{description}
 \item predicate:\\{ propositional function depending on some variables over a specified domain so that if we plug in specific calues for these variables we get propositions}
\end{description}
exemple:
\begin{math}
 P(x)=x>3, x \in R
 \newline
 P(x=2)= 2>3\->F
 \newline
 P(x=4)=4>3->T
 \newline
 P(x,y)=x>=3y^2
 \newline
 P(3,2)=3>=3*2^2->F
\end{math}
 we have to set a domain R(...,...)
\section{quantifiers}
\subsection{universal quantifiers}
$\forall xP(x)$ P(x) predicate
\newline
$\exists xP(x)$ p(x) predicate

True if for all x inthe domain
\newline
P(x) is ture
\newline
if the domain is finite then this is equivalence: $(Px_1) \wedge P(x_2 \wedge p(x_3)....$

\subsection{existantial quantifiers}
$\exists xP(x)$ P(x) predicate
\newline
This is true if and only if there exists an element x in the domain so that P(x) is true.
\newline
If the domain is finite than this is equivalent to $ P(x_1) \vee P(x_2),.... \vee p(x_n) $
\paragraph{exemple}
\begin{math}
 P(x,y)=x>=3y^2\\ x,y\in R
 \newline
 \forall x \exists y /P(x,y)=\forall x \exists y x>=3y^2 \\ \\ F
\end{math}
counter exemple: x = -1
\subsection{Negative quantifiers}
not$\forall xP(x)\equiv \forall xnotP(x)$
\newline
\[not\exists xP(x)\equiv \forall xnotP(x)\]
we say that 2 propositions involving predicate and quantifiers are equivalent if they have the same truth value irrespective if what predicate we insert irrespective of the domain of discours.
\newline
$not(\forall xP(x))$ is true if and only if $ \forall xP(x)$ is false. this is a predicate to say that there exist an $x \in D$ that P(X) is false equivalantly there exists an $x\in D$ that $notP(x)$ is true. But this is equivalent to $\exists xnotP(x)$.
\paragraph{proof}
if D was finite, than $\forall x(P(x)\wedge Q(x))\equiv (P(x_1)\wedge(Q(x_1)9\wedge P(x_2)\wedge(Q(x_2)\wedge...$
\newline
on joue avec les parentaises
\newline
$\equiv (\forall xP(x))\wedge (\forall xQ(x))$
\paragraph{our strategy}
show that whenever the left side has a truth value T than so does the right side and then show whenever the right side has a truth value T than so does the left side.

\paragraph{}
$\forall (x(P(x)\wedge Q(x))$: assume that true in words is true for all $x\in D$, hence P(x) is true for all x and so is Q(x).Hence ($\forall xP(x))$ is true and $(\forall xQ(x)$ is true Hence their conjunction,$ (\forall xP(x))\wedge (\forall x Q(x))$ is true.

\subsection{rules of infernce}
\paragraph{hypotheses assumption}
proposition:If (I take the DS course)p, I (am smart)q.
\newline
proposition: I take the DS course
\paragraph{argument form}
$p\longrightarrow q$ is true
\newline
$p$ is true
\newline
this is named a modus pnens.
\paragraph{}
we can say that q is true: $((p\longrightarrow q)\wedge p)\longrightarrow q$: should be a totology (always true)
\newline
We say: if the 2 arguments are correct, q is correct. If this is a totology we say that the argument form is valid.
\newline
\begin{tabular}{|c|c|c|c|c|}
 p&q&$p\longrightarrow q$&$(p\longrightarrow q\wedge p)\longrightarrow q$\cr
 \hline
 T&T&T&T&T\cr
 \hline
 T&F&F&F&T\cr
 \hline
 F&T&T&F&T\cr
 \hline
 F&F&T&F&T\cr
\end{tabular}
\paragraph{modus tollens}
$p\longrightarrow q$ is true
\newline
$p$ is true
\newline
$notq$
\newline
$\equiv notq\longrightarrow notp$
\paragraph{addition}
p
\newline
$p\vee q$
\paragraph{simplification}
$p\wedge q$
\newline
p
\paragraph{conjunction}
p
\newline
q
\newline
$p\wedge q$

\paragraph{hypothetic of syllogism}
$p\longrightarrow q$
\newline
$p\longrightarrow r$
\newline
$p\longrightarrow r$

\paragraph{resulution rule}
$p\vee q$
\newline
$notp\vee r$
\newline
$q\vee r$

\begin{tabular}{|c|c|c|c|c|c|c|c|}
 p&q&r&$p\vee q$&$notp\vee r$&$(p\vee q)\wedge (notp\vee r)$&$()-> (notq\wedge r)$&tout\cr
 \hline
 T&T&T&T&T&T&T&T\cr
 \hline
 T&T&T&F&T&F&F&T\cr
 \hline
 T&T&F&T&T&T&T&T\cr
 \hline
 F&T&F&F&T&F&F&T\cr
 \hline
 T&F&T&T&T&T&T&T\cr
 \hline
 T&F&T&F&T&T&T&T\cr
 \hline
 T&F&F&T&F&T&F&T\cr
 \hline
 F&F&F&F&F&T&F&T\cr
\end{tabular}
\newline
attention, il est fort probable que le tableau contienne une erreur!

\paragraph{universal instatuation}
$\forall xP(x)$
\newline
P(y) for $y\in D$

\paragraph{universal generalisation}
P(y) for every $y \in  D$
\newline
$\forall xP(x)$

\paragraph{existencial instatutation}
$\exists xP(x)$
\newline
P(y) for some $y\in D$

\paragraph{existencial generalisation}
P(y) fore some $y\in D$
\newline
$\exists xP(x)$

\paragraph{hamburger proof}
\begin{itemize}
 \item someone in this class visited the US
 \item anyone who visited the Us likes hammburger
\end{itemize}
(therefore) someone in this class likes hamburger.
\newline
D = people
\newline
C(x)= x is in this class.
\newline
V(x)= x has visited the US.
\newline
L(x) = x likes hamburger.
\begin{enumerate}
 \item $\exists x$ $C(x)\wedge V(x)$ 
 \item $forall x$ $V(x)\longrightarrow L(x)$
\end{enumerate}
$\exists x$ $C(x)\wedge L(x)$
\paragraph{demonstration}
\begin{enumerate}
 \item $\exists x$ $C(x)\wedge V(x)$
 \item $\forall x$ $V(x)\longrightarrow L(x)$
 \item C(y)$\wedge$V(y)
 \item V(y)$\longrightarrow$ L(y)
 \item V(y)
 \item L(y)
 \item C(y)
 \item C(y)$\wedge$L(y)
 \item $\exists x$ $C(x)\wedge L(x)$
\end{enumerate}

(1,2) Are the hypotheses of the proof.
\newline
(3,4) instantiation rule 
\newline
(5)simplification rule 
\newline
(6)modus ponens(4+5)
\newline
(7)simplifiaction(3)
\newline
(8)conjunciton(6+7)
\newline
(9)existencial generalisaiton(8)
\paragraph{claim}
Let n,m and k integer. If $m+n\geq$ or $n\geq k$
\newline
Proof:
$p=\forall k:(n+m\geq 2k)\longrightarrow((m\geq k)\vee(n\geq k))$
\newline
Look at notp and show that its truth value is F...
\paragraph{}
$notp=not(not(n+m\geq 2k)\longrightarrow((m\geq k)\vee(n\geq k))\equiv \exists k$ $not((m+n\geq 2k)\vee(m\geq k\vee n\geq k))$
\newline
$\equiv \exists k$ $(not(m+n\geq 2k)\wedge not(m\geq k\vee n\geq k))$
\newline
$\equiv \exists k$ $((m+n\geq 2k)\wedge (not(m\geq k)\wedge not(n\geq k)))$
\newline
$\equiv \exists k$ $((m+n\geq 2k)\wedge ((m< k)\wedge (n< k)))$:contradition(m,n est plus petis que k, mais ils doivent être plus grands que 2k.)
\paragraph{exemple}
$\exists !xP(x)$= there is a unique xP(x)
\newline
Question:$not(\exists !xP(x))=$ there is either no x with property P(x) or there are at least two with this property.
\paragraph{}
Proof:
\newline
$\exists xP(x)\wedge \forall y((x\neq y)\longrightarrow notP(y)))$
\newline
$not(\exists xP(x)\wedge \forall y((x\neq y)\longrightarrow notP(y))))\equiv \forall x notP(x)\vee\exists ynot((x=y)\vee notP(y))$
\newline
$\equiv \forall x notP(x)\vee\exists y((x\neq y)\wedge P(y))$

\subsection{different types of proof}
\begin{description}
 \item Direct proof:\\{$p\longrightarrow q$}
 \item indirect proofs:\\{$notq\longrightarrow not p$ contraposition/contradiciton($p\longrightarrow notq)$}
 \item non constructive proof:\\{view exemple}
\end{description}

\paragraph{claim(using comntraposition}
if N and m are integers and mn is even then is n even or m even.
\newline
Proof:
\newline
p=mn is even
\newline
q= n is even
\newline
r =m is even
\paragraph{}
$p\longrightarrow q\vee r$
\newline
$not(q\vee r)\longrightarrow notp$
\newline
$notq \wedge notr \longrightarrow notp$
\newline
if q and r are both odd then their pruduct is odd (m=2a+1 and n=2b+1)
\newline
$mn=(2a+1)(2b+1)=4ab+2a+2b+1=2(2ab+a+b)+1$ there odd!!!
\paragraph{exemple(non constructive)}
There exist irrational number x and y so that $x^y$ is irrational
\newline
Proof: Non constructive
\newline
$\sqrt{2}$ is irrational
\newline
$x=y=\sqrt{2}$
\newline
the question is: $\sqrt{2}^{\sqrt{2}}$ irrational
\newline
P(x,y)=$x^y$ is rational: $P(\sqrt{2},\sqrt{2})\vee P(\sqrt{2},\sqrt{2})$ has a truth value T.
\paragraph{pigeonhole principle}
n boxes and n+1 pigeons. AT least two pigeons share a box.
\newline
Graph G=(V,E)
\newline
V=$\{v_1,v_1,...,V_n\}$ set of veatices
\newline
E= set of edges ($V_2,V_5$) unordered triple.
\includegraphics{/home/logeek04/Documents/etude/discretesS/Schema1.jpg}
\newline
Claim: In a simple graph there exists two veatrices of the same degree as long as $|V|\geq 2$(the degree is the number of connection)
\newline
Proof:
\newline
note: deg($v_i)\leq n-1$
\newline
$v_1$ n'a pa de degré/connection et $v_n$ en a n-1, nous notons donc un décalage.

\section{Sets}
\begin{description}
 \item set:\\{is an unordered collection of objects}
\end{description}
\paragraph{exemple}
A is a set of primes less than 13.
\newline
$A=\{3,2,7,5,11\}=\{7,11,1,3,2,5\}$
\paragraph{exemple2}
B is a set of non-negative integers less or equal to 100.
\newline
$B=\{12,9,54,87,19,23,...,100\}$
\subsection{sete builder location}
$S=\{x:P(x)\}$ P(x) is a predicate $=\{x|P(x)\}$
\newline
domain must be specified
\newline
Note:$S=\{x:P(x)\}$ {$\forall x(x\in S\Leftrightarrow P(x))$.
\paragraph{exemple}
domain:integers
\newline
$A=\{x:P(x)\}$ P(x)= x is prime and x is smaller than 13
\newline
P(2)=T
\newline
P(8)=F
\paragraph{exemple2}
domain:integers
\newline
$B=\{x:Q(x)\}$ Q(X)= x is non negative and x is not most of 100.
\newline
Q(99)=T
\newline
Q(321)=F
\subsection{cordinality of a set}
Recall set A from before
\newline
$|A|=\#A=$ number of objet contain in A.
\paragraph{exemple}
$|A|=5$ $|B|=|0|$
\newline
soit $A=\{1,2,3\}$ so $\#A=3$ and soit $B=\{1,6\}$ so $\#B=2$ $\#(AuB)=5$
\newline
$|N|=\infty$ not the same infiniy of $|R|=\infty$ for exemple
\paragraph{}
$\varnothing$ empty set $\varnothing=\{\}$
\subsection{equality of sets}
A=B
\newline
A=B means that $x\in B$ and vice versa $\forall x(x\in A\Leftrightarrow x\in B)$
\newline
subset
\newline
$A\subseteq B$ A is a subset or equal of B $\forall x(x\in A\Rightarrow x\in B)$
\newline
$A\subset B$ propre subset
\newline
if $A\subset B\wedge (\exists y$ $y\in B\wedge y \notin A)$
\paragraph{theoreme}
For every A,$\varnothing\subseteq\leq A$ so $ A \subseteq\leq A$
\newline
Proof: $\varnothing \subseteq$ $\forall x(x\in\varnothing \Rightarrow x\in A)$
\subsection{power set}
P(A) $A=\{1,2,3\}$ is th eset whose elements are the subset of A (not nessesery all)
\newline
P(A)=$\{\varnothing,\{1\},\{1,2,3\},\{1,2\},...\}$
\newline
|P(A)|=|A|!+2=$2^{|a|}$
\newline
$P(\varnothing)=\{\varnothing\} \varnothing\subseteq\varnothing$
\newline
$P(P(\varnothing))=P(\{\varnothing\})=\{\varnothing,\{\varnothing\}\}$
\newline
$P(P(P(\varnothing)))=\{\varnothing,\{\varnothing\},\{\{\varnothing\}\},\{\varnothing,\{\varnothing\}\}\}$
\subsection{set operations}
A$=\{x:P(x)\}$
\={A} complements; this only makes sense if we have specified domain.
\newline
U is the universal set.
\newline
\={A}$=\{x:x\notin A\}=\{xnot(x\in A)\}$
\newline
AuB=$\{x:x\in A \vee x\in B\}$
\newline
AnB=$\{x:x\in A \wedge x\in B\}$
\newline
A/B=A-B=$\{x:x\in A \wedge x\notin B\}$
\newline
$A\vartriangle B=\{x:x\in A \oplus x\in B\}$
\subsection{set identifies}
\={AuB}=\={A}n\={B} de morgen's laws
\paragraph{menbership table}
\begin{tabular}{|c|c|c|c|c|c|c|}
A&B&AuB& \={AuB}&\={A}&\={B}& \={A}n\={B}\cr
0&0&0&1&1&1&1\cr
0&1&1&0&1&0&0\cr
1&0&1&0&0&1&0\cr
1&1&1&0&0&0&0\cr
\end{tabular}
so we can see that \={A}\={u}\={B}=\={A}n\={B}
\newline
\={AuB}$\subseteq$ \={A}n\={B}
\newline
\={AuB}$\supseteq$ \={A}n\={B}
\paragraph{}
proof:
\={AuB}=\{x:not((x$\in$ A)$\vee$(x$\in$ B))\}=\{x:(not(x$\in$ A)$\wedge$ not(x$\in$ B)\}
\newline
=\{x:(x$\notin$ A)$\wedge$ $(x\notin B)\}=\{x:(x\in $\={A}$) \wedge(x\in $\={B})$\}= $\={A}n\={B}
\paragraph{note}
the best way to visualise the relationship between the set and the a subset is to make au schema. not a proof
\subsection{inclusion -exclusion}

\includegraphics{/home/logeek04/Documents/etude/discretesS/Schema2.jpg}

$|AuB|=|A|+|B|-|AnB|$
\paragraph{exemple}
A=$\{x\in Z:0\leq x\leq100;$ x is divisible by $7\}$
\newline
B=\{$x\in Z:0\leq x\leq100;$ x is divisible by 3\}
\newline
AuB=\{$x\in Z:0\leq x\leq 100,7/x\vee 3/x$\}
\newline
$|AuB|$=$|A|+|B|-|AnB|$
\newline
AuB=\{$x\in Z:0\leq x\leq 100,7/x\wedge 3/x$\}=\{$x\in Z:0\leq x\leq 100;21/x$\}
\newline
$|A|=15$, $|B|=34$, $|AuB|=44$
\paragraph{}
Proof:
\newline

\includegraphics{/home/logeek04/Documents/etude/discretesS/Schema3.jpg}
show that for disjoint sets A and B, the coordinality of the union is the sum of the coorinalities.
\newline
Claim:
\newline
AuB=(A/B)\r{u}(AnB)\r{u}(B/A)
\newline
(A/B)n(AnB)=$\varnothing$
\newline
/(A/B)m(B/A)=$\varnothing$
\newline
(AnB)(B/A)=$\varnothing$
\newline
\paragraph{Proof by membership table}
\begin{tabular}{|c|c|c|c|c|c|c|}
 A&B&A/B&AnB&B/A&(A/B)u(AnB)u(B/A)&AuB\cr
 \hline
 0&0&0&0&0&0&0\cr
 \hline
 0&1&0&0&1&1&1\cr
\hline
 1&0&1&0&0&1&1\cr
 \hline
 1&1&0&1&0&1&1\cr
\end{tabular}
$|AuB|=|A/B|+|AnB|+|B/A|$
\newline
Using:$|A|=|A/B|+|AnB|$
\newline
we can see that:$|A|-|AnB|+|AnB|+|B|-|AnB|=|A|+|B|-|AnB|$
\paragraph{an other demonstration}
$|AuBuC|=|A|+|B|+|C|-|AnB|-|BnC|-|AnC|+|AnBnC|$

\includegraphics{/home/logeek04/Documents/etude/discretesS/Schema4.jpg}
\newline
we use:BuC=D
\newline
$|AuD|=|A|+|BuC|-|An(BuC)|=|A|+|B|+|C|-|BnC|-|(AnB)u(AnC)|=|A|+|B|+|C|-|BnC|-|AnB|-|AnC|+|AnBnC|$
\paragraph{exemple}
A=\{$x\in Z:0\leq x\leq 100, 7/x$\}$|A|=15$
\newline
B=\{$x\in Z:0\leq x\leq 100, 3/x$\}$|B|=34$
\newline
C=\{$x\in Z:0\leq x\leq 100, 5/x$\}$|C|=21
$\newline
$|AnB|=5,|BnC|=3,|AnC|=7,|AuBuC|=56$...
\paragraph{recall}
$|A$\r{u}$B|$=$|A|+|B|$ we call that a disjoint union (AnB=$\varnothing$)
\paragraph{Function}
$|A|=|B|$ if and only if there exists a bijection f:$A\longrightarrow B$
\begin{itemize}
 \item f:$A\longrightarrow\{1,2,3,..,n\}$
 \item f:$A\longrightarrow\ N(countable)$
\end{itemize}
f assigns for every elements in A a unique element B.
\includegraphics{/home/logeek04/Documents/etude/discretesS/Schema5.jpg}
\newline
For the notation:
\newline
f(A) is the range of f
\newline
f(A)$=\{b\in B,\exists a\in A/f(a)=b\}$
\newline
$f(A)\subseteq B$
\paragraph{}
f:$R\longrightarrow R$ f:$A\longrightarrow B$
\newline
g:$R\longrightarrow R$
\newline
h=f+g with h:$R\longrightarrow R$
\newline
\begin{itemize}
 \item f:$x\longrightarrow f(x)$
 \item g:$x\longrightarrow g(x)$
  \item h:$x\longrightarrow f(x)+g(x)$
 \end{itemize}
\paragraph{}
h=fg
\newline
h:$x\longrightarrow g(x)f(x)$
\paragraph{composition}
 f:$A\longrightarrow B$
 \newline
 g:$B\longrightarrow C$
\newline
gof:$A\longrightarrow C$
\newline
gof:$x\longrightarrow g(f(x))$
\includegraphics{/home/logeek04/Documents/etude/discretesS/Schema6.jpg}
\paragraph{}
f can be increasing.
\newline
f:R$\longrightarrow$R
\newline
$\forall x_1\forall x_2,x_2>x_1\longrightarrow f(x_2)\geq f(x_1)$
\newline
f is strictly increasing
\newline
it also works with a decreasing f.
\paragraph{}
\includegraphics{/home/logeek04/Documents/etude/discretesS/Schema7.jpg}
\begin{description}
 \item f is injectitve:\\{$f(a_1)=f(a_2)\longrightarrow a_1=a_2/a_1+a_2\longrightarrow f(a_1)\neq f(a_2)$}
 \item f is surjective:\\{$f(a)=B/ every argument of B is hit at lest once/\forall b\in B,\exists a\in A/f(a)=b$}
 \item f is bijective:\\{injective + surjective}
\end{description}
if f is bijective, we can define the inverse function $f^{-1}:B\longrightarrow A$ f(a)=b$\Leftrightarrow$ $f^{-1}(b)=a$ and $f^{-1}$ is also bijective.
\newline
$f^{-1}of:A\longrightarrow A$ $x\longrightarrow x,x\in A$
$fof^{-1}:B\longrightarrow B$ $x\longrightarrow x,x\in B$
\subsubsection{special functions}
\begin{itemize}
 \item rounding $\llcorner x\urcorner \in R$ who give the closest integer
 \item Ceiling: $\ulcorner x\urcorner$ smalest integer$ \geq x$
 \item Floor: $\llcorner x\lrcorner$ largest integer$\leq x$
 \item factorial: n!=$\prod^n_{n=1}$
\end{itemize}
\paragraph{exemple}
$\llcorner x\lrcorner=\llcorner x\lrcorner+\llcorner x+\frac{1}{3}\lrcorner+\llcorner\lrcorner x+\frac{2}{3}\lrcorner$
\newline
Proof:
\newline
$x=n+\varepsilon$ alors:$3x=3n+3\varepsilon$ $n\in Z,\varepsilon$ fractional part
\newline
$\llcorner 3x\lrcorner=\llcorner3n+3\varepsilon\lrcorner=3n$
\newline
$\llcorner x\lrcorner=\llcorner n+\varepsilon\lrcorner=n$
\newline
$\llcorner x+\frac{1}{3}\lrcorner=\llcorner n+\frac{1}{3}+\varepsilon\lrcorner=n$
\newline
$\llcorner x+\frac{2}{3}\lrcorner=\llcorner n+\frac{2}{3}+\varepsilon\lrcorner=n$
\newline
if $\frac{1}{3}\leq\varepsilon\leq\frac{2}{3},then$ $\llcorner 3x\lrcorner=3n+1$
\newline
else if $\frac{2}{3}\leq \varepsilon \leq 1$ $3n+2$
\subsection{cardinality}
\begin{description}
 \item two sets A and B have the same cardinality:\\{if and only if $\exists$ a bijection f from A to B, the cardinality is the formal word to say the size of a set}
 \end{description}

\begin{itemize}
 \item We say that the set A has a cardinality n,$n\in N$, if $\exists$ a bijection f between A and \{1,2,3,...,n\}
 \item We say that A is countable if it is either finite or $\exists$ a bijection f between A and N=\{1,2,3,...,n\}
 \item N is said to have cardinality: (aleph N:)$N_0$
 \item if A is neither finite nor countable then we say that it is uncountable
\end{itemize}
\paragraph{exemples}
A=\{0,1,2,3,....\}= $N_{>0}u\{0\}$
\newline
the cardinality $|A|=|N_{>0}|=N_0(aleph)$
\paragraph{Hilbert's Hotel}
We have an hotel with infinitive many room but it's full. A new client arrives and he can go into the hotel in the forst room. the man who was in the fisrt goes in the two,...
\subsection{yeah}
Let A and B be two finite set and let AnB=$\varnothing$. Then $|AuB|=|A|+|B|$
\newline
Proof:
\newline
Assume that $|A|=a and |B|=b$ a,b$\in N$
\paragraph{}
$f_a:A\rightarrow\{1,2,3,..,a\}$ f a bijection
\newline
$f_b:B\rightarrow\{1,2,3,..,a\}$ f a bijection
\newline
We need to proove that:
\newline
$f_{a,b}:AuB\rightarrow\{1,2,3,..,a+b\}$ f a bijection
\paragraph{}
$f_a(x), x\in A: 1,2,3,...,a$
\newline
$a+f_b(x), x\in B: 1,2,3,...,a,a+1,a+1,...,a+b$
\subsubsection{cardinalities}
$|N|=N_0$
\newline
$|Z|=N_0$
\newline
$|Q|=N_0$ (we can count with a tabular(c.f. analyse N est dénombrable))
\newline
As it has been seen in analyseI, the R is uncountable so £$|R|\neq N_0$ As the prof said:"The reals cannot be put in one-to-one correspondence with natural numbers. R is infinite, it is uncountable".
\newline
$|[-2;0[|=|R|$ Consequence of the definiton.
\subsubsection{for the Naturals}
We creat a functin to count the Nnumbers: if x positive 2x and if x negative -(2x+1)
\newline
Proove that f:$Z\rightarrow N$ is a bijection.
\newline
$f^{-1}:N\rightarrow Z$
\paragraph{}
$f^{-1}of=Z\rightarrow Z$
\newline
$f^{-1}of=id_z$
\newline
$fof^{-1}=N\rightarrow N$
\newline
$fof^{-1}=id_N$
\paragraph{$z \geq 0$}
$\llcorner\frac{2z+1}{2}\lrcorner=z$(after simplification)
\paragraph{$z\leq 0$}
$-|z|=z$(after simplification)

\subsubsection{for the reals}
By contradiction:
\newline
Assume: $f:(N\rightarrow (-\frac{pi}{2};\frac{pi}{2})$
\newline
g:$(-\frac{pi}{2};\frac{pi}{2})\rightarrow (0;1)$
\newline
$f:N\rightarrow(0;1)$ is a bijection.
\begin{itemize}
 \item $x_1=f(1)$
 \item $x_2=f(2)$
 \item $x_3=f(3)$
 \item $x_4=f(4)$
\end{itemize}
$x_1=0.x_{11}x_{12}....$  
$x_2=0x_{21}x_{22}....$
\newline
if we take the diagonale, we can build an extra number which is not in the list(it's named Cantor's diagonalization construction):
$x_+=0.x_{11}x_{22}......$
\paragraph{}
we are in contradiction because can't list the reals.
\subsubsection{exemple}
Set of real numbers with exclusively digit 7.
\newline
A:=\{7,7.7,7777.7777777,77.777,77,.....\}
\newline
infinite but countable!!!:
\newline
We order this by lenght (7=1,77,7.7=2,...) and we can constructt different finite sets with this.
\subsubsection{exemple}
Same as before but allow digits 7 and 8.
\newline
a:=\{7,8,77.88,87,7,..\}
\newline
infinte and uncountabe: we can constuct the Cantor's diagonalization.
\subsection{sequences}
1,2,3,4,5,6,...
\newline
$1,\frac{1}{2},\frac{1}{4},\frac{1}{8},...$
\paragraph{formelly}
$f:A\mapsto B$
\newline
$A\subseteq N$
\subsubsection{commun sequences}
\paragraph{Integer sequences}
\begin{description}
 \item Natural sequances:\\{1,2,3,4,....}
 \item even numbers(positive):\\{2,4,6,8....}
 \item factorial:\\{1,2,6,24,120,...}
 \item primes numbers:\\{2,3,5,7,11,13,...}
 \item Fibonacci numbers:\\{0,1,1,2,3,5,8,...}
\end{description}
\paragraph{arithmetic sequence}
$f(n)=a+nb/a,b\in R$
\newline
$\sum_{n=0}^{k}(a+bn)=(k+1)(a+\frac{bk}{2})$ (using dafuck plus loin...)
\paragraph{Geometric sequence}
$f(n)=ab^n/a,b\in R$
\newline
$\sum_{n=0}^(k)ab^n=a\frac{b^{k+1}-1}{b-1}$ si b n'est pas 1(using dafuck...)
\subsubsection{identity}
$fof^{-1}=id$
\newline
$f^{-1}of=id$
\subsubsection{binomial coeficients}
\[\frac{k(k+1)}{2}=
\begin{pmatrix}
k+1\cr
2
\end{pmatrix}
\]
\newline
\[
\begin{pmatrix}
m\cr
e
\end{pmatrix}
=\frac{m!}{e!(m-e)!}
\]
\subsubsection{dafuck}
$s_k=\sum_{n=0}^kb^n=1+b+b^2+b^3+...+b^k$
\newline
$S_k=\sum_{n=0}^kb^n=1+\sum_{n=1}^kb^n=1+b(\sum_{n=1}^kb^{n-1})=1+b(\sum_{n=0}^{k-1}b^n)=i+b(\sum_{n=0}^kb^n-b^k)$
\paragraph{}
$S_k=1+bS_k-b^{k+1}\Longrightarrow S_k-bS_k=1-b^{k+1}\Longrightarrow S_k(1-b)=1-b^{k+1}$
\newline
As long as $b\neq 1$
\newline
$S_k=\frac{1-b^{k+1}}{1-b}=\frac{k^{k+1}-1}{b-1}$
\subsubsection{transfromation de somme en fonction}
$f(x9=\sum_{n=0}^{k}X^n$
\newline
$f(x)=1+X\sum_{n=1}^kX^{n-1}=1+X(\sum_{n=0}^kX^n-X^k)=i+Xf(x)-X^{k+1}$
\newline
$f(x)(1-x)=1-x^{k+1}\Rightarrow f(x)=\frac{1-x^{k+1}}{1-x}$ for x$\neq$ L
\newline
so we have f'(x)=
\paragraph{}
$\frac{(k+1)x^k(x-1)-(x^{k+1}-1)}{(x-1)^2}=g(x)$
\newline
$\sum_{n=0}^knb^{n-1}=g(b)$
\newline
$\sum_{n=0}^knx^n=xg(x)$
\newline
$\sum_{n=0}^kn^2x^n$ il faut répéter l'opération( rajouter des x pour réduire la puissance de n)
\subsubsection{Aside}
$\sum_{n=0}^{\infty}b^n=\frac{1}{1-b},0\leq b<1$
\newline
$\frac{b^{k+1}-1}{b-1}$ si b se rapproche aussi pret que l'on veut de 1:
\newline
$\frac{(k+1)b^k}{1}=-(k+1)$
\paragraph{attention}
dans ces formules il faut faire attention a ce que b ne soit pas égal a 1 car on aurait une division par 0

\section{algorithms}
$Big O, Big \Omega, Big$ $Theta, Small$ o
\newline
\subsection{first algoritm}
Find the maximum of a sequence of n real numbers
\paragraph{input}
$a_1,a_2,a_3,...$ (n real numbers, unsorted)
\paragraph{algoritm}
$max=a_1$
\newline
for i=2 to n:
\newline
$if a_i>max$ then $max=a_i$
\paragraph{output}
max
\paragraph{the question}
How long take this algoritm. If we want to find the conplexity, the question is, what kind of operation are we doing.
\newline
Assumption: most costly operation is comparison. for our algorithm we need n-1 comparisons.
\subsection{linear searching}
\paragraph{task}
find x in sequence, if contained return position, else return 0
\paragraph{input}
$x\in R$ $a_1,a_2,...,a_n$ (real numbers, unsorted)
\paragraph{algorith}
pos=0
\newline
for 1=1 to n
\newline
if $a_1=x$ then pos =i
\paragraph{output}
pos
\subsection{binary search}
Some task as befor $a_1$ are now sorted
\paragraph{input}
$x\in R$
\newline
$a_1,a_2,a_3,...a_n$ (reals, sorted, $a_i\leq a_{i+1}, i=1$ to n-1)
\newline
i=1 left most position of search field
\newline
j=n right most position of search field
\paragraph{algorith}
while(i<j)
\newline
$m=\llcorner\frac{i+j}{2}\lrcorner$
\newline
if $x<a_i$ then j=m-1 else i=m
\newline
if $x=a_i$ then pas=i else pos=0
\paragraph{output}
pos\paragraph{complexity}
$\ulcorner log_2(n)\urcorner$
\subsection{sorting (bubble sort)}
\paragraph{task}
Output is the sorted list of the numbers:\{1,7,0,3,-2\}
\paragraph{input}
$a_1,a_2,...,a_n$ (reals, unsorted)
\paragraph{algorithm}
for i=n down to 2
\newline
place the maximum of ($a_1,...,a_i$) at position i
\paragraph{output}
\{-2,0,1,3,7\}
\paragraph{complexity}
(n-1)+(n-2)+(n-3)+...+1
\newline
$\frac{(n-1)n}{2}\equiv \frac{n^2}{2}$
\paragraph{}
we can do much better with a linear complexity
\subsection{matrix multiplication}
\includegraphics{/home/logeek04/Documents/etude/discretesS/Schem8.jpg}
We need to compute $C_{ij}$
\includegraphics{/home/logeek04/Documents/etude/discretesS/Schema9.jpg}
$y_i=A_{i1}x_i+A_{i2}x_2+...=\sum_{in}^{n}A_{ij}x_j$
\paragraph{}
we can see that:
\newline
$C_{ij}=\sum_{k=i}^{n}A_{ik}B_{kj}$
\paragraph{complexity}
$n^3$ multiplications...
\subsection{satisfiability}
n Boolean variable $x_1,x_2,x_3,...,x_n$
\newline
Given Boolean function C($x_1,x_2,...,x_n$)
\paragraph{paragraph}
Find truth assignment to the Boolean variable so that the truth value of C is T. Proove that no such assignment exists.
\paragraph{resolution's way}
we take all possible combinaitions of assignemments with a big truth table. The big O is $2^n$ and we can't do better.
\subsection{Big O}
\begin{itemize}
 \item Let f and g be funciton from R to R, we say f(x) is O(g(x)) if there exist $x_0$ and C that $|f(x)|<C|g(x)|$
\end{itemize}
for $x\geq x_0$, $x_0$ and C are colled withness.
\paragraph{exemple}
$10^5x^2$ is $O(x^2)$ 
\newline
$|10^5x^2|\leq Cx^2$ for $x>x_0$
\newline
$C=10^5$
\newline
$x_0=0$
\newline
you can forget the constant
\paragraph{exemple}
f(n)=$0.00001n^2+240n+10^{10}$
\newline
f is $O(x^2)$
\newline
Need to find $n_0,C$ so that $|f(n)|\leq Cn^2$ for $n\geq n_0$
\newline
$C=3$
\newline
$n_0=max\{240,10^{\frac{10}{2}}\}$
\paragraph{claim}
f=O(g)
\newline
h=O(k)
\newline
f+h=O(g+k),f,g,h,k$R\mapsto R$
\newline
$|f+h|\leq C|max\{g,k\}|$
\newline
$|f(x)|\leq C_{f,g}|g(x)|$ for $x>x_{fg}$
\newline
h=O(k)
\newline
$|h(x)|\leq c_{n,k}|k(x)|$ for $x>x_{fg}$
\subsection{notation}
\begin{description}
 \item O:\\{uppperbound}
 \item $\Omega$:\\{lowerbound} 
 \item $\Theta$:\\{equal growth role}
 \item o:\\{small notation}
\end{description}
\paragraph{}
f(n)=$\Omega(g(n)),\exists C>0$ and $n_0$ so that $|f(n)|\geq C|g(n)|$ for $n\geq n_0$
\newline
g=O(g):$g=\Omega(f)$
\newline
f=$\Theta(g):f=O(g)\wedge f=\Omega(f)$
\subsubsection{exemple}
$\sum_{i=0}^{d}a_in^i$ is O($n^d$)
\paragraph{proof}
$|\sum_{i=0}^{d}a_in^i|\leq\sum_{i=0}^{d}a_i|n^i\leq(\sum_{i=0}^{d}|a_i|)n^d
$\subsubsection{exemple}
$\sum_{i=0}^na_ii^d$ is O($n^{d+1}$)
\newline
$\sum_{i=0}^{n}|a_i|i^d\leq A\sum_{i=0}^nn^d\leq A(n+1)n^d\leq 2An^{d+1}$
\subsubsection{demonstration}
Assume that f(x) is =(g(x))
\newline
the question is: is log(f(x)) O(log(g(x))?
\paragraph{check}
f(x) is O(g(x))
\newline
$|f(x)|\leq C|g(x)|$ for $x\geq x_0$
\newline
$log|f(x)|\leq log(C)+log|g(x)|$ for $x\geq x_0$
\newline
$f(x)\rightarrow\infty$ when $x\rightarrow\infty$
\newline
$logf(x)\leq log(C)+log(g(x))$
\newline
$\Rightarrow log(f(x))\leq 2log(g(x))$ for $x\geq x_0$
\subsubsection{big $Omega$ notation}
f(x) is $\Omega$(g(x)) if $\exists C>0$ and $x_0$ so that:
\newline
$|f(x)|\geq C|g(x)|$ for $x\geq x_0$
\paragraph{claim}
f(x) is O(g(x))$\Leftrightarrow$ g(x) is $\Omega$(f(x))
\paragraph{proof}
Assume that f(x) is O(g(x)) ,$\exists C_1,x_0$ so that is true.
\newline
$|f(x)|\leq C|g(x)|$ for $x\geq x_0$
\newline
if C=0:
\newline
then the statement is also correctif we choose C=1
\newline
Hence without loss of generality, we can assume that $C>0$ Hence:
\newline
$|f(x)|\leq C|g(x)|$ for $x\geq x_0$ is quivalent to
\newline
$|g(x)|\geq \frac{1}{C}|f(x)|$ for $x\geq x_0$
\newline
Hence the supplies that g(x) is $\Omega(f(x))$
\begin{description}
 \item f(x) is $\Theta$(g(x)):\\{if f(x) is O(g(x)) and $\Omega(g(x))$, si there exist constant $C_1,C_2,x_0$ so that: $C_1|g(x)|\leq f(x)\leq C_2|g(x)|$}
\end{description}
\paragraph{example}
$log_{10}x$ is $\Theta(log_{e}(x))$
\newline
p(x) is $\Theta(q(x))$
\newline
$p(x)=\sum_{i=0}^dp_ix^i$ $p_d\neq0$
\newline
$q(x)=\sum_{i=0}^dq_ix^i$ $q_d\neq 0$
\subsubsection{small o notation}
We say that f(x) il o(g(x))
\newline
$\lim_{x\to \infty}\frac{|f(x)|}{|g(x)|}=0$
\paragraph{example}
$\sqrt{n}$ is o(n)
\newline
$\lim_{n\to \infty}\frac{\sqrt{n}}{n}=0$
\newline
log(n) is o$n^{\varepsilon}$) $\varepsilon>0$
\newline
$\lim_{n\to \infty}\frac{log(n)}{n^{\varepsilon}}=0$
\begin{itemize}
 \item $\Theta(1)$
 \item $\Theta(log(n))$ search odrer that
 \item $\Theta(n)$ finding a maximum, search unordered list
 \item $\Theta(n^2)$ naive sorting
 \item $\Theta(n^3)$ naive multiplication ofo matrix
 \item $\Theta(n^d)$ polynomial time
 \item $\Theta(e^n)$ expoential, SAT problem
\end{itemize}
%schema1

\section{modular arithmetic}
We say that m divides n, written $\frac{n}{m}$ if there exists an integer q so that mq=n
\paragraph{alternatively}
n has factor m, n is a multiple of m
\subsection{properties}
\begin{itemize}
 \item if $\frac{m}{a}$ and $\frac{m}{b}$, tham $\frac{m}{a+b}$
\end{itemize}
$\frac{m}{a}\Leftrightarrow mq_a=a$ $q_a\in Z$
\newline
$\frac{m}{b}\Leftrightarrow mq_b=b$ $q_b\in Z$
\newline
$\frac{m}{a+b}\Leftrightarrow mq=a+b$ $q\in Z$
\begin{itemize}
 \item if $\frac{m}{a}$ then $\forall b\in Z, \frac{m}{ab}$
\end{itemize}
$\frac{m}{a}\Leftrightarrow mq_0=a$ $q_0\in Z$
\newline
$m(q_0b)=ab$ $b\in Z$
\newline
$\frac{m}{ab}\Leftrightarrow mq=ab$ $q\in Z$
\begin{itemize}
 \item $\frac{m}{n}$ and $\frac{n}{a}$ then $\frac{m}{a}$
\end{itemize}
$\frac{m}{n}\Leftrightarrow mq_n=n$ $q_n\in Z$
\newline
$\frac{n}{a}\Leftrightarrow nq_0=a$ $q_0\in Z$
\newline
$m(q_nq_a)=nq_a=a$
\newline
$\frac{m}{a}\Leftrightarrow mq=a$ $q\in Z$
\begin{itemize}
 \item if $\frac{m}{a}$ and $\frac{m}{a}$ then $\forall \alpha,\varsigma \in Z,\frac{m}{\alpha a+\varsigma b}$
\end{itemize}
From(*)
\newline
$\frac{m}{a}\rightarrow \frac{m}{\alpha a}$
\newline
$\frac{m}{b}\rightarrow \frac{m}{\varsigma b}$
\newline
then use (**) to conclude
\subsection{euclidean division}
Let $n\in Z$ and $m\in Z$, $m\neq 0,$ then there exist unique integer q and r, $0\leq r < m,$ so that n=mq + r ((dividend)=(divisor)(quotient)+(remainder))
\newline
notation:(make sens since q and r are unique...)
\begin{itemize}
 \item n div m = q
 \item n mod m = r
\end{itemize}
\paragraph{m fixed, $m\neq 0$ }
$0\leq a mod m< m$
\newline
$o\leq b mod m< m$
\newline
$a mod m = b mod m$
\newline
a=b mod m
\newline
a is an argument to mod m
\newline
a and b are in the same equivalence class mod m
\paragraph{example}
m=6
\newline
\{-5,1,7,13,19,...\}
\newline
\{-6,0,6,12,...\}
\newline
\{-4,2,8,...\}
\paragraph{basic rules}
\begin{enumerate}
 \item a = b mod m a=b+km $k\in Z$
 \item a = b mod m, c = d mod m then a+c = b+d mod m (we can proove using (1))
 \item ac = bd mod m (same as 2)
 \item (a+b) mod m : ((a mod m)+(b mod m))mod m
 \item ab mod m: ((a mod m)(b mod m))mod m 
\end{enumerate}
a,b are in th esame equivalence classe, and c,d are together in a other equivalence classe.
\paragraph{exemple}
$17^{81} mod 5=17*17*...*17 mod 5$
\newline
$(17 mod 5)(17mod5)...(17mod5)mod5$(17 mod 5=2 or on fait 81 fois 17mod5)
\newline
$=2^{81}mod5$
\newline
$2^{10*8}2mod5$
\newline
$(-1)^8*2mod5=2mod5$
\paragraph{}
Efficient computation of $a^b$mod m. Can assume without less of generality that $0\leq a\leq m-1$
\newline
$a^bmod\\m=(amodm)^bmod\\m$
\paragraph{more efficient}
$b=\sum_{i=0}^Lb_i2^i$ $b\in\{0,1\}$
\newline
$L=\llcorner log_2(b)\lrcorner$
\paragraph{to compute}
$a^b=a^{\sum_{i=0}^Lb_i2^i}=\prod_{i=0}^L(a^{2^i})^{b_i}$
\newline
$a^{81}=a^{1+16+64}=a+a^{16}+a^{64}$
\newline
a= 17 mod 5=2 $\Rightarrow a=2: a^{16}=2^{16}((((2^2mod5)^2mod5)^2mod5)^2mod5=1$
\newline
$a^{64}=(a^{16})^4=1mod5$
\paragraph{complexity}
multiplication $\leq L$ squaring(reduction) $\Theta(L)$
\subsubsection{application}
Cryptography:(Caesar's Cipher)
\newline
plaim text(message)$\Rightarrow$ Cipher text(encrypted)
\newline
f:\{a,b,c,d,......z\}$\Rightarrow$\{0,1,...,25\} Hello$\Rightarrow$7,4,11,11,14
\newline
g:\{0,1,2,...,25\}$\Rightarrow$\{0,1,...,25\}
\newline
k:key $g_k(i)\Rightarrow(i+k)mod26$
\newline
$k\in [0,25]$
\newline
encriptation: $f^{-1}og_kof:\{a,b,c,...\}\Rightarrow\{a,b,c...\}$
\newline
$f^{-1}og_kof(Hello)=khoor$
\newline 
decryptation:$f^{-1}og_k{-1}ofof^{-1}og_kof$
\subsubsection{Pseudorandom numbers}
$X_n=aX_{n-1}+c mod m$
\newline
$a,c,m$ are appropriate integers
\newline
$x_1,x_2,x_3$ pseudorandom nubers
\paragraph{Hashing}
items that have a unique key k:$\in N$ because th ekey are listed, we can use a binary search $\Theta(log(n))$
\newline
Scheme using Hashing n items:
%schema1
\paragraph{collision}
$k_1,k_2$ such $k_1=k_2$mod m
\subsection{group}
Is a set G, together with a binary opération, so that the following properties hold, call the opéraiton *
\begin{enumerate}
 \item $\forall a,b\in G, a*b\in G$:* $GxG\mapsto G$
 \item assocoiativity: $\forall a,b,c\in G,(a*b)*c=a*(b*c)$
 \item identity exists: $\exists e\in G,$ so that $\forall a\in G,a*c=e*a=a$
 \item comuutative groupe: $\forall a,b\in G,a+b=b+a$
 \item inverse: $\forall a\in G,\exists b\in G,$ so that $a*b=e$
\end{enumerate}
\paragraph{note}
if a*b=b*a $\forall a,b\in G$ then G is commutative
\subsubsection{claim}
let $m\in Z,m+0$
\newline
$g=\{0,1,---,m-1\}$
\newline
*= + mod m
\newline
*=$\cdot$ mod m 
\paragraph{check}
\begin{enumerate}
 \item claused: mod m makes sure that result is m:\{0,1,...,m-1\}
 \item aoosciativity: 3,4,2 ((3+4mod 6)+2)mod6= 3+4+2 mod 6 (because integer with + is associative)
 \item identity: e=0 because 0+a =a+0=0 mod m
 \item inverse: m=6 a=2 $\exists b$ so that b+2=e=0
\end{enumerate}
\subsection{for the groupe}
G,$\cdot$ modm/closed,associativity,identity,commutative(it's ok)
\paragraph{Fact}
if m is a prime, then every element in G has a multiplication inverse, but if m is not a prime, then not all 
elements of G have an inverse with respect to multiplicaiton.
\paragraph{example}
m=5
\newline
G=\{0,1,2,3,4\} identity multiplication is 1
\begin{tabular}{|c|}
 1=1\cr
 \hline
 $2\cdot3=1$\cr
 \hline
 $3\cdot2$\cr
 \hline
 $4\cdot4$
\end{tabular}
\subsubsection{primes}
Primes: An integer $\geq$ 2 which can only be divided by itself or 1. (you know what it is)
\paragraph{claim}
$|set of primes|=\aleph_0$
\newline
Proof: sot of prime $\subseteq N$ Assuem that the set of primes is finite:
\newline
\{$p_1,p_2,p_3,...,p_n$\} then let $a=\prod_{i=1}^np_i+1$
\newline
if a is prim , then we fave a contradiction since $a>p_i,i=1,...,n,$ and hence $a\notin\{p_1,...,p_n\}$
\newline
a mod $p_i$=1 i=1,...,n
\newline
Hence a is not adivisible by on eof the primes $p_i$. hence a must contain a new prime, which leads to a contradiction
\paragraph{prime number theoreme}
pi(x)=$|\{n\in N_{\geq2}:$ n is prime$\}|$ (pi is a prime number counting function)
\newline
pi(10)=4(2,3,5,7)
\newline
pi(x) is O(x)
\paragraph{claim}
$pi(x)\sim\frac{x}{log_e(x)}$
\newline
ln the sense that:$\lim_{x \to \infty}\frac{pi(x)}{\frac{x}{log_e(x)}}=1$
\paragraph{trick}
$log_e(x)\sim log_2(x)-log_{10}x$, $log_2(1000)\sim 10$ (not identities but just a good estimation)
\newline
x=100 digit number
\newline
$log_2(x)\sim 330$
\newline
$log_{10}(x)\sim 100$
\newline
so the $log_e(x)\sim 230$
\newline
Pick randomly a 100 digit number integer
\newline
if we represent x in binary and it has k digits, then the probability of  a random number to be prime is $\geq \frac{1}{k}$
\paragraph{}
How to find a large prime:
\begin{enumerate}
 \item Pick a large integer at random.(difficult)
 \item Test if the choosen number is prime
 \item if yes we are done, if no go to one
\end{enumerate}
How to test if a large integer is prime?:
\newline
Given a: test if a is prime
\newline
for:i=2 to a-1 a
\newline
write a=iq+r; (euclideon division)
\newline
compute a mod i= r;
\newline
if r=0 htne a is divisible by i.
\newline
if a is not divisible by any i then it is prime
\paragraph{slightly better}
\begin{itemize}
 \item Skip evne numbers other than 2
 \item stop  at $\sqrt{a}$
\end{itemize}
\paragraph{siev eof Erathostenes}
1 2 3 4 5 6 7 8 9 ...
\newline
1 2 3 5 7 9...
\newline
after that we can skip all the multiple of 3 (but not 3), ect...
\subsubsection{prime number test}
Fermat's Little theorem:\\
if p is prim ethen aor all $a\in N$ $a^i=$a mod p
\newline
if we can find an a so that $a^p\neq$ a mod p 
\newline
then p is not prime!
\newline
check:a=2:n= 2 3 4 5 6 7 8 9 10
\newline
$2^2=$ 2 mod 2(correct)
\newline
$2^3=$ 2 mod 3(correct)
\newline
$2^4=$ 2 mod 4(does'nt pass)
\newline
$2^5=$ 2 mod 5(correct)
\newline
$2^6=$ 2 mod 6(does'nt pass)
\newline
if it pass, the number is a prime
\newline
pseudoprime with respect to a=2
\newline
with this, we know the numbers that we can avoid becouse they are not primes(the other are just pseudoprimes) ;)
\paragraph{good news}
$\exists$ integers which pass Format's test for all a; Cormichael numbers
\newline
they are many fewer pseudoprimes than primes
\paragraph{to compute this}
$2^a=$ 2 mod a
\newline
a,b$\in N$
\newline
c=$a\cdot b$ easy
\newline
For p,q prime, you get pg = a
\newline
Task: Given a, find p,q (very difficult)
\newline
f(pq)=pq(easy, but the inverse is difficult)
\begin{enumerate}
 \item gives to b a number
 \item B gets number and says if he/she wins depending on odd even
 \item Areveals page(name of person)
 \item B can verify
\end{enumerate}
\subsubsection{induction}
\begin{description}
 \item induction:\\{we are on a ladder, if we can reach step 1, and given that we can reach the step n, we can also reach step n+1 then we can reach any step o fa infinite ladder}
\end{description}
P(n) sequence od propositions:$P(n)=\sum_{i=1}^ni=\frac{n(n+1)}{2}$
\paragraph{principle of induction}
$[P(1)\wedge(\forall P(n)\rightarrow P(n+1))]\rightarrow\forall nP(n)$
\paragraph{}
P(i) is base case, base step
\paragraph{we will prove}
$P(n)=\sum_{i=1}^ni=\frac{n(n+1)}{2}$
\paragraph{}
Base case:
\newline
P(1) has truth value T
\newline
$\sum_{i=1}^1i=1$(explicit computation) is that equal to $\frac{n(n+1)}{2}$ when n =1: yes
\paragraph{}4
P(n+1)=$\sum_{i=1}^{n+1}i=\frac{(n+1)(n+2)}{2}$ want to show that this truth value T assuming that P(n) has truth value T
\newline
$\sum_{i=1}^ni+(n+1)=(P(n)=T)=\frac{n(n+1)}{2}+(n+1)=\frac{(n+1)(n+2)}{2}$
\paragraph{example}
P(n)=$\sum_{i=0}^nr^1=\frac{r^{n+1}-1}{r-1}$ $r\neq 1; n\geq 0$
\newline
Base case
\newline
P(0)=$\sum_{i=0}^ur^0=1=\frac{r-1}{r-1}=1:T$
\paragraph{}
induction: P(n)$\rightarrow$ P(n+1) $\forall n\geq 0$
\paragraph{}
P(n+1)=$\sum_{i=0}^nr^i+r^{n+1}=\frac{r^{n+1}-1}{r-1}+r^{n+1}=\frac{r^{n+2}-1}{r-1}$
\paragraph{why it this rule valid}
$P(1)\wedge(\forall nP(n)\rightarrow P(n+1))\rightarrow \forall nP(n)$
\newline
proof by contradiction:
\paragraph{}
Assume
\newline
$P(1)\wedge(\forall nP(n)\rightarrow P(n+1))\rightarrow \urcorner\forall nP(n)$
show that it is a contradiction 
\begin{enumerate}
 \item P(1) hypothesis
 \item $\forall n (P(n)\rightarrow P(n+1)$ hypothesis
 \item $\urcorner\forall n$ $P(n)$ hypothesis
 \item $\exists n$  $ \urcorner P(n)$ from 3 by logical quivalence
 \item $\urcorner P8s)$ for som es$\in N_{\geq1}$ by 4
 \item $\urcorner P(s*)$ by discussion over...
 \item $s*\neq 1$ by 1
 \item $\urcorner P(s*-1)$ by 2: contradiction since s* was suppose to be the smallest element of S
\end{enumerate}
S=\{$ s\in N_{\geq 1}:\urcorner P(s)$has a truth value T}
$|s|\geq 1$ by  3 and 4
\newline
Let s* tbe the smallest element of S
\paragraph{}
we use that $N_{\geq 1}$ is well ordered. Every subset of $N_{\geq 1}$ has a unique smallest element.
\paragraph{generalizations}
$P(b)\wedge(\forall\geq b$ $P(n)\rightarrow P(n+1))\rightarrow\forall n\geq b$ $P(N)$
\newline
Strong induction: $P(1)\wedge \forall n(\forall m\leq n$ $P(m)\rightarrow P(n+1))\rightarrow\forall n P(n)$
Which means that if it's true for all the steps until m before n, it's true 
\paragraph{proof by induction}
$\sum_{i=0}^ni^2=\frac{n(n+1)(2n+1)}{6}$
\newline
How to find or guess formula?
\newline
$f(x)=\sum_{i=0}^nx^i=\frac{x^{n+1}-1}{x-1},x\neq 1$
\newline
$\lim_{x\to 1}(xf(x))=\sum_{i=0}^ni^2x^{i-1}=\sum_{i=0}^ni^2$
\newline
$\sum_{i=0}^ni^2\sim\int_{0}^nx^2dx=\int_0^n x^2dx=\frac{x^3}{3}|_{x=0}^n=\frac{n^3}{3}$ 
\newline
the integrale is the area under the function of the sum so it's the lowerbound 
\newline
Guess;$\sum_{i=0}^ni^2=S(n)=\frac{n^3}{3}+an^2+bn+c$
\begin{itemize}
 \item S(0)=0 we can conclude that c=0
 \item S(1)=1 =$\frac{1}{3}+a+b$
 \item S(2)=5=$\frac{8}{3}+4a+2b$
\end{itemize}
if we - the 2 equation find: 3=2+2a
\newline
a=$\frac{1}{2}$, b=$\frac{1}{6}$ so we found: $n^3+\frac{1}{2}n^2+\frac{1}{6}$ which gives the formula
\subsubsection{recursions}
factorial(n)= n! for $n\geq0$ 0!=1 =$\prod_{i=1}^ni,n\geq 1$
\newline
factorial(n)=if n=0 return 1, else return n*factorial(n-1)
\paragraph{}
Fibonacci sequence:
\newline
$f(n), n\geq 0:0,1,2,3,5,8$
\newline
f(n)=f(n-1)+f(n-2),$n\geq2$
\newline
fibonacci(n)=if $n<2$ return n, else return fibonacci(n-1)+fibonacci(n-2)
\paragraph{}
power(a,e,m)=$a^emodm,m\geq1,e\geq0$
\newline
power(a,4,m)=$power(a,2,m)^2$mod m
\newline
power(a,e,m)=$power(a,\frac{e}{2},m)^2$ mod m
\newline
example:$a^{100}=(a^{50})^2=((a^{25})^2)^2=(((a^{12})^2)^2)^2$
\newline
compute: power(a,e,m) $m>1,e\geq0$: if e=0 return 1
\newline
else t=power$(a,\llcorner\frac{e}{2}\lrcorner,m)^2$mod m 
\newline
if (e is even) return t, else return t*a mod m
\paragraph{example}
power(a,5,m) e=5=0: no t=$power(a,2,m)^2$ mod m
\newline
if (e is even) NO!
\newline
else (e is odd) return t*a = $a^4*a=a^5$
\newline
power(a,2,m)
\newline
t=power$(a,1,m)^2$mod m
\newline
return t=$a^2$
\newline
power(a,1,m)
\newline
$e^2=0$ No
\newline
t=power$(a,0,m)^2$ mod m
\newline
return t*a = 1*a =a
\newline
complexity log(n)
\subsection{sorting}
\paragraph{Basic idea}
$n\geq 0$
\newline
We are given n elemenets $a-1,...,a_n(a_i\in R)$
\begin{enumerate}
 \item $n\leq 1$ then list is already sorted
 \item if $n\geq 2$ break problem into 2 parts and sort each part, call parts $L_1,L_2$, $L=L_1u^oL_2$ 
 \item create from $L_1$ and $L_2$ a marged sorted list complexity n
\end{enumerate}
\paragraph{note}
if 2 list are sorted, the complexity is n (you take the first of one list and you add it in the second,...)
\paragraph{merge sort}
Sort list L of n items if $n\leq 1$, list is already sorted
\newline
else :
\begin{enumerate}
 \item Create two subproblems $L_1$ and $L_2$ by splitting L into two sets of appoximably equal size
 \item recurse. sort $L_1$ and $L_2$
 \item merge $L_1$ and $L_2$ into a sorted list L
\end{enumerate}
\paragraph{proof of complexity}
M(n)= $nlog_2(n)$ complexity compare to $n^2$ for bubble sort,n is a power of 2, 2M($\frac{n}{2}$)
\newline
M(n)=$n+2M(\frac{n}{2}$ want to solee for M(n), and want to proove that this thing is nlog(n), with $n=2^k$
\paragraph{Base case}
n=1 $nlog_2(n)=0$
\paragraph{induction}
M(n)=n+2M($\frac{n}{2}$)
\newline
M(n)=n+2($\frac{n}{2}log_2(\frac{n}{2}))$
\newline
=n+n($log_2(n)-1$)=$nlog_2(n)$
\paragraph{quick sort}
Sort list L of n items $a_1,a_2,...,a_n$
\newline
if $n\leq 1$, then lsit already sorted 
\newline
else
\begin{enumerate}
 \item Break problem into 2 parts: $L_1\{a_{i}:a_i<a_n,i=1,...,n\},L_2\{a_i:a_1\geq a_1;i=1,...,n\}$
 \item recurse and sort $L_1$ and $L_2$ (Q(r)+Q(n-r-1) $0\leq r\leq n-1$
 \item merge $L_1$ and $L_2$ to get sorted list
\end{enumerate}
\includegraphics{/home/logeek04/Documents/etude/discretesS/Schema10.jpg}
\newline
attention le \{\} représente ou devrait être le pivot par rapport à la list, faux sur les schéma 
\paragraph{recursions}
a,b$\in Z$
\newline
gcd(a,b):if b=0 return a; else return gcd(b,amodb);
\newline
\begin{itemize}
 \item a=$\prod p_i^{\alpha_i}$
 \item b=$\prod p_i^{\rho_i}$
\end{itemize}
$p_i$ primes, $\alpha_i,rho_i\in N$
\newline
gcd(a,b)=$\prod p_i^{min(\alpha_i,\rho_i)}$ (the largest number that divides botnh a and b)
\newline
max:\{$c\in N, c|a\wedge c|b$\
\paragraph{permutations on n numbers}
n=2 (you can do 2 permutation for n=2)(1,2)(2,1)
\newline
n=3 (you can do 6 permutations)
\newline
n (you can do ! permutations)
\paragraph{}
the recustion say that you can fix the last element and permute the n-1 elements (you so it for all the elements of the groupe n)
\paragraph{}
permute(b) $a_1,...,a_b$ for $0\leq b \leq n$
\newline
if b=1 nothing to be done
\newline
else
\newline
for i=1 to b
\newline
    swap $a_i$ with $a_b$
\newline
    permute(b-1)
\newline
    swap($a_i$ with $a_b$) to replace the elements in the right order (1,2,3),(3,2,1),(1,2,3)
\paragraph{}
swapping means that (1,2,3,4) changes in (4,2,3,1)
\section{counting probability and graphs}
A set of interessting objects
\newline
$|A|$ want to determine the cardinality of A
\paragraph{example}
\begin{itemize}
 \item How many hand shakes at a party with n people
 \item How many winnig ticket are there at a sertain lotery
\end{itemize}
How many legal expression are using n parenthesis:
\begin{itemize}
 \item n=1: ()
 \item n=2: (()),()()
 \item n=3: ((())),()()(),(())(),()(()),(()())
\end{itemize}
How many trees are there on n labeled vertice
%schéma1
\newline
Nuber of complete binary trees on n+1 leaves
%schéma2
\paragraph{techniques}
\begin{itemize}
 \item combinatorial techniques
 \item algebric proofs
\end{itemize}
\subsubsection{2 simple principle}
A=$A_1uoA_2$ $A_1nA_2=\varnothing$
\begin{enumerate}
 \item $|A|=|A_1|+|A_2|$(sum rule)
 \item A=$uoA_i$ (disjoint union)
\end{enumerate}
\paragraph{1}
$|A|=\sum|A_i|$ (disjoint union)
\newline
$|A|=|A_1|+|A_2|-|A_1nA_2|$
\paragraph{2}
A=$A_1xA_2$=\{$(a_1,a_2):a_2\in A_1,a_2\in A_2$\}
\newline
$|A|=|A_1|\cdot |A_2|$
\subsubsection{a few simple example}
A=\{a,b,c,...\} English alphabet
\newline
$|A|=26$
\begin{itemize}
 \item How many strings over A of lenght 6
\end{itemize}
A=AxAxAxAxAxA=$A^6$
\newline
$|A|=|A^6|=26^6$ (product rule)
\begin{itemize}
 \item How many strings of lenght 6 starting with about a vowel V=\{a,e,i,o,u\}
\end{itemize}

$A=VxA^5$
\newline
$|A|=|V|x|A^5|=5*26^5$
\begin{itemize}
 \item How many strings of lenght 6 where fisrt and last charactere are a vowel
\end{itemize}
$A=VxA^4xV$
\newline
$|A|=|V|^2\cdot |A|^4=5^2*26^4$
\begin{itemize}
 \item How many strings of lenght 6 where first or 6th charactere are from V
\end{itemize}
$A_1=VxA^4x(A/V)$
\newline
$A_2=(A/V)xA^4xV$
\newline
$A_1uA_2$ all sequences that have a vowel at position 1 and 2 a consonant at 6 or vice versa
\newline
$A_3=VxA^4xV$
\newline
$|A|=|a_1|+|A_2|+|A_3|=5*26^4*21+21*26^4*5+5*26^4+5$
\subsubsection{pigeonhole principle}
if $k>n$ then at least one bin must contain at least 2 objects
\paragraph{slightly more general}
$\ulcorner\frac{k}{N}\urcorner$
\newline
there exists at least one bin with
\newline
$\ulcorner\frac{k}{N}\urcorner$ or more objects
\paragraph{example}
A\{1,2,3,4,5,6,7,8,9,10,11,12\}
\newline
Task: pick 8 of these 12 numbers out of this subset, how many pairs sum up to 13
\newline
Claim there are always at least 2 pairs out of the subset of 8 numbers that add to 13
%schema3
\paragraph
6 bin but 8 numbers, at lesast one bin must conatain $\ulcorner\frac{8}{6}\urcorner$ numbers
\paragraph{taks}
given p printers and c computers
\newline
$c>p$
\newline
%schéma4
\newline
How many cables do I need so that any subset of p computers con jprint aon all p printers
\paragraph{optimal solution}
%schéma5
connect p+1,..,c to every printer
\newline
connecitons: $1\leq i\leq p$:connect computer i to printer i
\newline
$p+1\leq i\leq c$: conect computer i to all printers
\newline
\#of cables required: p(c-p)+p=pc-p(p-1)
\paragraph{claim}
this  solution is optimal(using pigonehole principle)
\newline
Assume that c(c-p)+p+p-1 suffice.
\newline
$\ulcorner c -p+1-\frac{1}{p}\urcorner$
\begin{itemize}
 \item bin:printers
 \item objects: connections
\end{itemize}
By the pingeonhole principle there is at least one printer which is 
%schéma6

\subsubsection{choose r objects out of n}
does order matter?
\begin{itemize}
 \item yes: permutation
 \item no: combination
\end{itemize}
with reprtition or not?
\paragraph{4 counting problems}
\begin{enumerate}
 \item permuting without repetition: gold,silver,bronze out of 10 computer
 \item combination without repetition: 3 representation out of 10
 \item permutation with repetition: 3 digit pin code
 \item combination with repetition: how many ways there are to pick 3 cookies out of 10
\end{enumerate}
\subsubsection{permutation without repetition}
p(n,r)=$\frac{n!}{(n-r)!}$
\subsubsection{combination without repetition}
c(u,r)=$\frac{n!}{r!(n-r)!}$
\newline
\[
\begin{pmatrix}
n\cr
r
\end{pmatrix}
\]
\paragraph{note}
\[
\begin{pmatrix}
 n\cr
 r\cr
\end{pmatrix}
=
\begin{pmatrix}
 n\cr
 n-r\cr
\end{pmatrix}
\]
\subsubsection{permutation with repetition}
$n^r$
\subsubsection{combination with repetition}
f(n,r)=
\[
\begin{pmatrix}
 r+n^{-1}\cr
 r-n\cr
\end{pmatrix}
=C(r+n^{-1},n-1)
\]
\subsection{cards}
Strandart deck of cards
\begin{itemize}
 \item 52 cards
 \item 13 kinds
 \item 4 color
\end{itemize}
\paragraph
How many hands are there
\[
\begin{pmatrix}
 52\cr
 5\cr
\end{pmatrix}
=2598960
\]

\paragraph{cards} 13 kinds/4 suits
\begin{itemize}
 \item How many hands contain 5 kinds?
\end{itemize}
\[
\begin{pmatrix}
13\cr
5\cr
\end{pmatrix}
\cdot 4^5=1317888
\]
\newline
(51\% de chance)
\begin{itemize}
 \item How many hands where all cards have the same suit? (flush)
\end{itemize}
pick color, pick 5 out of this color of 13 from this color
\newline
\[
4
\begin{pmatrix}
13\cr
5\cr
\end{pmatrix}
=5148
\]
(0.2\% de chance)
\begin{itemize}
 \item How many hands with 4 cards of one kind
\end{itemize}
pick a kind, take all cards of this kind, pick any of the remaining cards
\newline
\[
\begin{pmatrix}
13\cr
1\cr
\end{pmatrix}
\cdot 1 \cdot
\begin{pmatrix}
 52-4\cr
 1\cr
\end{pmatrix}
=624
\]
\newline
(0.024\% de chance)
\subsubsection{Pascal's identity}
$0<k\leq n+1$
\newline
\[
\begin{pmatrix}
n+1\cr
k
\end{pmatrix}
=
\begin{pmatrix}
 n\cr
 k\cr
\end{pmatrix}
+
\begin{pmatrix}
 n\cr
 k-1\cr
\end{pmatrix}
\]
\paragraph{Algebric}
$\frac{(n+1)!}{k!(n+1-k)!}=?\frac{n!}{k!(n-k)!}+\frac{n!}{(k-1)!(n-k+1)!}$
\newline
if we make some modifications, we found the proposition
$
\begin{pmatrix}
n+1\cr
k\cr
\end{pmatrix}
$
this means that we pick k elements subset in a set of n+1 elements
\subsubsection{binomial thm}
$n\geq =$
\newline
$(x+y)^n=\sum_{k=0}^n\begin{pmatrix}
                      n\cr
                      k\cr
                     \end{pmatrix}
x^ky^{n-k}$
\paragraph{combinatorial}
$(x+y)(x+y)...(x+y)$
\newline
n factors
\paragraph{Algebric proof}
by induction
\newline
check that correct for n=0(1=1)
\paragraph
P(n)= binomial identity for n
\newline
$P(n)\to P(n+1)$
\paragraph
use the induction hypothesis
\newline
$\sum_{k=0}^{n}\begin{pmatrix}
                n\cr
                k\cr
               \end{pmatrix}
x^{k+1}y^{n-k}+\sum_{k=0}^n\begin{pmatrix}
                            n\cr
                            k\cr
                           \end{pmatrix}
k^ky^{n-k+1}$
\newline
$=x^{n+1}+\sum_{k=1}^n\begin{pmatrix}
                       n\cr
                       k-1\cr
                      \end{pmatrix}
x^ky^{n-k+1}+\sum_{k=1}^n\begin{pmatrix}
                          n\cr
                          k\cr
                         \end{pmatrix}
x^ky^{n-k+1}+y^{n-k+1}$
\newline
pascal's identity
\newline
$=\begin{pmatrix}
   n+1\cr
   0\cr
  \end{pmatrix}
x^{n+1}y^0+\sum_{k=1}^n\begin{pmatrix}
                        n+1\cr
                        k\cr
                       \end{pmatrix}
x^ky^{n+1-k}+\begin{pmatrix}
             n+1\cr
             k+1\cr
             \end{pmatrix}
y^{n+1}x^0=\sum_{k=0}^{n+1}
\begin{pmatrix}
n+1\cr
k
\end{pmatrix}
x^k y^{n+1-k}$
\paragraph{note}
x=y=1
\newline
$(x+y)^n=2^n=\sum_{k=0}^{n}\begin{pmatrix}
                            n\cr
                            k\cr
                           \end{pmatrix}
\cdot 1=\sum_{k=0}^n\begin{pmatrix}
                     n\cr
                     k\cr
                    \end{pmatrix}$
\newline
donc
\newline
$2^n=\sum_{k=0}^n\begin{pmatrix}
                  n\cr
                  k\cr
                 \end{pmatrix}
$
\subsubsection{Vandermonde(convolution)}
\[
\begin{pmatrix}
 n+m\cr
 r\cr
\end{pmatrix}
=\sum_{r=0}
\begin{pmatrix}
 n\cr
 r-k\cr
\end{pmatrix}
\begin{pmatrix}
 m\cr
 k\cr
\end{pmatrix}
n,m,r\in N_0
\]
\paragraph{Algebric proof}
\[
\begin{pmatrix}
n+m\cr
r\cr
\end{pmatrix}
\]
\newline
$(x+y)^{n+m}=\sum_{r=0}^{n+m}
\begin{pmatrix}
n+m\cr
r
\end{pmatrix}
x^ry^{n+m-r}
$
\newline
$=(x+y)^n(x+y)^m$ (we use twice the binomial)
\newline
...
\newline

\subsubsection{example}
$d_0$ basic captital
\newline
$d_n=d_{n-1}\cdot 1.02$ 2\% annual interest
\newline
$d_n=d_0(1.02)^n$
\subsubsection{Towers of Hanoi}
you know what it is
\newline
n disks, all of different dize
\newline
for n=1 $h_1=1$
\newline
for n=2 $h_2=3=2h_1+1$
\newline
for n=n $h_n=2h_{n-1}+1$
\subsubsection{bit strings with no 00}
n=1 0,1 (2)
\newline
n=2 01,10,11 (3)
\newline
n=3 $\urcorner$000,001,100 8-3 (5)
\newline
n=n $a_n=a_{n-a}+a_{n-2}$
\subsubsection{two more examples}
\# of n-digits integers with an even \# of zeros
\begin{itemize}
 \item n=1: 1,2,...,9  9 digits 
 \item n=2: 0(1,2,...,9) + 00
 \item n=3 (1,2,...,9)0(1,2,...,9) +0()()
\end{itemize}
$a_n=9\cdot a_{n-1}+1\cdot (10^{n-1}-a_{n-1})$
\newline
$=a_n=8a_{n-1}+10^{n-1}$
\paragraph{second example}
$x_1*x_2*...*x_{n+1}$(n operations)
\newline
\begin{itemize}
 \item $(x_1*x_2)*x_3$  $x_1*(x_2*x_3)$
\end{itemize}
$c_n$=\#of ways tjo put parentheses
\newline
\begin{itemize}
 \item $c_1 x_1*x_2$
 \item $c_2= x_1*x_2*x-3$
\end{itemize}
$C_n=C_0C_{n-1}+C_1C_{n-2}+C_2C_{n-3}+...+C_{n-1}C_0$
\subsubsection{summary of recustions}
\begin{enumerate}
 \item $d_n=1.02d_{n-1}$
 \item $d_n=1.02d_{n-1}+y$
 \item $a_n=a_{n-1}+a_{n-2}$
 \item $h_n=2h_{n-1}+1$
 \item $a_n=8a_{n-1}+10^{n-1},a_1=9$
 \item $C_n=C_0C_{n-1}+C_1C_{n-2}+C_2C_{n-3}+...+C_{n-1}C_0$
 \item $m_n=2m_{\frac{n}{2}}+n$ (mergesort)
 \item $d_n=(1.02)^{n}d_0$
 \end{enumerate}
\subsubsection{example}
$a_n=C_1a_{n-1}+C_2a_{n-2}$
\begin{itemize}
 \item $a_1,a_2$ initial conditions
 \item $C_2\neq 0$
\end{itemize}
\paragraph{Guess}
$a_n=r^n$ $r\in R$
\newline
$r^n=C_1r^{n-1}+C_2r^{n-2}$
\newline
$r^2=C_1r+C_2\Rightarrow r^2+rC_1+C_2=0$
\paragraph{2 possible situations}
\begin{enumerate}
 \item $ax^2+bx+c=0\Rightarrow \frac{-b+-\sqrt{b^2-4ac}}{2a}$
 \subitem $r_{1,2}=\frac{C_1+-\sqrt{C_1^2+4C_2}}{2}$
 \newline
 $a_n=\alpha_1r_1^n+\alpha_2r_2^n$ $\alpha_1,\alpha_2\in R$
 \newline
 we need to check that: $a_n=C_1a_{n-1}+C_2a_{n-2}$ with $a_n=\alpha_1r_1^n+\alpha_2r_2^n$
 \newline
 $\alpha_1r_1^n+\alpha_2r_2^n=C_1(\alpha_1r^{n-1}+\alpha_2r_2^{n-1})+C_2(\alpha_1r^{n-2}+\alpha_2r_2^{n-2})$
 \newline
 $\alpha_1(r_1^n-C_1r_2^{n-1}-C_2r_1^{n-2})+\alpha_2(r_2^n-C_1r_1^{n-1}-C_2r_2^{n-2})=?0$
 \newline
 $\alpha_1\cdot 0+\alpha_2\cdot 0=?0$
 \paragraph
 free to choose $\alpha_1,\alpha_2$
 \newline
 want: $a_0=\alpha_1r_1^0+\alpha_2r_2^0$
 \newline
  want: $a_1=\alpha_1r_1^1+\alpha_2r_2^1$
  \newline
  \[
  \begin{pmatrix}
   a_0\cr
   a_1\cr
  \end{pmatrix}
=\begin{pmatrix}
  1&1\cr
  r_1&r_2\cr
 \end{pmatrix}
\begin{pmatrix}
 \alpha_1\cr
 \alpha_2\cr
\end{pmatrix}
  \]
  \newline
  $|A|\neq 0$
  \newline
  $|A|=r_2-r-1\neq 0$ (they are different)
  \newline
  \[
  A^{-1}
  \begin{pmatrix}
   r_2&-1\cr
   -r_1&1\cr
  \end{pmatrix}
\]
\newline
so we have:
\newline
\[
\begin{pmatrix}
 \alpha_1\cr
 \alpha_2\cr
\end{pmatrix}
=
\begin{pmatrix}
 r_2&-1\cr
 -r_1&1\cr
\end{pmatrix}
\begin{pmatrix}
 a_0\cr
 a_1\cr
\end{pmatrix}
\]
\paragraph{so the explicit solutions are}
$\alpha_1=\frac{a_0r_2-a_1}{r_2-r_1},\alpha_2=\frac{-a_0r_1+a_1}{r_2-r_1}$
 \item $r_1=\frac{C_1}{2}r_2,(r-\frac{C_1}{2}=0)$
 \subitem $a_n=\alpha_1r_1^n+alpha_2r_2^2$ does not work in this case; not enought degrees of freedom
 \newline
 In other words:$a_n=nr_1^n$
 \paragraph{Claim}
 $a_n=nr_1^n$ is a solution
 \newline
 $a_n=C_1a_{n-1}+C_2a_{n-2}$
 \newline
 $nr_1^2=?C_1(n-1)r-1+C_2(n-2)$ (after some simplifications ;))
 \newline
 $r_1=\frac{C-1}{2}$ and $C_2=-\frac{C_1^2}{4}($in the root:$C_1^2+4C_2=0)$
 \newline
 other simplification and we find this is correct
\[
\begin{pmatrix}
 a_0\cr
 a_1\cr
\end{pmatrix}
=
\begin{pmatrix}
 1&0\cr
 r_1&r_1\cr
\end{pmatrix}
\begin{pmatrix}
\alpha_1\cr
\alpha_2\cr
\end{pmatrix}
\]
\newline
$||=r_1\neq 0$
 \end{enumerate}
\subsubsection{let's do an example}
\paragraph{Fibonnaci}
$f_0=0$ $f_1=1$ $f_n=f_{n-1}+f_{n-2}$
\newline
Look for the solutions of the form $f_n=r^n$
\newline
$r^2=r+1$
\newline
$r^2-r-1=0$ characteristic equation
\newline
$r_{1,2}=\frac{1+-\sqrt{1+4}}{2}=\frac{1+-\sqrt{5}}{2}$
\newline
$f_n=\alpha_1(\frac{1+\sqrt{5}}{2})^n+\alpha_2(\frac{1-\sqrt{5}}{2})^n$
\begin{itemize}
 \item $0=\alpha_1+\alpha_2$
 \item 1= $\alpha_1\frac{1+\sqrt{5}}{2}+\alpha_2\frac{1-\sqrt{5}}{2}$
\end{itemize}
we find with that: $\alpha_1=\frac{1}{\sqrt{5}},\alpha_2=-\frac{1}{5}$
\newline
$f_n=\frac{1}{\sqrt{5}}(\frac{1+\sqrt{5}}{2})^n-\frac{1}{\sqrt{5}}(\frac{1-\sqrt{5}}{2})^n=\Theta((\frac{1+\sqrt{5}}{2})^n)$
\subsubsection{example 2}
$de_n=4d_{n-1}-4d_{n-2}$ $d_0=d_1=1$
\newline
find solutions of the form $d_n=r^n$
\newline
$r^2=4r-4$
\newline
$r^2-4r+4=0$
\newline
$r_1=r_2=2$
\newline
$(r-2)^2=0$
\begin{itemize}
 \item $d_n=\alpha_1r_1^n+\alpha_2\cdot n\cdot r_1^n=\alpha_12^n+\alpha_2n2^n$
 \item $1=d_0=\alpha_1$
\end{itemize}
$\alpha_2=-\frac{1}{2}$
\paragraph{solution}
$d_n=2^n-\frac{1}{2}n2^n$
\paragraph{remark}
$a_n=C_1a_{n-1}+C_2a_{n-2}+...+C_ka_{n-k},a_0,..,a_{k-1}$ given $C_k\neq 0$
\newline
$a_n=r^n$
\subsubsection{inhomogeneous equations}
$h_n=2h_{n-1}+1$ (linear equation)
\newline
if we forget the (+1): $h_n=2h_{n-1}=C2^n$
\paragraph{Guess}
$h_n=p$
\newline
$\Rightarrow$p=2p+1 (p=-1)
\newline
so we found that $h_n=2^n-1$
\subsubsection{characteristic polynomial}
$r^2-r-1=0;r_{1,2}$ solutions
\newline
$a_n=\alpha r_1^n+\zeta r_2^n$
\newline
pick so that $a_0,a_1$ are correct
\subsubsection{3+4}
$h_n=2h_{n-1}+1,n\geq 1,h_0=0$
\newline
First look for any solution, irrespective of boundary condition
\paragraph{Guess}
\begin{itemize}
 \item $h_0=p$ constant
 \item $p=2p+1$ p=-1
 \item -p=1
\end{itemize}
$h_n=-1+2^nc$
\newline
$0=h_0=-1+2^0c$ $c=1$
\newline
$h_n=2^n-1$
\newline
the question: How do we guess the particular solution?
\subsubsection{generating functions}
And now to some thing completely different... generating functions
\begin{tabular}{cccccc}
$f_1$&$f_2$&$f_3$&$f_4$&$f_5$&...\cr
$0x^0$&$1x^1$&$1x^2$&$2x^3$&$3x^4$&...\cr
\end{tabular}
$0x^0$+$1x^1$+$1x^2$+$2x^3$+$3x^4$+$5x^5$+... formal power serie
\newline
no convergence here: x is just a symbol
\newline
$F(x)=x+x^2+2x^3+3x^4+5x^5+8x^6+...,F(x)=\sum_{n\geq0}f_nx^n$
\paragraph{manipulations}
$F(x)+G(x)=H(x)$ define the sum...
\newline
$\sum_{n\geq0}f_nx^n+\sum_{n\geq0}g_nx^n=\sum_{n\geq0}(f_n+g_n)x^n, (f_n+g_n)=h_n$
\begin{itemize}
 \item F+g=G+F
 \item F+G+H=F+(G+H)
 \item Neutral element E(x)=0 F+E=E+F=F
 \item F(x)+G(x)=E(x), G(x)=-F(x)
 \item $F(x)\cdot G(x)=(\sum_{n\geq0}f_nx^n)(\sum_{n\geq0}g_kx^k)=\sum_{n\geq0}(\sum_{k=0}^nf_{k}g_{n-k})x^n$
 \item $F(x)\cdot G(x)=G(x)\cdot F(x)$
 \item (FG)H=F(GH)
 \item C(x)=1 neutral element
 \item FC=CF=F
\end{itemize}
\paragraph{multiplication}
$(1+x+x^2+x^3+x^4+...)(1+x+x^2+x^3+x^4+...)=1+2x+3x^2+4x^3+...+x^n$
\paragraph{F(x) mutltiplication inverse}
$\exists g(x)$ so that $F(x)\cdot G(x)=1$
\newline
$F(x)=1+x+x^2+x^2+...$
\newline
$G(x)=\alpha+\zeta x=1-x$
\newline
$1\cdot F(x)=1+x+x^2+x^3+...$
\newline
$=xF(x)=-x-x^2-x^3-...$
\newline
so we can see that $F(x)\cdot G(x)=1$
\newline
\begin{itemize}
 \item n=0 $f_0g_0=1\Rightarrow g_0=\frac{1}{f_0}$ if $f_0\neq0$
 \item n=1 $f_0g_1+f_1g_0=0$ $g_1=\frac{-f_1g_0}{f_0}$ if $f_0\neq 0$
 \item n:$\sum_{k=0}^nf_ng_{n-k}=0$
\end{itemize}
$f_0g_n=-\sum_{k=1}^nf_kg_{n-k}$
\newline
$g_n=-\frac{1}{f_0}\sum_{k=1}^nf_kg_{n-k}$
\paragraph{well defined}
$f_0+0$
\paragraph{example}
f(x)=$1+x+x^2+x^3+...$
\newline
$f_0=1\neq 0$
\newline
G(x)=$g_0+g_1x+g_2x^2+...$
\newline
$g_0=\frac{1}{f_0}=1,g_1=-\frac{f_1g_0}{f_0}=-\frac{f_1}{f_0^2}=-1,g_2=0,g_n=0 n\geq 2$
\newline
F(x)=$1+x+x^2+x^3+...$
\newline
G(x)=1-x
\newline
$\frac{1}{F(x)}$
\newline
$\frac{1}{x-1}=1+x+x^2+x^3+...$
\newline
(Analysis:$\sum_{i=0}^{\infty}x^i=\frac{1}{1-x},|x|<1$)
\subsubsection{Tower of Hanoi Problem}
$h_nn=2h_{n-1}+1, n\geq 1,h_0=0$
\newline
$h_1x=2h_0x+1x$
\newline
$h_2x^2=2h_1x^2+1x^2$
\newline
...
\paragraph
generic term: $h_nx^n=2h_{n-1}x^n+1x^n$
\newline
$H(x)=\sum_{n\geq0}h_nx^n$
\paragraph{}
$h_n x^n=2h_{n-1}x^n+1x^n$
\newline
transformation:
\newline
$H(x)-h_0=2xH(x)+\frac{x}{1-x}$
\newline
$H(x)-h_0=2xH(x)+\frac{x}{1-x}$
\newline
H(x)(1-2x)=$h_0+\frac{x}{1-x},h_0=0$
\newline
$=\frac{x}{1-x}$
\newline
$H(x)=\frac{x}{(1-x)(1-2x)}=x\frac{1}{1-3x+2x^2}$
\subsubsection{partial fraction expansion}
$\frac{x}{(1-x)(1-2x)}=\frac{x}{1-x}+\frac{\zeta}{1-2x}$
\newline
$x=\alpha(1-2x)+\zeta(1-x)$
\newline
$x=\alpha+\zeta-x(2\alpha+\zeta)$
\newline
$2\alpha+\zeta=\alpha=-1;\zeta=1$
\newline
$\frac{1}{1-2x}=\sum_{n\geq 0}2^nx^n,\frac{1}{1-x}=\sum_{n\geq0}x^n$
\paragraph{}
$H(x)=\sum_{n\geq 0}(2^n-1)x^n=\sum_{n\geq0}h_nx^n$
\paragraph{note}
$a_n=C_0a_{n-1}+C_1a_{n-2}+...+$Inhomogenious term;$n\geq n_0$
\newline
H(x)=$\sum_{n\geq 1}h_nx^n$
\newline
$H(x)=\frac{p(x)}{q(x)}$
\newline

$\frac{1}{(x-x_1)(x-x_2)(x-x_3)...(x-x_k)}=\sum\frac{\alpha_i}{(x_{x_i})}$ $\alpha_i$ constant
\newline
$\frac{1}{(x-x_1)(x-x_2)^2}=\frac{\alpha}{x-x_1}+\frac{\alpha\beta\gamma}{(x-x_2)^2}$
\newline
$\frac{1}{1-x}=1+x+x^2+x^3+...=\sum_{n\geq0}x^n$
\newline
$\frac{1}{(1-x)^2}$  $(\frac{1}{1-x})'=\frac{1}{(1-x)^2}$
\newline
so we see that $\frac{1}{(1-x)^2}=\sum_{n\geq0}nx^{n-1}$
------------------------------------------------------------
\section{probabilities}
\paragraph{example 1}
\begin{itemize}
 \item lottery,blackjack, games
 \newline
what is the probability of winning?
\item you go to the doctor
\newline
You take a tewst a rare disease

\subitem 99\% accurate
\subitem 1 in a million has the disease
\subitem wou test positive
\subitem should you be worried?
\end{itemize}
\subsection{game show}
monty Hall problem
\includegraphics{/home/logeek04/Documents/etude/discretesS/Schema11.jpg}
\subsection{Estimating population size}
\begin{itemize}
 \item Walk around, every ten minutes, look at person closest to you
 \item memorize his face 
 \item if you have seen that person before, stop
 \item when stopped, let t be the number of people wou have seen so far 
\end{itemize}
\paragraph{estimate}
$\frac{t^2}{2}$
\newline 
[birthday paradox]
\subsection{object in k-D}
\includegraphics{/home/logeek04/Documents/etude/discretesS/Schema12.jpg}
\begin{itemize}
 \item wnat to know volume
 \item pick random points uniformly
 \item within a box that contains object (without less of generality sidelenght of box is 1)
 \item count proportion of points that are contains in the object 
\end{itemize}
estimate volume by this proportion
\subsection{consider a social network}
\begin{itemize}
 \item one person is infected
 \item interactions; probability p that the person you interact with gets infected
 \item will disease spreed to the whole population 
\end{itemize}
\includegraphics{/home/logeek04/Documents/etude/discretesS/Schema13.jpg}
d=5
\newline
$p\geq \frac{d}{2}$
\subsection{setup}
sample space $S(\Omega)$
\newline
we have the set of possible outcomes that are possible
\newline
we have the universe
\subsubsection{cards}
\begin{itemize}
 \item $S=\begin{pmatrix}
         52\cr
         5\cr
        \end{pmatrix}hands
$
\item S:a specific hand 
\item E:3 of a kind all of the same suit
\end{itemize}
Basic Axioms of Probability 
\newline
Assign numbwers from [0,1] to all events $E\subseteq C$ so that the following holds. We call those numbers probability and denote then b P(B)
\begin{enumerate}
 \item $P(\varnothing)=0$
 \item P(S)=1
 \item if $E_1,E_2\subseteq S$ and $E_1\cap E_2=\varnothing$, $P(E_1uE_2)=P(E_1)+P(E_2)$
\end{enumerate}
\subsubsection{example}
Die:
\begin{itemize}
 \item S=\{1,2,3,4,5,6\} fair die, uniform distribution
\end{itemize}
Assign to each outcome a number in [0,1], call this number p(s), so that $\sum_{s\in S} p(s)= 1$ (in thi sexample $\frac{1}{6}$ because it's a fair die)
\newline
$P(E)=\sum_{s\in S}p(s)$
\paragraph{}
E outcome is even
\newline
E=\{2,4,6\}
\newline
$P(E)=\sum_{s\in S}p(s)=\sum_{s\in \{2,4,6\}}p(s)=\frac{1}{6}\cdot|\{2,4,6\}|=\frac{3}{6}=\frac{1}{2}$
\subsubsection{distribution (for countable S)}
p:$S\to [0,1]$ so that $\sum_{s\in S}p(s)=1$
\newline
p(s) distribution (probability mass function)
\newline
$P(E)=\sum_{s\in E}p(s)$
\subsubsection{conditional probability, independence, bayes lows, low of told probability}
\paragraph{example}
die:
\newline
$p(s)=\frac{1}{6},\forall s\in S$
\newline
E: outcome is even $\{2,4,6\}$
\newline
Perhaps we have extra information that are useless. For example, we might know that outcome $\leq 3$
\newline
Given this extra information, what is know the probability of E
\paragraph{}
in general, given two event sA ad B, we can define the conditional probability of A given B, denote it by P(A/B), as
\newline
$P(A/B)=\frac{P(A\cap B)}{P(B)}$
\paragraph{note}
only make sens if P(B)=0
\paragraph{go back to our example}
$P(even/\leq 3)=\frac{P(even \cap \leq 3}{P(\leq 3)}=\frac{1\cdot \frac{1}{6}}{3\cdot \frac{1}{6}}=\frac{1}{3}$
\subsubsection{example}
6 rolls/flips of a coin
\newline
A= last of the 6 outcomes is heads
\newline
B= the first 5 outcomes are tail
\newline
$P(A)=\frac{1}{2}$
\newline
$P(A/B)=\frac{1}{2}$
\subsubsection{independence}
we say that 2 events A and B are independent if $P(A\cap B)=P(A)P(B)$
\newline
if A and B are independent then P(A/B)=$\frac{P(A\cap B)}{P(B)}=\frac{P(A)P(B)}{P(B)}=P(A)$
\subsubsection{Bayes low}
$P(A/B)\frac{P(A\cap B)}{P(B)}\Leftrightarrow P(A\cap B)=P(A/B)P(B)$
\newline
$P(B/A)=\frac{P(A\cap B)}{P(A)}\Leftrightarrow P(A\cap B)=P(B/A)P(A)$
\begin{itemize}
 \item $P(A/B)P(B)=P(B/A)P(A)$
 \item $P(A/B)=P(B/A)\frac{P(A)}{P(B)}$
\end{itemize}
\paragraph{}
you got  tested for a rare disease. that test is 99 \% accurate. You test positive. Sould you be worried:
\begin{itemize}
 \item T positive or negative
 \item D disease  yes,no
\end{itemize}
$P(D=y)=10^{-6}$
\newline
$P(T=p/D=y)=0.99$
\newline
$P(T=n/D=y)=0.01$
\newline
$P(T=p/D=n)=0.01$
\newline
$P(T=n/D=n)=0.99$
\newline
$P(D=y/T=p)=\frac{P(T=p/D=y)P(D=y)}{P(T=p)}$
\paragraph{low of told probability}
we have 2 events A and B
\newline
$A\cap (B u.  B^-)$
\newline
$P(A)=P(A\cap B)+P(A\cap B^-)$
\newline
=$P(A/B)P(B)+P(A/B^-)P(B^-)$
\paragraph{back to the example}
$P(T=p)=P(T=p/D=y)P(D=y)+P(P=p/D=n)P(D=n)\simeq 0.01$
\newline
$P(D=y/T=p)=\frac{P(T=p/D=y)P(D=y)}{P(T=p)}\simeq 10^{-4}$
\subsubsection{spam filter}
$P(opportunity/spam)=\frac{1}{10}$
\newline
P(opportunity/not spam)==$\frac{1}{100}$
\newline
P(spam)=$\frac{1}{2}$;P(not spam)=$\frac{1}{2}$
\newline
$P(opportunity)=P(oppo/spam)P(spam)+P(oppor/not spam)P(not spam)=\frac{11}{100}\frac{1}{2}$
\newline
P(spam/ contains the word opportunity)=$\frac{P(opportunity/spam)P(spam)}{P(opportunity)}=\frac{\frac{1}{10}\frac{1}{2}}{\frac{11}{100}\frac{1}{2}}=\frac{10}{11}\sim 90\%$
\subsection{axioms of probability}
$E\subseteq S,$ event
\newline
$0\leq P(E)\leq 1$
\begin{itemize}
 \item $P(\varnothing)=0$
 \item $P(S)=1$
 \item $A,B\in S,A\cap B=\varnothing$
 \newline
 P(AuB)=P(A)+P(B)
\end{itemize}
\paragraph{distribution}
countable S
\newline
For all $s\in S$, assign 
\newline
$0\leq p(s)\leq 1$
\newline
$P(E)=\sum_{s\in S}p(s),\sum_{s\in S}p(s)=1$
\newline
random variable (Notation: typically capital letters,x,y oucomes are typically small caps
\newline
expectation, linéarity of expectation
\newline
Bernoulli triots, Bernoulli distribution, geometric distribution
\subsubsection{random variable}
random variable: A random variable is a mapform the sample space to the reals $X:S\to R$
\newline
$P(X=r)= \sum_{s\in S:X(s)=r}p(s)$(X: a random variable, r: a real number)
\paragraph{example}
throw 2 die X(s) is sum of outcomes
\newline
$P(X=7)=\sum_{s=\in S:X(s)=7}\frac{6}{36}=\frac{1}{6}$
\paragraph{expectation}
Die: average value
\newline
$1\frac{1}{6}+2\frac{1}{6}+3\frac{1}{6}+4\frac{1}{6}+5\frac{1}{6}+6\frac{1}{6}$
\newline
A random variable X define on S
\newline
$E[X]=\sum_{r\in X(s)}P(X=r)\cdot r$
\newline
$=\sum_{r \in X(s)}(\sum_{s\in S: X(s)=r}p(s))\cdot r=\sum_{s\in S}X(s)p(s)$
\paragraph{other example}
Singel throw of a die: X(s)= valeu of outcome $E[X]=3.5$
\newline
two throw of a value: X= sum of outcomes. What values can X take on?
\newline

$E[X]=\sum_{r\in \{2,3,..,12\}}P(X=r)=P(X=2)+..+P(X=12)=7$
\newline
$\sum_{s\in S}p(s)X(s)=\frac{1}{36}\sum_{(i,j)}(i+j=7$
\subsubsection{linearity of expectation}
S sample space 
\newline
x,y two random variables
\paragraph{claim}
E[y+x]=E[x]+E[y]
\paragraph{proof}
$E[X+Y]=\sum_{s\in S}(X(s)+Y(s))p(s)=\sum_{s\in S}X(s)p(s)+\sum_{s\in S}Y(s)p(s)$
\newline
$=E[X]+E[Y]$
\paragraph{bask to our exemple}
X= sum of the 2 throws
\newline
$E[X]=E[X_1+X_2]=E[X_1]+E[x_2]=3.5+3.5=7$(it's a die ;) )
\newline
$X_1$= value of the first throw
\newline
$X_2$= value of the second throw
\newline
$X=X_1+X_2$
\subsubsection{Bernoulli trials}
experiment has 2 possible outcomes
\newline
\{fail, success\}, (\{0,1\}, \{f,s\}, \{-1,1\})
\newline
probability of success: p
\newline
probability of failur:1-p=q
\newline
Sample space, n such Bernoulli trials that are independent; probabilities  of success and failure do not depend on previous outcomes
\begin{itemize}
 \item S=(0,1,0,0,0)
\end{itemize}
$P(s)=P^{\#1\in S}\cdot q^{\#0\in S}$
\newline
$p((1,0,1))=p^2q=p^2(1-p)$
\newline
$\sum_{s\in S}p(s)=\sum_{k=0}^n\begin{pmatrix}
                                n\cr
                                k\cr
                               \end{pmatrix}
p^kq^{n-k}=(p+q)^n=(1)^n=1$
\subsection{bernoulli trials}
for one trial; probability of success is p continue until success, how many trials we need to wait in average
\newline
$\sum_{k\geq 0}x^k=\frac{1}{1-x}$
\newline
$\sum_{k\geq 0}kx^{k-1}=\frac{1}{(1-x)^2}$

\subsubsection{recall}
For events A and B we say that they are independent if $P(a\cap B)=P(A)P(B)$
\newline
so $P(A/B)=\frac{P(A\cap B)}{P(B)}=\frac{P(A)P(B)}{P(B)}=P(A)$
\begin{itemize}
 \item $P(A/B)=P(A)$
 \item $P(A\cap B)=P(A)P(B)$
 \item $P(A(x)\cap B(y))=P(A(x))P(B(y))$
 \item $P(X=x,Y=y)=P(X=x)P(Y=y)$
\end{itemize}
then we say that X and Y are independent 
\paragraph{example}
Toss a fair coin twice
\begin{itemize}
 \item X= \# of tail
 \item Y=\# of head
\end{itemize}
E[X+Y]=E[x]+E[Y] always true
\newline
$E[X\cdot Y]=?E[X]E[Y]=1$ (wrong in general you can see it in the following paragraph)
\newline
E[X]=1=E[Y]
\begin{itemize}
 \item X=0, Y=2 (HH)
 \item X=1, Y=1 (HT)
 \item X=1, Y=1 (TH)
 \item X=2, Y=0 (TT)
\end{itemize}
\paragraph{}
$E[X\cdot Y]=\frac{1}{4}(0+1+1+0)=\frac{1}{2}\neq 1$
\paragraph{but if X and Y are independent}
then $E[X\cdot Y]=E[X]E[Y]$
\newline
proof: $E[X\cdot Y]=\sum_{s\in S}p(s)X(s)Y(s)=\sum_{x,y}(\sum_{s\in S, X(s)=x,Y(s)=y}p(s))x\cdot y$
\subsection{MOrkov inequality}
Let X be a non-negative randoom variable with fifintie mean E[X]. Then 
\newline
$P(X\geq \alpha)\leq \frac{E[X]}{\alpha} \alpha >0$
\newline
meaningful iff $\alpha \geq E[X]$
\newline
$\alpha=kE[X]\Rightarrow P(X\geq kE[X]\leq \frac{1}{k})$
\paragraph{proof}
$E[X]=\sum_{s\in S}p(s)X(s)$
\newline
$\geq \sum_{s\in S}p(s)X(s)\geq \sum_{s\in S}p(s)\alpha=\alpha\sum_{s\in S}p(s)=\alpha P(X\geq \alpha)$
\newline
$E[X]\geq \alpha P(X\geq \alpha)$
\newline
$P(X\geq \alpha)\geq \frac{E[X]}{ \alpha}$
\paragraph{existance proof no first moment method }
Assume that X is integer, $X\in N_{\geq 0}$
\newline
$P(X\leq n)\leq \frac{E[X]}{n}$
\newline
$P(X\leq n)=1-P(X\geq n)\geq 1-\frac{E [X]}{n}\geq 0$ if $E[X]>n$




\end{document}
